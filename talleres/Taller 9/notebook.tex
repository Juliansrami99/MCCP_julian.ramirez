
% Default to the notebook output style

    


% Inherit from the specified cell style.




    
\documentclass[11pt]{article}

    
    
    \usepackage[T1]{fontenc}
    % Nicer default font (+ math font) than Computer Modern for most use cases
    \usepackage{mathpazo}

    % Basic figure setup, for now with no caption control since it's done
    % automatically by Pandoc (which extracts ![](path) syntax from Markdown).
    \usepackage{graphicx}
    % We will generate all images so they have a width \maxwidth. This means
    % that they will get their normal width if they fit onto the page, but
    % are scaled down if they would overflow the margins.
    \makeatletter
    \def\maxwidth{\ifdim\Gin@nat@width>\linewidth\linewidth
    \else\Gin@nat@width\fi}
    \makeatother
    \let\Oldincludegraphics\includegraphics
    % Set max figure width to be 80% of text width, for now hardcoded.
    \renewcommand{\includegraphics}[1]{\Oldincludegraphics[width=.8\maxwidth]{#1}}
    % Ensure that by default, figures have no caption (until we provide a
    % proper Figure object with a Caption API and a way to capture that
    % in the conversion process - todo).
    \usepackage{caption}
    \DeclareCaptionLabelFormat{nolabel}{}
    \captionsetup{labelformat=nolabel}

    \usepackage{adjustbox} % Used to constrain images to a maximum size 
    \usepackage{xcolor} % Allow colors to be defined
    \usepackage{enumerate} % Needed for markdown enumerations to work
    \usepackage{geometry} % Used to adjust the document margins
    \usepackage{amsmath} % Equations
    \usepackage{amssymb} % Equations
    \usepackage{textcomp} % defines textquotesingle
    % Hack from http://tex.stackexchange.com/a/47451/13684:
    \AtBeginDocument{%
        \def\PYZsq{\textquotesingle}% Upright quotes in Pygmentized code
    }
    \usepackage{upquote} % Upright quotes for verbatim code
    \usepackage{eurosym} % defines \euro
    \usepackage[mathletters]{ucs} % Extended unicode (utf-8) support
    \usepackage[utf8x]{inputenc} % Allow utf-8 characters in the tex document
    \usepackage{fancyvrb} % verbatim replacement that allows latex
    \usepackage{grffile} % extends the file name processing of package graphics 
                         % to support a larger range 
    % The hyperref package gives us a pdf with properly built
    % internal navigation ('pdf bookmarks' for the table of contents,
    % internal cross-reference links, web links for URLs, etc.)
    \usepackage{hyperref}
    \usepackage{longtable} % longtable support required by pandoc >1.10
    \usepackage{booktabs}  % table support for pandoc > 1.12.2
    \usepackage[inline]{enumitem} % IRkernel/repr support (it uses the enumerate* environment)
    \usepackage[normalem]{ulem} % ulem is needed to support strikethroughs (\sout)
                                % normalem makes italics be italics, not underlines
    

    
    
    % Colors for the hyperref package
    \definecolor{urlcolor}{rgb}{0,.145,.698}
    \definecolor{linkcolor}{rgb}{.71,0.21,0.01}
    \definecolor{citecolor}{rgb}{.12,.54,.11}

    % ANSI colors
    \definecolor{ansi-black}{HTML}{3E424D}
    \definecolor{ansi-black-intense}{HTML}{282C36}
    \definecolor{ansi-red}{HTML}{E75C58}
    \definecolor{ansi-red-intense}{HTML}{B22B31}
    \definecolor{ansi-green}{HTML}{00A250}
    \definecolor{ansi-green-intense}{HTML}{007427}
    \definecolor{ansi-yellow}{HTML}{DDB62B}
    \definecolor{ansi-yellow-intense}{HTML}{B27D12}
    \definecolor{ansi-blue}{HTML}{208FFB}
    \definecolor{ansi-blue-intense}{HTML}{0065CA}
    \definecolor{ansi-magenta}{HTML}{D160C4}
    \definecolor{ansi-magenta-intense}{HTML}{A03196}
    \definecolor{ansi-cyan}{HTML}{60C6C8}
    \definecolor{ansi-cyan-intense}{HTML}{258F8F}
    \definecolor{ansi-white}{HTML}{C5C1B4}
    \definecolor{ansi-white-intense}{HTML}{A1A6B2}

    % commands and environments needed by pandoc snippets
    % extracted from the output of `pandoc -s`
    \providecommand{\tightlist}{%
      \setlength{\itemsep}{0pt}\setlength{\parskip}{0pt}}
    \DefineVerbatimEnvironment{Highlighting}{Verbatim}{commandchars=\\\{\}}
    % Add ',fontsize=\small' for more characters per line
    \newenvironment{Shaded}{}{}
    \newcommand{\KeywordTok}[1]{\textcolor[rgb]{0.00,0.44,0.13}{\textbf{{#1}}}}
    \newcommand{\DataTypeTok}[1]{\textcolor[rgb]{0.56,0.13,0.00}{{#1}}}
    \newcommand{\DecValTok}[1]{\textcolor[rgb]{0.25,0.63,0.44}{{#1}}}
    \newcommand{\BaseNTok}[1]{\textcolor[rgb]{0.25,0.63,0.44}{{#1}}}
    \newcommand{\FloatTok}[1]{\textcolor[rgb]{0.25,0.63,0.44}{{#1}}}
    \newcommand{\CharTok}[1]{\textcolor[rgb]{0.25,0.44,0.63}{{#1}}}
    \newcommand{\StringTok}[1]{\textcolor[rgb]{0.25,0.44,0.63}{{#1}}}
    \newcommand{\CommentTok}[1]{\textcolor[rgb]{0.38,0.63,0.69}{\textit{{#1}}}}
    \newcommand{\OtherTok}[1]{\textcolor[rgb]{0.00,0.44,0.13}{{#1}}}
    \newcommand{\AlertTok}[1]{\textcolor[rgb]{1.00,0.00,0.00}{\textbf{{#1}}}}
    \newcommand{\FunctionTok}[1]{\textcolor[rgb]{0.02,0.16,0.49}{{#1}}}
    \newcommand{\RegionMarkerTok}[1]{{#1}}
    \newcommand{\ErrorTok}[1]{\textcolor[rgb]{1.00,0.00,0.00}{\textbf{{#1}}}}
    \newcommand{\NormalTok}[1]{{#1}}
    
    % Additional commands for more recent versions of Pandoc
    \newcommand{\ConstantTok}[1]{\textcolor[rgb]{0.53,0.00,0.00}{{#1}}}
    \newcommand{\SpecialCharTok}[1]{\textcolor[rgb]{0.25,0.44,0.63}{{#1}}}
    \newcommand{\VerbatimStringTok}[1]{\textcolor[rgb]{0.25,0.44,0.63}{{#1}}}
    \newcommand{\SpecialStringTok}[1]{\textcolor[rgb]{0.73,0.40,0.53}{{#1}}}
    \newcommand{\ImportTok}[1]{{#1}}
    \newcommand{\DocumentationTok}[1]{\textcolor[rgb]{0.73,0.13,0.13}{\textit{{#1}}}}
    \newcommand{\AnnotationTok}[1]{\textcolor[rgb]{0.38,0.63,0.69}{\textbf{\textit{{#1}}}}}
    \newcommand{\CommentVarTok}[1]{\textcolor[rgb]{0.38,0.63,0.69}{\textbf{\textit{{#1}}}}}
    \newcommand{\VariableTok}[1]{\textcolor[rgb]{0.10,0.09,0.49}{{#1}}}
    \newcommand{\ControlFlowTok}[1]{\textcolor[rgb]{0.00,0.44,0.13}{\textbf{{#1}}}}
    \newcommand{\OperatorTok}[1]{\textcolor[rgb]{0.40,0.40,0.40}{{#1}}}
    \newcommand{\BuiltInTok}[1]{{#1}}
    \newcommand{\ExtensionTok}[1]{{#1}}
    \newcommand{\PreprocessorTok}[1]{\textcolor[rgb]{0.74,0.48,0.00}{{#1}}}
    \newcommand{\AttributeTok}[1]{\textcolor[rgb]{0.49,0.56,0.16}{{#1}}}
    \newcommand{\InformationTok}[1]{\textcolor[rgb]{0.38,0.63,0.69}{\textbf{\textit{{#1}}}}}
    \newcommand{\WarningTok}[1]{\textcolor[rgb]{0.38,0.63,0.69}{\textbf{\textit{{#1}}}}}
    
    
    % Define a nice break command that doesn't care if a line doesn't already
    % exist.
    \def\br{\hspace*{\fill} \\* }
    % Math Jax compatability definitions
    \def\gt{>}
    \def\lt{<}
    % Document parameters
    \title{mcpp\_taller9\_Julian\_Ramirez}
    
    
    

    % Pygments definitions
    
\makeatletter
\def\PY@reset{\let\PY@it=\relax \let\PY@bf=\relax%
    \let\PY@ul=\relax \let\PY@tc=\relax%
    \let\PY@bc=\relax \let\PY@ff=\relax}
\def\PY@tok#1{\csname PY@tok@#1\endcsname}
\def\PY@toks#1+{\ifx\relax#1\empty\else%
    \PY@tok{#1}\expandafter\PY@toks\fi}
\def\PY@do#1{\PY@bc{\PY@tc{\PY@ul{%
    \PY@it{\PY@bf{\PY@ff{#1}}}}}}}
\def\PY#1#2{\PY@reset\PY@toks#1+\relax+\PY@do{#2}}

\expandafter\def\csname PY@tok@w\endcsname{\def\PY@tc##1{\textcolor[rgb]{0.73,0.73,0.73}{##1}}}
\expandafter\def\csname PY@tok@c\endcsname{\let\PY@it=\textit\def\PY@tc##1{\textcolor[rgb]{0.25,0.50,0.50}{##1}}}
\expandafter\def\csname PY@tok@cp\endcsname{\def\PY@tc##1{\textcolor[rgb]{0.74,0.48,0.00}{##1}}}
\expandafter\def\csname PY@tok@k\endcsname{\let\PY@bf=\textbf\def\PY@tc##1{\textcolor[rgb]{0.00,0.50,0.00}{##1}}}
\expandafter\def\csname PY@tok@kp\endcsname{\def\PY@tc##1{\textcolor[rgb]{0.00,0.50,0.00}{##1}}}
\expandafter\def\csname PY@tok@kt\endcsname{\def\PY@tc##1{\textcolor[rgb]{0.69,0.00,0.25}{##1}}}
\expandafter\def\csname PY@tok@o\endcsname{\def\PY@tc##1{\textcolor[rgb]{0.40,0.40,0.40}{##1}}}
\expandafter\def\csname PY@tok@ow\endcsname{\let\PY@bf=\textbf\def\PY@tc##1{\textcolor[rgb]{0.67,0.13,1.00}{##1}}}
\expandafter\def\csname PY@tok@nb\endcsname{\def\PY@tc##1{\textcolor[rgb]{0.00,0.50,0.00}{##1}}}
\expandafter\def\csname PY@tok@nf\endcsname{\def\PY@tc##1{\textcolor[rgb]{0.00,0.00,1.00}{##1}}}
\expandafter\def\csname PY@tok@nc\endcsname{\let\PY@bf=\textbf\def\PY@tc##1{\textcolor[rgb]{0.00,0.00,1.00}{##1}}}
\expandafter\def\csname PY@tok@nn\endcsname{\let\PY@bf=\textbf\def\PY@tc##1{\textcolor[rgb]{0.00,0.00,1.00}{##1}}}
\expandafter\def\csname PY@tok@ne\endcsname{\let\PY@bf=\textbf\def\PY@tc##1{\textcolor[rgb]{0.82,0.25,0.23}{##1}}}
\expandafter\def\csname PY@tok@nv\endcsname{\def\PY@tc##1{\textcolor[rgb]{0.10,0.09,0.49}{##1}}}
\expandafter\def\csname PY@tok@no\endcsname{\def\PY@tc##1{\textcolor[rgb]{0.53,0.00,0.00}{##1}}}
\expandafter\def\csname PY@tok@nl\endcsname{\def\PY@tc##1{\textcolor[rgb]{0.63,0.63,0.00}{##1}}}
\expandafter\def\csname PY@tok@ni\endcsname{\let\PY@bf=\textbf\def\PY@tc##1{\textcolor[rgb]{0.60,0.60,0.60}{##1}}}
\expandafter\def\csname PY@tok@na\endcsname{\def\PY@tc##1{\textcolor[rgb]{0.49,0.56,0.16}{##1}}}
\expandafter\def\csname PY@tok@nt\endcsname{\let\PY@bf=\textbf\def\PY@tc##1{\textcolor[rgb]{0.00,0.50,0.00}{##1}}}
\expandafter\def\csname PY@tok@nd\endcsname{\def\PY@tc##1{\textcolor[rgb]{0.67,0.13,1.00}{##1}}}
\expandafter\def\csname PY@tok@s\endcsname{\def\PY@tc##1{\textcolor[rgb]{0.73,0.13,0.13}{##1}}}
\expandafter\def\csname PY@tok@sd\endcsname{\let\PY@it=\textit\def\PY@tc##1{\textcolor[rgb]{0.73,0.13,0.13}{##1}}}
\expandafter\def\csname PY@tok@si\endcsname{\let\PY@bf=\textbf\def\PY@tc##1{\textcolor[rgb]{0.73,0.40,0.53}{##1}}}
\expandafter\def\csname PY@tok@se\endcsname{\let\PY@bf=\textbf\def\PY@tc##1{\textcolor[rgb]{0.73,0.40,0.13}{##1}}}
\expandafter\def\csname PY@tok@sr\endcsname{\def\PY@tc##1{\textcolor[rgb]{0.73,0.40,0.53}{##1}}}
\expandafter\def\csname PY@tok@ss\endcsname{\def\PY@tc##1{\textcolor[rgb]{0.10,0.09,0.49}{##1}}}
\expandafter\def\csname PY@tok@sx\endcsname{\def\PY@tc##1{\textcolor[rgb]{0.00,0.50,0.00}{##1}}}
\expandafter\def\csname PY@tok@m\endcsname{\def\PY@tc##1{\textcolor[rgb]{0.40,0.40,0.40}{##1}}}
\expandafter\def\csname PY@tok@gh\endcsname{\let\PY@bf=\textbf\def\PY@tc##1{\textcolor[rgb]{0.00,0.00,0.50}{##1}}}
\expandafter\def\csname PY@tok@gu\endcsname{\let\PY@bf=\textbf\def\PY@tc##1{\textcolor[rgb]{0.50,0.00,0.50}{##1}}}
\expandafter\def\csname PY@tok@gd\endcsname{\def\PY@tc##1{\textcolor[rgb]{0.63,0.00,0.00}{##1}}}
\expandafter\def\csname PY@tok@gi\endcsname{\def\PY@tc##1{\textcolor[rgb]{0.00,0.63,0.00}{##1}}}
\expandafter\def\csname PY@tok@gr\endcsname{\def\PY@tc##1{\textcolor[rgb]{1.00,0.00,0.00}{##1}}}
\expandafter\def\csname PY@tok@ge\endcsname{\let\PY@it=\textit}
\expandafter\def\csname PY@tok@gs\endcsname{\let\PY@bf=\textbf}
\expandafter\def\csname PY@tok@gp\endcsname{\let\PY@bf=\textbf\def\PY@tc##1{\textcolor[rgb]{0.00,0.00,0.50}{##1}}}
\expandafter\def\csname PY@tok@go\endcsname{\def\PY@tc##1{\textcolor[rgb]{0.53,0.53,0.53}{##1}}}
\expandafter\def\csname PY@tok@gt\endcsname{\def\PY@tc##1{\textcolor[rgb]{0.00,0.27,0.87}{##1}}}
\expandafter\def\csname PY@tok@err\endcsname{\def\PY@bc##1{\setlength{\fboxsep}{0pt}\fcolorbox[rgb]{1.00,0.00,0.00}{1,1,1}{\strut ##1}}}
\expandafter\def\csname PY@tok@kc\endcsname{\let\PY@bf=\textbf\def\PY@tc##1{\textcolor[rgb]{0.00,0.50,0.00}{##1}}}
\expandafter\def\csname PY@tok@kd\endcsname{\let\PY@bf=\textbf\def\PY@tc##1{\textcolor[rgb]{0.00,0.50,0.00}{##1}}}
\expandafter\def\csname PY@tok@kn\endcsname{\let\PY@bf=\textbf\def\PY@tc##1{\textcolor[rgb]{0.00,0.50,0.00}{##1}}}
\expandafter\def\csname PY@tok@kr\endcsname{\let\PY@bf=\textbf\def\PY@tc##1{\textcolor[rgb]{0.00,0.50,0.00}{##1}}}
\expandafter\def\csname PY@tok@bp\endcsname{\def\PY@tc##1{\textcolor[rgb]{0.00,0.50,0.00}{##1}}}
\expandafter\def\csname PY@tok@fm\endcsname{\def\PY@tc##1{\textcolor[rgb]{0.00,0.00,1.00}{##1}}}
\expandafter\def\csname PY@tok@vc\endcsname{\def\PY@tc##1{\textcolor[rgb]{0.10,0.09,0.49}{##1}}}
\expandafter\def\csname PY@tok@vg\endcsname{\def\PY@tc##1{\textcolor[rgb]{0.10,0.09,0.49}{##1}}}
\expandafter\def\csname PY@tok@vi\endcsname{\def\PY@tc##1{\textcolor[rgb]{0.10,0.09,0.49}{##1}}}
\expandafter\def\csname PY@tok@vm\endcsname{\def\PY@tc##1{\textcolor[rgb]{0.10,0.09,0.49}{##1}}}
\expandafter\def\csname PY@tok@sa\endcsname{\def\PY@tc##1{\textcolor[rgb]{0.73,0.13,0.13}{##1}}}
\expandafter\def\csname PY@tok@sb\endcsname{\def\PY@tc##1{\textcolor[rgb]{0.73,0.13,0.13}{##1}}}
\expandafter\def\csname PY@tok@sc\endcsname{\def\PY@tc##1{\textcolor[rgb]{0.73,0.13,0.13}{##1}}}
\expandafter\def\csname PY@tok@dl\endcsname{\def\PY@tc##1{\textcolor[rgb]{0.73,0.13,0.13}{##1}}}
\expandafter\def\csname PY@tok@s2\endcsname{\def\PY@tc##1{\textcolor[rgb]{0.73,0.13,0.13}{##1}}}
\expandafter\def\csname PY@tok@sh\endcsname{\def\PY@tc##1{\textcolor[rgb]{0.73,0.13,0.13}{##1}}}
\expandafter\def\csname PY@tok@s1\endcsname{\def\PY@tc##1{\textcolor[rgb]{0.73,0.13,0.13}{##1}}}
\expandafter\def\csname PY@tok@mb\endcsname{\def\PY@tc##1{\textcolor[rgb]{0.40,0.40,0.40}{##1}}}
\expandafter\def\csname PY@tok@mf\endcsname{\def\PY@tc##1{\textcolor[rgb]{0.40,0.40,0.40}{##1}}}
\expandafter\def\csname PY@tok@mh\endcsname{\def\PY@tc##1{\textcolor[rgb]{0.40,0.40,0.40}{##1}}}
\expandafter\def\csname PY@tok@mi\endcsname{\def\PY@tc##1{\textcolor[rgb]{0.40,0.40,0.40}{##1}}}
\expandafter\def\csname PY@tok@il\endcsname{\def\PY@tc##1{\textcolor[rgb]{0.40,0.40,0.40}{##1}}}
\expandafter\def\csname PY@tok@mo\endcsname{\def\PY@tc##1{\textcolor[rgb]{0.40,0.40,0.40}{##1}}}
\expandafter\def\csname PY@tok@ch\endcsname{\let\PY@it=\textit\def\PY@tc##1{\textcolor[rgb]{0.25,0.50,0.50}{##1}}}
\expandafter\def\csname PY@tok@cm\endcsname{\let\PY@it=\textit\def\PY@tc##1{\textcolor[rgb]{0.25,0.50,0.50}{##1}}}
\expandafter\def\csname PY@tok@cpf\endcsname{\let\PY@it=\textit\def\PY@tc##1{\textcolor[rgb]{0.25,0.50,0.50}{##1}}}
\expandafter\def\csname PY@tok@c1\endcsname{\let\PY@it=\textit\def\PY@tc##1{\textcolor[rgb]{0.25,0.50,0.50}{##1}}}
\expandafter\def\csname PY@tok@cs\endcsname{\let\PY@it=\textit\def\PY@tc##1{\textcolor[rgb]{0.25,0.50,0.50}{##1}}}

\def\PYZbs{\char`\\}
\def\PYZus{\char`\_}
\def\PYZob{\char`\{}
\def\PYZcb{\char`\}}
\def\PYZca{\char`\^}
\def\PYZam{\char`\&}
\def\PYZlt{\char`\<}
\def\PYZgt{\char`\>}
\def\PYZsh{\char`\#}
\def\PYZpc{\char`\%}
\def\PYZdl{\char`\$}
\def\PYZhy{\char`\-}
\def\PYZsq{\char`\'}
\def\PYZdq{\char`\"}
\def\PYZti{\char`\~}
% for compatibility with earlier versions
\def\PYZat{@}
\def\PYZlb{[}
\def\PYZrb{]}
\makeatother


    % Exact colors from NB
    \definecolor{incolor}{rgb}{0.0, 0.0, 0.5}
    \definecolor{outcolor}{rgb}{0.545, 0.0, 0.0}



    
    % Prevent overflowing lines due to hard-to-break entities
    \sloppy 
    % Setup hyperref package
    \hypersetup{
      breaklinks=true,  % so long urls are correctly broken across lines
      colorlinks=true,
      urlcolor=urlcolor,
      linkcolor=linkcolor,
      citecolor=citecolor,
      }
    % Slightly bigger margins than the latex defaults
    
    \geometry{verbose,tmargin=1in,bmargin=1in,lmargin=1in,rmargin=1in}
    
    

    \begin{document}
    
    
    \maketitle
    
    

    
    \section{Taller 9}\label{taller-9}

Métodos Computacionales para Políticas Públicas - URosario

\textbf{Entrega: viernes 1-nov-2019 11:59 PM}

    \textbf{Julián Santiago Ramírez}

julians.ramirez@urosario.edu.co

    \subsection{Instrucciones:}\label{instrucciones}

\begin{itemize}
\tightlist
\item
  Guarde una copia de este \emph{Jupyter Notebook} en su computador,
  idealmente en una carpeta destinada al material del curso.
\item
  Modifique el nombre del archivo del \emph{notebook}, agregando al
  final un guión inferior y su nombre y apellido, separados estos
  últimos por otro guión inferior. Por ejemplo, mi \emph{notebook} se
  llamaría: mcpp\_taller9\_santiago\_matallana
\item
  Marque el \emph{notebook} con su nombre y e-mail en el bloque verde
  arriba. Reemplace el texto "{[}Su nombre acá{]}" con su nombre y
  apellido. Similar para su e-mail.
\item
  Desarrolle la totalidad del taller sobre este \emph{notebook},
  insertando las celdas que sea necesario debajo de cada pregunta. Haga
  buen uso de las celdas para código y de las celdas tipo
  \emph{markdown} según el caso.
\item
  Recuerde salvar periódicamente sus avances.
\item
  Cuando termine el taller:

  \begin{enumerate}
  \def\labelenumi{\arabic{enumi}.}
  \tightlist
  \item
    Descárguelo en PDF. Si tiene algún problema con la conversión,
    descárguelo en HTML.
  \item
    Suba todos los archivos a su repositorio en GitHub, en una carpeta
    destinada exclusivamente para este taller, antes de la fecha y hora
    límites.
  \end{enumerate}
\end{itemize}

    \begin{center}\rule{0.5\linewidth}{\linethickness}\end{center}

    NLTK Book (http://www.nltk.org/book/), Exercises: - Chapter 1: 22, 26,
28 - Chapter 2: 2, 4, 11

    \section{Capítulo 1}\label{capuxedtulo-1}

    \begin{Verbatim}[commandchars=\\\{\}]
{\color{incolor}In [{\color{incolor}15}]:} \PY{k+kn}{import} \PY{n+nn}{nltk}
\end{Verbatim}


    \begin{Verbatim}[commandchars=\\\{\}]
{\color{incolor}In [{\color{incolor}6}]:} \PY{k+kn}{from} \PY{n+nn}{nltk}\PY{n+nn}{.}\PY{n+nn}{book} \PY{k}{import} \PY{o}{*} 
\end{Verbatim}


    \begin{Verbatim}[commandchars=\\\{\}]
*** Introductory Examples for the NLTK Book ***
Loading text1, {\ldots}, text9 and sent1, {\ldots}, sent9
Type the name of the text or sentence to view it.
Type: 'texts()' or 'sents()' to list the materials.
text1: Moby Dick by Herman Melville 1851
text2: Sense and Sensibility by Jane Austen 1811
text3: The Book of Genesis
text4: Inaugural Address Corpus
text5: Chat Corpus
text6: Monty Python and the Holy Grail
text7: Wall Street Journal
text8: Personals Corpus
text9: The Man Who Was Thursday by G . K . Chesterton 1908

    \end{Verbatim}

    \section{Punto 22}\label{punto-22}

Find all the four-letter words in the Chat Corpus (text5). With the help
of a frequency distribution (FreqDist), show these words in decreasing
order of frequency.

    \begin{Verbatim}[commandchars=\\\{\}]
{\color{incolor}In [{\color{incolor}9}]:} \PY{c+c1}{\PYZsh{}\PYZsh{}\PYZsh{} Buscamos las palabras que tengan tamaño 4 }
        \PY{n}{frecuencia} \PY{o}{=} \PY{n}{FreqDist}\PY{p}{(}\PY{n}{i} \PY{k}{for} \PY{n}{i} \PY{o+ow}{in} \PY{n}{text5} \PY{k}{if} \PY{n+nb}{len}\PY{p}{(}\PY{n}{i}\PY{p}{)} \PY{o}{==} \PY{l+m+mi}{4}\PY{p}{)}
        \PY{c+c1}{\PYZsh{}\PYZsh{} En este caso para verlo mas claron imprimo solo 50 de manera decreciente, porque son muchas palabras}
        \PY{n+nb}{print}\PY{p}{(}\PY{n}{frecuencia}\PY{o}{.}\PY{n}{most\PYZus{}common}\PY{p}{(}\PY{l+m+mi}{50}\PY{p}{)}\PY{p}{)}    
\end{Verbatim}


    \begin{Verbatim}[commandchars=\\\{\}]
[('JOIN', 1021), ('PART', 1016), ('that', 274), ('what', 183), ('here', 181), ('{\ldots}', 170), ('have', 164), ('like', 156), ('with', 152), ('chat', 142), ('your', 137), ('good', 130), ('just', 125), ('lmao', 107), ('know', 103), ('room', 98), ('from', 92), ('this', 86), ('well', 81), ('back', 78), ('hiya', 78), ('they', 77), ('dont', 75), ('yeah', 75), ('want', 71), ('love', 60), ('guys', 58), ('some', 58), ('been', 57), ('talk', 56), ('nice', 52), ('time', 50), ('when', 48), ('haha', 44), ('make', 44), ('girl', 43), ('need', 43), ('U122', 42), ('MODE', 41), ('will', 40), ('much', 40), ('then', 40), ('over', 39), ('work', 38), ('were', 38), ('take', 37), ('U121', 36), ('U115', 36), ('song', 36), ('even', 35)]

    \end{Verbatim}

    \section{Punto 26}\label{punto-26}

What does the following Python code do? sum(len(w) for w in text1) Can
you use it to work out the average word length of a text?

    \subsubsection{Rta:}\label{rta}

Lo que realiza el código es que recorre cada palabra del texto 1 y luego
cuenta el número de caracteres que hay en el texto 1

    \begin{Verbatim}[commandchars=\\\{\}]
{\color{incolor}In [{\color{incolor}10}]:} \PY{n}{total} \PY{o}{=} \PY{n+nb}{sum}\PY{p}{(}\PY{n+nb}{len}\PY{p}{(}\PY{n}{w}\PY{p}{)} \PY{k}{for} \PY{n}{w} \PY{o+ow}{in} \PY{n}{text1}\PY{p}{)}
         \PY{n+nb}{print}\PY{p}{(}\PY{l+s+s2}{\PYZdq{}}\PY{l+s+s2}{Cantidad de carácteres en el texto 1: }\PY{l+s+s2}{\PYZdq{}}\PY{p}{,} \PY{n}{total}\PY{p}{)}
         \PY{c+c1}{\PYZsh{}\PYZsh{} Necesito primero la cantidad de palabras que hay en el texto 1}
         \PY{n}{total\PYZus{}palabras} \PY{o}{=} \PY{n+nb}{len}\PY{p}{(}\PY{n}{text1}\PY{p}{)}
         \PY{c+c1}{\PYZsh{}\PYZsh{} Realizo el promedio}
         \PY{n}{promedio} \PY{o}{=} \PY{n}{total} \PY{o}{/} \PY{n}{total\PYZus{}palabras}
         \PY{n+nb}{print}\PY{p}{(}\PY{l+s+s2}{\PYZdq{}}\PY{l+s+s2}{El promedio del tamaño de una palabra en el texto 1 es: }\PY{l+s+s2}{\PYZdq{}}\PY{p}{,} \PY{n}{promedio}\PY{p}{)}
\end{Verbatim}


    \begin{Verbatim}[commandchars=\\\{\}]
Cantidad de carácteres en el texto 1:  999044
El promedio del tamaño de una palabra en el texto 1 es:  3.830411128023649

    \end{Verbatim}

    \section{Punto 28}\label{punto-28}

Define a function percent(word, text) that calculates how often a given
word occurs in a text, and expresses the result as a percentage.

    \begin{Verbatim}[commandchars=\\\{\}]
{\color{incolor}In [{\color{incolor}14}]:} \PY{c+c1}{\PYZsh{} Funcion percent, dos argumentos}
         \PY{k}{def} \PY{n+nf}{percent}\PY{p}{(}\PY{n}{word}\PY{p}{,} \PY{n}{text}\PY{p}{)}\PY{p}{:}
             \PY{c+c1}{\PYZsh{}\PYZsh{} primero cuento las veces que aparece la palabra en el texto}
             \PY{n}{cuenta}\PY{o}{=}\PY{n}{text}\PY{o}{.}\PY{n}{count}\PY{p}{(}\PY{n}{word}\PY{p}{)}
             \PY{c+c1}{\PYZsh{}\PYZsh{} Luego miro que tan largo es el texto}
             \PY{n}{tamaño}\PY{o}{=} \PY{n+nb}{len}\PY{p}{(}\PY{n}{text}\PY{p}{)}
             \PY{c+c1}{\PYZsh{}\PYZsh{} Porcentaje}
             \PY{n}{por}\PY{o}{=}\PY{p}{(}\PY{n}{cuenta}\PY{o}{/}\PY{n}{tamaño}\PY{p}{)}\PY{o}{*}\PY{l+m+mi}{100}
             \PY{k}{return} \PY{n}{por}
         
         \PY{c+c1}{\PYZsh{}\PYZsh{} Pruebas}
         \PY{n}{a}\PY{o}{=}\PY{n}{percent}\PY{p}{(}\PY{l+s+s1}{\PYZsq{}}\PY{l+s+s1}{God}\PY{l+s+s1}{\PYZsq{}}\PY{p}{,}\PY{n}{text3}\PY{p}{)}
         \PY{n+nb}{print}\PY{p}{(}\PY{l+s+s2}{\PYZdq{}}\PY{l+s+s2}{Porcentaje de cantidad de veces que aparece }\PY{l+s+s2}{\PYZsq{}}\PY{l+s+s2}{God}\PY{l+s+s2}{\PYZsq{}}\PY{l+s+s2}{ en el texto 3: }\PY{l+s+s2}{\PYZdq{}}\PY{p}{,}\PY{n}{a}\PY{p}{)}
         \PY{n}{b}\PY{o}{=}\PY{n}{percent}\PY{p}{(}\PY{l+s+s1}{\PYZsq{}}\PY{l+s+s1}{Finance}\PY{l+s+s1}{\PYZsq{}}\PY{p}{,}\PY{n}{text7}\PY{p}{)}
         \PY{n+nb}{print}\PY{p}{(}\PY{l+s+s2}{\PYZdq{}}\PY{l+s+s2}{Porcentaje de cantidad de veces que aparece }\PY{l+s+s2}{\PYZsq{}}\PY{l+s+s2}{Finance}\PY{l+s+s2}{\PYZsq{}}\PY{l+s+s2}{ en el texto 7: }\PY{l+s+s2}{\PYZdq{}}\PY{p}{,}\PY{n}{b}\PY{p}{)}
\end{Verbatim}


    \begin{Verbatim}[commandchars=\\\{\}]
Porcentaje de cantidad de veces que aparece 'God' en el texto 3:  0.5160396747386293
Porcentaje de cantidad de veces que aparece 'Finance' en el texto 7:  0.008939568516826254

    \end{Verbatim}

    \section{Capítulo 2}\label{capuxedtulo-2}

    \section{Punto 2}\label{punto-2}

Use the corpus module to explore austen-persuasion.txt. How many word
tokens does this book have? How many word types?

    \begin{Verbatim}[commandchars=\\\{\}]
{\color{incolor}In [{\color{incolor}21}]:} \PY{c+c1}{\PYZsh{} Traemos el archivo}
         \PY{n}{austen}\PY{o}{=}\PY{n}{nltk}\PY{o}{.}\PY{n}{corpus}\PY{o}{.}\PY{n}{gutenberg}\PY{o}{.}\PY{n}{words}\PY{p}{(}\PY{l+s+s1}{\PYZsq{}}\PY{l+s+s1}{austen\PYZhy{}persuasion.txt}\PY{l+s+s1}{\PYZsq{}}\PY{p}{)}
         \PY{n+nb}{print}\PY{p}{(}\PY{l+s+s1}{\PYZsq{}}\PY{l+s+s1}{Número de tokens:}\PY{l+s+s1}{\PYZsq{}}\PY{p}{,}\PY{n+nb}{len}\PY{p}{(}\PY{n}{austen}\PY{p}{)}\PY{p}{)}
         \PY{n+nb}{print}\PY{p}{(}\PY{l+s+s1}{\PYZsq{}}\PY{l+s+s1}{Tipos de palabras:}\PY{l+s+s1}{\PYZsq{}}\PY{p}{,}\PY{n+nb}{len}\PY{p}{(}\PY{n+nb}{set}\PY{p}{(}\PY{n}{austen}\PY{p}{)}\PY{p}{)}\PY{p}{)}
\end{Verbatim}


    \begin{Verbatim}[commandchars=\\\{\}]
Número de tokens: 98171
Tipos de palabras: 6132

    \end{Verbatim}

    \section{Punto 4}\label{punto-4}

Read in the texts of the State of the Union addresses, using the
state\_union corpus reader. Count occurrences of men, women, and people
in each document. What has happened to the usage of these words over
time?

    \begin{Verbatim}[commandchars=\\\{\}]
{\color{incolor}In [{\color{incolor}25}]:} \PY{k+kn}{from} \PY{n+nn}{nltk}\PY{n+nn}{.}\PY{n+nn}{corpus} \PY{k}{import} \PY{n}{state\PYZus{}union}
         \PY{k+kn}{import} \PY{n+nn}{matplotlib}\PY{n+nn}{.}\PY{n+nn}{pyplot} \PY{k}{as} \PY{n+nn}{plt}
\end{Verbatim}


    \begin{Verbatim}[commandchars=\\\{\}]
{\color{incolor}In [{\color{incolor}44}]:} \PY{c+c1}{\PYZsh{}\PYZsh{} Frecuencia Condicional}
         \PY{c+c1}{\PYZsh{} Palabras que queremos buscar}
         \PY{n}{lista}\PY{o}{=}\PY{p}{[}\PY{l+s+s1}{\PYZsq{}}\PY{l+s+s1}{men}\PY{l+s+s1}{\PYZsq{}}\PY{p}{,} \PY{l+s+s1}{\PYZsq{}}\PY{l+s+s1}{women}\PY{l+s+s1}{\PYZsq{}}\PY{p}{,} \PY{l+s+s1}{\PYZsq{}}\PY{l+s+s1}{people}\PY{l+s+s1}{\PYZsq{}}\PY{p}{]}
         \PY{c+c1}{\PYZsh{}\PYZsh{} Realizamos la busqueda por año}
         \PY{n}{DistF}\PY{o}{=} \PY{n}{nltk}\PY{o}{.}\PY{n}{ConditionalFreqDist}\PY{p}{(}
             \PY{p}{(}\PY{n}{i}\PY{p}{,} \PY{n}{fileid}\PY{p}{[}\PY{p}{:}\PY{l+m+mi}{4}\PY{p}{]}\PY{p}{)}
             \PY{c+c1}{\PYZsh{} Recorremos el texto y buscamos que si cada palabra coincide con las que estamos buscando}
             \PY{k}{for} \PY{n}{fileid} \PY{o+ow}{in} \PY{n}{state\PYZus{}union}\PY{o}{.}\PY{n}{fileids}\PY{p}{(}\PY{p}{)}
             \PY{k}{for} \PY{n}{j} \PY{o+ow}{in} \PY{n}{state\PYZus{}union}\PY{o}{.}\PY{n}{words}\PY{p}{(}\PY{n}{fileid}\PY{p}{)}
             \PY{k}{for} \PY{n}{i} \PY{o+ow}{in} \PY{n}{lista}
             \PY{c+c1}{\PYZsh{} En caso tal de que si coincida}
             \PY{k}{if} \PY{n}{j}\PY{o}{.}\PY{n}{lower}\PY{p}{(}\PY{p}{)}\PY{o}{.}\PY{n}{startswith}\PY{p}{(}\PY{n}{i}\PY{p}{)}\PY{p}{)}
\end{Verbatim}


    \begin{Verbatim}[commandchars=\\\{\}]
{\color{incolor}In [{\color{incolor}45}]:} \PY{n}{plt}\PY{o}{.}\PY{n}{figure}\PY{p}{(}\PY{n}{figsize}\PY{o}{=}\PY{p}{(}\PY{l+m+mi}{15}\PY{p}{,}\PY{l+m+mi}{10}\PY{p}{)}\PY{p}{)}
         \PY{n}{DistF}\PY{o}{.}\PY{n}{plot}\PY{p}{(}\PY{p}{)}
\end{Verbatim}


    \begin{center}
    \adjustimage{max size={0.9\linewidth}{0.9\paperheight}}{output_21_0.png}
    \end{center}
    { \hspace*{\fill} \\}
    
    \section{Punto 11}\label{punto-11}

Investigate the table of modal distributions and look for other
patterns. Try to explain them in terms of your own impressionistic
understanding of the different genres. Can you find other closed classes
of words that exhibit significant differences across different genres?

    Segun el texto: "The Brown Corpus is a convenient resource for studying
systematic differences between genres, a kind of linguistic inquiry
known as stylistics. Let's compare genres in their usage of modal verbs.
The first step is to produce the counts for a particular genre. Remember
to import nltk before doing the following:" Luego importamos brown

    \begin{Verbatim}[commandchars=\\\{\}]
{\color{incolor}In [{\color{incolor}47}]:} \PY{k+kn}{from} \PY{n+nn}{nltk}\PY{n+nn}{.}\PY{n+nn}{corpus} \PY{k}{import} \PY{n}{brown}
         \PY{c+c1}{\PYZsh{}\PYZsh{} Tomado del libro los modals serian}
         \PY{n}{modals} \PY{o}{=} \PY{p}{[}\PY{l+s+s1}{\PYZsq{}}\PY{l+s+s1}{can}\PY{l+s+s1}{\PYZsq{}}\PY{p}{,} \PY{l+s+s1}{\PYZsq{}}\PY{l+s+s1}{could}\PY{l+s+s1}{\PYZsq{}}\PY{p}{,} \PY{l+s+s1}{\PYZsq{}}\PY{l+s+s1}{may}\PY{l+s+s1}{\PYZsq{}}\PY{p}{,} \PY{l+s+s1}{\PYZsq{}}\PY{l+s+s1}{might}\PY{l+s+s1}{\PYZsq{}}\PY{p}{,} \PY{l+s+s1}{\PYZsq{}}\PY{l+s+s1}{must}\PY{l+s+s1}{\PYZsq{}}\PY{p}{,} \PY{l+s+s1}{\PYZsq{}}\PY{l+s+s1}{will}\PY{l+s+s1}{\PYZsq{}}\PY{p}{]}
         \PY{c+c1}{\PYZsh{}\PYZsh{} Tomado del libro}
         \PY{n}{genres} \PY{o}{=} \PY{p}{[}\PY{l+s+s1}{\PYZsq{}}\PY{l+s+s1}{news}\PY{l+s+s1}{\PYZsq{}}\PY{p}{,} \PY{l+s+s1}{\PYZsq{}}\PY{l+s+s1}{religion}\PY{l+s+s1}{\PYZsq{}}\PY{p}{,} \PY{l+s+s1}{\PYZsq{}}\PY{l+s+s1}{hobbies}\PY{l+s+s1}{\PYZsq{}}\PY{p}{,} \PY{l+s+s1}{\PYZsq{}}\PY{l+s+s1}{science\PYZus{}fiction}\PY{l+s+s1}{\PYZsq{}}\PY{p}{,} \PY{l+s+s1}{\PYZsq{}}\PY{l+s+s1}{romance}\PY{l+s+s1}{\PYZsq{}}\PY{p}{,} \PY{l+s+s1}{\PYZsq{}}\PY{l+s+s1}{humor}\PY{l+s+s1}{\PYZsq{}}\PY{p}{]}
         \PY{n}{tabla} \PY{o}{=} \PY{n}{nltk}\PY{o}{.}\PY{n}{ConditionalFreqDist}\PY{p}{(}
                     \PY{p}{(}\PY{n}{genre}\PY{p}{,} \PY{n}{word}\PY{o}{.}\PY{n}{lower}\PY{p}{(}\PY{p}{)}\PY{p}{)}                                               
                     \PY{k}{for} \PY{n}{genre} \PY{o+ow}{in} \PY{n}{brown}\PY{o}{.}\PY{n}{categories}\PY{p}{(}\PY{p}{)}
                     \PY{k}{for} \PY{n}{word} \PY{o+ow}{in} \PY{n}{brown}\PY{o}{.}\PY{n}{words}\PY{p}{(}\PY{n}{categories}\PY{o}{=}\PY{n}{genre}\PY{p}{)}\PY{p}{)}
         \PY{n}{tabla}\PY{o}{.}\PY{n}{tabulate}\PY{p}{(}\PY{n}{conditions}\PY{o}{=}\PY{n}{genres}\PY{p}{,} \PY{n}{samples}\PY{o}{=}\PY{n}{modals}\PY{p}{)}
\end{Verbatim}


    \begin{Verbatim}[commandchars=\\\{\}]
                  can could   may might  must  will 
           news    94    87    93    38    53   389 
       religion    84    59    79    12    54    72 
        hobbies   276    59   143    22    84   269 
science\_fiction    16    49     4    12     8    17 
        romance    79   195    11    51    46    49 
          humor    17    33     8     8     9    13 

    \end{Verbatim}

    \begin{itemize}
\tightlist
\item
  Humor: Es el que usa menos los diferentes modales, bajo mi impresion
  no son muy requeridos en este genero.
\item
  Hobbies: Usa en gran manera "can", "will" y "may" indica que las
  personas expresan en mayor medida que pueden o realizaran algunas
  actividades
\item
  News: Es el que usa mas la palabra will, es comun que una noticia
  estos indican que sucederá despues del acontecimiento.
\item
  Romance: Es el que usa mas la palabra "could" indicando la posible
  existencia de diferentes cosas que no son seguras que sucedan en el
  futuro.
\item
  Religion: Se usa las palabras can y may. Tiene sentido es comun que
  aquí se expresen mas oraciones con mandamientos o cosas que puede
  realizar una persona.
\end{itemize}

    \begin{Verbatim}[commandchars=\\\{\}]
{\color{incolor}In [{\color{incolor}59}]:} \PY{c+c1}{\PYZsh{} Mis palabras}
         \PY{n}{julian\PYZus{}pal} \PY{o}{=} \PY{p}{[}\PY{l+s+s1}{\PYZsq{}}\PY{l+s+s1}{but}\PY{l+s+s1}{\PYZsq{}}\PY{p}{,}\PY{l+s+s1}{\PYZsq{}}\PY{l+s+s1}{then}\PY{l+s+s1}{\PYZsq{}}\PY{p}{,}\PY{l+s+s1}{\PYZsq{}}\PY{l+s+s1}{also}\PY{l+s+s1}{\PYZsq{}}\PY{p}{,}\PY{l+s+s1}{\PYZsq{}}\PY{l+s+s1}{however}\PY{l+s+s1}{\PYZsq{}}\PY{p}{,}\PY{l+s+s1}{\PYZsq{}}\PY{l+s+s1}{thus}\PY{l+s+s1}{\PYZsq{}}\PY{p}{,}\PY{l+s+s1}{\PYZsq{}}\PY{l+s+s1}{so}\PY{l+s+s1}{\PYZsq{}}\PY{p}{]}
         \PY{c+c1}{\PYZsh{}\PYZsh{} Tomado del libro}
         \PY{n}{genres} \PY{o}{=} \PY{p}{[}\PY{l+s+s1}{\PYZsq{}}\PY{l+s+s1}{news}\PY{l+s+s1}{\PYZsq{}}\PY{p}{,} \PY{l+s+s1}{\PYZsq{}}\PY{l+s+s1}{religion}\PY{l+s+s1}{\PYZsq{}}\PY{p}{,} \PY{l+s+s1}{\PYZsq{}}\PY{l+s+s1}{hobbies}\PY{l+s+s1}{\PYZsq{}}\PY{p}{,} \PY{l+s+s1}{\PYZsq{}}\PY{l+s+s1}{science\PYZus{}fiction}\PY{l+s+s1}{\PYZsq{}}\PY{p}{,} \PY{l+s+s1}{\PYZsq{}}\PY{l+s+s1}{romance}\PY{l+s+s1}{\PYZsq{}}\PY{p}{,} \PY{l+s+s1}{\PYZsq{}}\PY{l+s+s1}{humor}\PY{l+s+s1}{\PYZsq{}}\PY{p}{]}
         \PY{n}{tabla} \PY{o}{=} \PY{n}{nltk}\PY{o}{.}\PY{n}{ConditionalFreqDist}\PY{p}{(}
                     \PY{p}{(}\PY{n}{genre}\PY{p}{,} \PY{n}{word}\PY{o}{.}\PY{n}{lower}\PY{p}{(}\PY{p}{)}\PY{p}{)}                                               
                     \PY{k}{for} \PY{n}{genre} \PY{o+ow}{in} \PY{n}{brown}\PY{o}{.}\PY{n}{categories}\PY{p}{(}\PY{p}{)}
                     \PY{k}{for} \PY{n}{word} \PY{o+ow}{in} \PY{n}{brown}\PY{o}{.}\PY{n}{words}\PY{p}{(}\PY{n}{categories}\PY{o}{=}\PY{n}{genre}\PY{p}{)}\PY{p}{)}
         \PY{n}{tabla}\PY{o}{.}\PY{n}{tabulate}\PY{p}{(}\PY{n}{conditions}\PY{o}{=}\PY{n}{genres}\PY{p}{,} \PY{n}{samples}\PY{o}{=}\PY{n}{julian\PYZus{}pal}\PY{p}{)}
\end{Verbatim}


    \begin{Verbatim}[commandchars=\\\{\}]
                    but    then    also however    thus      so 
           news     283      73     129      49      10      81 
       religion     175      59      57      19      24      99 
        hobbies     221      88     110      41      21     121 
science\_fiction      89      18       2       7       2      26 
        romance     387     143      16       2       0     192 
          humor     101      37      13      10       5      56 

    \end{Verbatim}

    En este caso use conectores. Podemos ver que en "news" y "romance" es
mas comun usar "but" seguramente porque siempre se muestran diferentes
puntos de vista. Las palabras "so" y "then" se usa mucho en hobbies y
romance indicando que dado que paso algo, esto conllevo a que les guste
algo o que dos personas se separaran. En humor tambien se exponen
diferentes puntos de vista pero con menor frecuencia

    \begin{center}\rule{0.5\linewidth}{\linethickness}\end{center}


    % Add a bibliography block to the postdoc
    
    
    
    \end{document}
