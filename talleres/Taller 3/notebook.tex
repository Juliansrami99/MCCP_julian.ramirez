
% Default to the notebook output style

    


% Inherit from the specified cell style.




    
\documentclass[11pt]{article}

    
    
    \usepackage[T1]{fontenc}
    % Nicer default font (+ math font) than Computer Modern for most use cases
    \usepackage{mathpazo}

    % Basic figure setup, for now with no caption control since it's done
    % automatically by Pandoc (which extracts ![](path) syntax from Markdown).
    \usepackage{graphicx}
    % We will generate all images so they have a width \maxwidth. This means
    % that they will get their normal width if they fit onto the page, but
    % are scaled down if they would overflow the margins.
    \makeatletter
    \def\maxwidth{\ifdim\Gin@nat@width>\linewidth\linewidth
    \else\Gin@nat@width\fi}
    \makeatother
    \let\Oldincludegraphics\includegraphics
    % Set max figure width to be 80% of text width, for now hardcoded.
    \renewcommand{\includegraphics}[1]{\Oldincludegraphics[width=.8\maxwidth]{#1}}
    % Ensure that by default, figures have no caption (until we provide a
    % proper Figure object with a Caption API and a way to capture that
    % in the conversion process - todo).
    \usepackage{caption}
    \DeclareCaptionLabelFormat{nolabel}{}
    \captionsetup{labelformat=nolabel}

    \usepackage{adjustbox} % Used to constrain images to a maximum size 
    \usepackage{xcolor} % Allow colors to be defined
    \usepackage{enumerate} % Needed for markdown enumerations to work
    \usepackage{geometry} % Used to adjust the document margins
    \usepackage{amsmath} % Equations
    \usepackage{amssymb} % Equations
    \usepackage{textcomp} % defines textquotesingle
    % Hack from http://tex.stackexchange.com/a/47451/13684:
    \AtBeginDocument{%
        \def\PYZsq{\textquotesingle}% Upright quotes in Pygmentized code
    }
    \usepackage{upquote} % Upright quotes for verbatim code
    \usepackage{eurosym} % defines \euro
    \usepackage[mathletters]{ucs} % Extended unicode (utf-8) support
    \usepackage[utf8x]{inputenc} % Allow utf-8 characters in the tex document
    \usepackage{fancyvrb} % verbatim replacement that allows latex
    \usepackage{grffile} % extends the file name processing of package graphics 
                         % to support a larger range 
    % The hyperref package gives us a pdf with properly built
    % internal navigation ('pdf bookmarks' for the table of contents,
    % internal cross-reference links, web links for URLs, etc.)
    \usepackage{hyperref}
    \usepackage{longtable} % longtable support required by pandoc >1.10
    \usepackage{booktabs}  % table support for pandoc > 1.12.2
    \usepackage[inline]{enumitem} % IRkernel/repr support (it uses the enumerate* environment)
    \usepackage[normalem]{ulem} % ulem is needed to support strikethroughs (\sout)
                                % normalem makes italics be italics, not underlines
    

    
    
    % Colors for the hyperref package
    \definecolor{urlcolor}{rgb}{0,.145,.698}
    \definecolor{linkcolor}{rgb}{.71,0.21,0.01}
    \definecolor{citecolor}{rgb}{.12,.54,.11}

    % ANSI colors
    \definecolor{ansi-black}{HTML}{3E424D}
    \definecolor{ansi-black-intense}{HTML}{282C36}
    \definecolor{ansi-red}{HTML}{E75C58}
    \definecolor{ansi-red-intense}{HTML}{B22B31}
    \definecolor{ansi-green}{HTML}{00A250}
    \definecolor{ansi-green-intense}{HTML}{007427}
    \definecolor{ansi-yellow}{HTML}{DDB62B}
    \definecolor{ansi-yellow-intense}{HTML}{B27D12}
    \definecolor{ansi-blue}{HTML}{208FFB}
    \definecolor{ansi-blue-intense}{HTML}{0065CA}
    \definecolor{ansi-magenta}{HTML}{D160C4}
    \definecolor{ansi-magenta-intense}{HTML}{A03196}
    \definecolor{ansi-cyan}{HTML}{60C6C8}
    \definecolor{ansi-cyan-intense}{HTML}{258F8F}
    \definecolor{ansi-white}{HTML}{C5C1B4}
    \definecolor{ansi-white-intense}{HTML}{A1A6B2}

    % commands and environments needed by pandoc snippets
    % extracted from the output of `pandoc -s`
    \providecommand{\tightlist}{%
      \setlength{\itemsep}{0pt}\setlength{\parskip}{0pt}}
    \DefineVerbatimEnvironment{Highlighting}{Verbatim}{commandchars=\\\{\}}
    % Add ',fontsize=\small' for more characters per line
    \newenvironment{Shaded}{}{}
    \newcommand{\KeywordTok}[1]{\textcolor[rgb]{0.00,0.44,0.13}{\textbf{{#1}}}}
    \newcommand{\DataTypeTok}[1]{\textcolor[rgb]{0.56,0.13,0.00}{{#1}}}
    \newcommand{\DecValTok}[1]{\textcolor[rgb]{0.25,0.63,0.44}{{#1}}}
    \newcommand{\BaseNTok}[1]{\textcolor[rgb]{0.25,0.63,0.44}{{#1}}}
    \newcommand{\FloatTok}[1]{\textcolor[rgb]{0.25,0.63,0.44}{{#1}}}
    \newcommand{\CharTok}[1]{\textcolor[rgb]{0.25,0.44,0.63}{{#1}}}
    \newcommand{\StringTok}[1]{\textcolor[rgb]{0.25,0.44,0.63}{{#1}}}
    \newcommand{\CommentTok}[1]{\textcolor[rgb]{0.38,0.63,0.69}{\textit{{#1}}}}
    \newcommand{\OtherTok}[1]{\textcolor[rgb]{0.00,0.44,0.13}{{#1}}}
    \newcommand{\AlertTok}[1]{\textcolor[rgb]{1.00,0.00,0.00}{\textbf{{#1}}}}
    \newcommand{\FunctionTok}[1]{\textcolor[rgb]{0.02,0.16,0.49}{{#1}}}
    \newcommand{\RegionMarkerTok}[1]{{#1}}
    \newcommand{\ErrorTok}[1]{\textcolor[rgb]{1.00,0.00,0.00}{\textbf{{#1}}}}
    \newcommand{\NormalTok}[1]{{#1}}
    
    % Additional commands for more recent versions of Pandoc
    \newcommand{\ConstantTok}[1]{\textcolor[rgb]{0.53,0.00,0.00}{{#1}}}
    \newcommand{\SpecialCharTok}[1]{\textcolor[rgb]{0.25,0.44,0.63}{{#1}}}
    \newcommand{\VerbatimStringTok}[1]{\textcolor[rgb]{0.25,0.44,0.63}{{#1}}}
    \newcommand{\SpecialStringTok}[1]{\textcolor[rgb]{0.73,0.40,0.53}{{#1}}}
    \newcommand{\ImportTok}[1]{{#1}}
    \newcommand{\DocumentationTok}[1]{\textcolor[rgb]{0.73,0.13,0.13}{\textit{{#1}}}}
    \newcommand{\AnnotationTok}[1]{\textcolor[rgb]{0.38,0.63,0.69}{\textbf{\textit{{#1}}}}}
    \newcommand{\CommentVarTok}[1]{\textcolor[rgb]{0.38,0.63,0.69}{\textbf{\textit{{#1}}}}}
    \newcommand{\VariableTok}[1]{\textcolor[rgb]{0.10,0.09,0.49}{{#1}}}
    \newcommand{\ControlFlowTok}[1]{\textcolor[rgb]{0.00,0.44,0.13}{\textbf{{#1}}}}
    \newcommand{\OperatorTok}[1]{\textcolor[rgb]{0.40,0.40,0.40}{{#1}}}
    \newcommand{\BuiltInTok}[1]{{#1}}
    \newcommand{\ExtensionTok}[1]{{#1}}
    \newcommand{\PreprocessorTok}[1]{\textcolor[rgb]{0.74,0.48,0.00}{{#1}}}
    \newcommand{\AttributeTok}[1]{\textcolor[rgb]{0.49,0.56,0.16}{{#1}}}
    \newcommand{\InformationTok}[1]{\textcolor[rgb]{0.38,0.63,0.69}{\textbf{\textit{{#1}}}}}
    \newcommand{\WarningTok}[1]{\textcolor[rgb]{0.38,0.63,0.69}{\textbf{\textit{{#1}}}}}
    
    
    % Define a nice break command that doesn't care if a line doesn't already
    % exist.
    \def\br{\hspace*{\fill} \\* }
    % Math Jax compatability definitions
    \def\gt{>}
    \def\lt{<}
    % Document parameters
    \title{mcpp\_taller3\_julian\_ramirez}
    
    
    

    % Pygments definitions
    
\makeatletter
\def\PY@reset{\let\PY@it=\relax \let\PY@bf=\relax%
    \let\PY@ul=\relax \let\PY@tc=\relax%
    \let\PY@bc=\relax \let\PY@ff=\relax}
\def\PY@tok#1{\csname PY@tok@#1\endcsname}
\def\PY@toks#1+{\ifx\relax#1\empty\else%
    \PY@tok{#1}\expandafter\PY@toks\fi}
\def\PY@do#1{\PY@bc{\PY@tc{\PY@ul{%
    \PY@it{\PY@bf{\PY@ff{#1}}}}}}}
\def\PY#1#2{\PY@reset\PY@toks#1+\relax+\PY@do{#2}}

\expandafter\def\csname PY@tok@w\endcsname{\def\PY@tc##1{\textcolor[rgb]{0.73,0.73,0.73}{##1}}}
\expandafter\def\csname PY@tok@c\endcsname{\let\PY@it=\textit\def\PY@tc##1{\textcolor[rgb]{0.25,0.50,0.50}{##1}}}
\expandafter\def\csname PY@tok@cp\endcsname{\def\PY@tc##1{\textcolor[rgb]{0.74,0.48,0.00}{##1}}}
\expandafter\def\csname PY@tok@k\endcsname{\let\PY@bf=\textbf\def\PY@tc##1{\textcolor[rgb]{0.00,0.50,0.00}{##1}}}
\expandafter\def\csname PY@tok@kp\endcsname{\def\PY@tc##1{\textcolor[rgb]{0.00,0.50,0.00}{##1}}}
\expandafter\def\csname PY@tok@kt\endcsname{\def\PY@tc##1{\textcolor[rgb]{0.69,0.00,0.25}{##1}}}
\expandafter\def\csname PY@tok@o\endcsname{\def\PY@tc##1{\textcolor[rgb]{0.40,0.40,0.40}{##1}}}
\expandafter\def\csname PY@tok@ow\endcsname{\let\PY@bf=\textbf\def\PY@tc##1{\textcolor[rgb]{0.67,0.13,1.00}{##1}}}
\expandafter\def\csname PY@tok@nb\endcsname{\def\PY@tc##1{\textcolor[rgb]{0.00,0.50,0.00}{##1}}}
\expandafter\def\csname PY@tok@nf\endcsname{\def\PY@tc##1{\textcolor[rgb]{0.00,0.00,1.00}{##1}}}
\expandafter\def\csname PY@tok@nc\endcsname{\let\PY@bf=\textbf\def\PY@tc##1{\textcolor[rgb]{0.00,0.00,1.00}{##1}}}
\expandafter\def\csname PY@tok@nn\endcsname{\let\PY@bf=\textbf\def\PY@tc##1{\textcolor[rgb]{0.00,0.00,1.00}{##1}}}
\expandafter\def\csname PY@tok@ne\endcsname{\let\PY@bf=\textbf\def\PY@tc##1{\textcolor[rgb]{0.82,0.25,0.23}{##1}}}
\expandafter\def\csname PY@tok@nv\endcsname{\def\PY@tc##1{\textcolor[rgb]{0.10,0.09,0.49}{##1}}}
\expandafter\def\csname PY@tok@no\endcsname{\def\PY@tc##1{\textcolor[rgb]{0.53,0.00,0.00}{##1}}}
\expandafter\def\csname PY@tok@nl\endcsname{\def\PY@tc##1{\textcolor[rgb]{0.63,0.63,0.00}{##1}}}
\expandafter\def\csname PY@tok@ni\endcsname{\let\PY@bf=\textbf\def\PY@tc##1{\textcolor[rgb]{0.60,0.60,0.60}{##1}}}
\expandafter\def\csname PY@tok@na\endcsname{\def\PY@tc##1{\textcolor[rgb]{0.49,0.56,0.16}{##1}}}
\expandafter\def\csname PY@tok@nt\endcsname{\let\PY@bf=\textbf\def\PY@tc##1{\textcolor[rgb]{0.00,0.50,0.00}{##1}}}
\expandafter\def\csname PY@tok@nd\endcsname{\def\PY@tc##1{\textcolor[rgb]{0.67,0.13,1.00}{##1}}}
\expandafter\def\csname PY@tok@s\endcsname{\def\PY@tc##1{\textcolor[rgb]{0.73,0.13,0.13}{##1}}}
\expandafter\def\csname PY@tok@sd\endcsname{\let\PY@it=\textit\def\PY@tc##1{\textcolor[rgb]{0.73,0.13,0.13}{##1}}}
\expandafter\def\csname PY@tok@si\endcsname{\let\PY@bf=\textbf\def\PY@tc##1{\textcolor[rgb]{0.73,0.40,0.53}{##1}}}
\expandafter\def\csname PY@tok@se\endcsname{\let\PY@bf=\textbf\def\PY@tc##1{\textcolor[rgb]{0.73,0.40,0.13}{##1}}}
\expandafter\def\csname PY@tok@sr\endcsname{\def\PY@tc##1{\textcolor[rgb]{0.73,0.40,0.53}{##1}}}
\expandafter\def\csname PY@tok@ss\endcsname{\def\PY@tc##1{\textcolor[rgb]{0.10,0.09,0.49}{##1}}}
\expandafter\def\csname PY@tok@sx\endcsname{\def\PY@tc##1{\textcolor[rgb]{0.00,0.50,0.00}{##1}}}
\expandafter\def\csname PY@tok@m\endcsname{\def\PY@tc##1{\textcolor[rgb]{0.40,0.40,0.40}{##1}}}
\expandafter\def\csname PY@tok@gh\endcsname{\let\PY@bf=\textbf\def\PY@tc##1{\textcolor[rgb]{0.00,0.00,0.50}{##1}}}
\expandafter\def\csname PY@tok@gu\endcsname{\let\PY@bf=\textbf\def\PY@tc##1{\textcolor[rgb]{0.50,0.00,0.50}{##1}}}
\expandafter\def\csname PY@tok@gd\endcsname{\def\PY@tc##1{\textcolor[rgb]{0.63,0.00,0.00}{##1}}}
\expandafter\def\csname PY@tok@gi\endcsname{\def\PY@tc##1{\textcolor[rgb]{0.00,0.63,0.00}{##1}}}
\expandafter\def\csname PY@tok@gr\endcsname{\def\PY@tc##1{\textcolor[rgb]{1.00,0.00,0.00}{##1}}}
\expandafter\def\csname PY@tok@ge\endcsname{\let\PY@it=\textit}
\expandafter\def\csname PY@tok@gs\endcsname{\let\PY@bf=\textbf}
\expandafter\def\csname PY@tok@gp\endcsname{\let\PY@bf=\textbf\def\PY@tc##1{\textcolor[rgb]{0.00,0.00,0.50}{##1}}}
\expandafter\def\csname PY@tok@go\endcsname{\def\PY@tc##1{\textcolor[rgb]{0.53,0.53,0.53}{##1}}}
\expandafter\def\csname PY@tok@gt\endcsname{\def\PY@tc##1{\textcolor[rgb]{0.00,0.27,0.87}{##1}}}
\expandafter\def\csname PY@tok@err\endcsname{\def\PY@bc##1{\setlength{\fboxsep}{0pt}\fcolorbox[rgb]{1.00,0.00,0.00}{1,1,1}{\strut ##1}}}
\expandafter\def\csname PY@tok@kc\endcsname{\let\PY@bf=\textbf\def\PY@tc##1{\textcolor[rgb]{0.00,0.50,0.00}{##1}}}
\expandafter\def\csname PY@tok@kd\endcsname{\let\PY@bf=\textbf\def\PY@tc##1{\textcolor[rgb]{0.00,0.50,0.00}{##1}}}
\expandafter\def\csname PY@tok@kn\endcsname{\let\PY@bf=\textbf\def\PY@tc##1{\textcolor[rgb]{0.00,0.50,0.00}{##1}}}
\expandafter\def\csname PY@tok@kr\endcsname{\let\PY@bf=\textbf\def\PY@tc##1{\textcolor[rgb]{0.00,0.50,0.00}{##1}}}
\expandafter\def\csname PY@tok@bp\endcsname{\def\PY@tc##1{\textcolor[rgb]{0.00,0.50,0.00}{##1}}}
\expandafter\def\csname PY@tok@fm\endcsname{\def\PY@tc##1{\textcolor[rgb]{0.00,0.00,1.00}{##1}}}
\expandafter\def\csname PY@tok@vc\endcsname{\def\PY@tc##1{\textcolor[rgb]{0.10,0.09,0.49}{##1}}}
\expandafter\def\csname PY@tok@vg\endcsname{\def\PY@tc##1{\textcolor[rgb]{0.10,0.09,0.49}{##1}}}
\expandafter\def\csname PY@tok@vi\endcsname{\def\PY@tc##1{\textcolor[rgb]{0.10,0.09,0.49}{##1}}}
\expandafter\def\csname PY@tok@vm\endcsname{\def\PY@tc##1{\textcolor[rgb]{0.10,0.09,0.49}{##1}}}
\expandafter\def\csname PY@tok@sa\endcsname{\def\PY@tc##1{\textcolor[rgb]{0.73,0.13,0.13}{##1}}}
\expandafter\def\csname PY@tok@sb\endcsname{\def\PY@tc##1{\textcolor[rgb]{0.73,0.13,0.13}{##1}}}
\expandafter\def\csname PY@tok@sc\endcsname{\def\PY@tc##1{\textcolor[rgb]{0.73,0.13,0.13}{##1}}}
\expandafter\def\csname PY@tok@dl\endcsname{\def\PY@tc##1{\textcolor[rgb]{0.73,0.13,0.13}{##1}}}
\expandafter\def\csname PY@tok@s2\endcsname{\def\PY@tc##1{\textcolor[rgb]{0.73,0.13,0.13}{##1}}}
\expandafter\def\csname PY@tok@sh\endcsname{\def\PY@tc##1{\textcolor[rgb]{0.73,0.13,0.13}{##1}}}
\expandafter\def\csname PY@tok@s1\endcsname{\def\PY@tc##1{\textcolor[rgb]{0.73,0.13,0.13}{##1}}}
\expandafter\def\csname PY@tok@mb\endcsname{\def\PY@tc##1{\textcolor[rgb]{0.40,0.40,0.40}{##1}}}
\expandafter\def\csname PY@tok@mf\endcsname{\def\PY@tc##1{\textcolor[rgb]{0.40,0.40,0.40}{##1}}}
\expandafter\def\csname PY@tok@mh\endcsname{\def\PY@tc##1{\textcolor[rgb]{0.40,0.40,0.40}{##1}}}
\expandafter\def\csname PY@tok@mi\endcsname{\def\PY@tc##1{\textcolor[rgb]{0.40,0.40,0.40}{##1}}}
\expandafter\def\csname PY@tok@il\endcsname{\def\PY@tc##1{\textcolor[rgb]{0.40,0.40,0.40}{##1}}}
\expandafter\def\csname PY@tok@mo\endcsname{\def\PY@tc##1{\textcolor[rgb]{0.40,0.40,0.40}{##1}}}
\expandafter\def\csname PY@tok@ch\endcsname{\let\PY@it=\textit\def\PY@tc##1{\textcolor[rgb]{0.25,0.50,0.50}{##1}}}
\expandafter\def\csname PY@tok@cm\endcsname{\let\PY@it=\textit\def\PY@tc##1{\textcolor[rgb]{0.25,0.50,0.50}{##1}}}
\expandafter\def\csname PY@tok@cpf\endcsname{\let\PY@it=\textit\def\PY@tc##1{\textcolor[rgb]{0.25,0.50,0.50}{##1}}}
\expandafter\def\csname PY@tok@c1\endcsname{\let\PY@it=\textit\def\PY@tc##1{\textcolor[rgb]{0.25,0.50,0.50}{##1}}}
\expandafter\def\csname PY@tok@cs\endcsname{\let\PY@it=\textit\def\PY@tc##1{\textcolor[rgb]{0.25,0.50,0.50}{##1}}}

\def\PYZbs{\char`\\}
\def\PYZus{\char`\_}
\def\PYZob{\char`\{}
\def\PYZcb{\char`\}}
\def\PYZca{\char`\^}
\def\PYZam{\char`\&}
\def\PYZlt{\char`\<}
\def\PYZgt{\char`\>}
\def\PYZsh{\char`\#}
\def\PYZpc{\char`\%}
\def\PYZdl{\char`\$}
\def\PYZhy{\char`\-}
\def\PYZsq{\char`\'}
\def\PYZdq{\char`\"}
\def\PYZti{\char`\~}
% for compatibility with earlier versions
\def\PYZat{@}
\def\PYZlb{[}
\def\PYZrb{]}
\makeatother


    % Exact colors from NB
    \definecolor{incolor}{rgb}{0.0, 0.0, 0.5}
    \definecolor{outcolor}{rgb}{0.545, 0.0, 0.0}



    
    % Prevent overflowing lines due to hard-to-break entities
    \sloppy 
    % Setup hyperref package
    \hypersetup{
      breaklinks=true,  % so long urls are correctly broken across lines
      colorlinks=true,
      urlcolor=urlcolor,
      linkcolor=linkcolor,
      citecolor=citecolor,
      }
    % Slightly bigger margins than the latex defaults
    
    \geometry{verbose,tmargin=1in,bmargin=1in,lmargin=1in,rmargin=1in}
    
    

    \begin{document}
    
    
    \maketitle
    
    

    
    \section{Taller 3}\label{taller-3}

Métodos Computacionales para Políticas Públicas - URosario

\textbf{Entrega: viernes 23-ago-2019 11:59 PM}

    \textbf{Julián Santiago Ramírez} julians.ramirez@urosario.edu.co

    \subsection{Instrucciones:}\label{instrucciones}

\begin{itemize}
\tightlist
\item
  Guarde una copia de este \emph{Jupyter Notebook} en su computador,
  idealmente en una carpeta destinada al material del curso.
\item
  Modifique el nombre del archivo del \emph{notebook}, agregando al
  final un guión inferior y su nombre y apellido, separados estos
  últimos por otro guión inferior. Por ejemplo, mi \emph{notebook} se
  llamaría: mcpp\_taller3\_santiago\_matallana
\item
  Marque el \emph{notebook} con su nombre y e-mail en el bloque verde
  arriba. Reemplace el texto "{[}Su nombre acá{]}" con su nombre y
  apellido. Similar para su e-mail.
\item
  Desarrolle la totalidad del taller sobre este \emph{notebook},
  insertando las celdas que sea necesario debajo de cada pregunta. Haga
  buen uso de las celdas para código y de las celdas tipo
  \emph{markdown} según el caso.
\item
  Recuerde salvar periódicamente sus avances.
\item
  Cuando termine el taller:

  \begin{enumerate}
  \def\labelenumi{\arabic{enumi}.}
  \tightlist
  \item
    Descárguelo en PDF.
  \item
    Suba los dos archivos (.pdf y .ipynb) a su repositorio en GitHub
    antes de la fecha y hora límites.
  \end{enumerate}
\end{itemize}

(El valor de cada ejercicio está en corchetes {[} {]} después del número
de ejercicio.)

    \begin{center}\rule{0.5\linewidth}{\linethickness}\end{center}

    Antes de iniciar, por favor descarge el archivo
2019\_2\_mcpp\_taller\_3\_listas\_ejemplos.py del repositorio, guárdelo
en la misma carpeta en la que está trabajando este taller y ejecútelo
con el siguiente comando:

    \section{run
2019\_2\_mcpp\_taller\_3\_listas\_ejemplos.py}\label{run-2019_2_mcpp_taller_3_listas_ejemplos.py}

    \begin{Verbatim}[commandchars=\\\{\}]
{\color{incolor}In [{\color{incolor}3}]:} \PY{n}{run} \PY{l+m+mi}{2019}\PY{n}{\PYZus{}2\PYZus{}mcpp\PYZus{}taller\PYZus{}3\PYZus{}listas\PYZus{}ejemplos}\PY{o}{.}\PY{n}{py}
\end{Verbatim}


    Este archivo contiene tres listas (l0, l1 y l2) que usará para las
tareas de esta sección. Puede ver los valores de las listas simplemente
escribiendo sus nombres y ejecutándolos en el Notebook. Inténtelo para
verificar que 2019\_2\_mcpp\_taller\_3\_listas\_ejemplos.py quedó bien
cargado. Debería ver:

    In {[}1{]}: l0 Out{[}1{]}: {[}{]}

In {[}2{]}: l1 Out{[}2{]}: {[}1, 'abc', 5.7, {[}1, 3, 5{]}{]}

In {[}3{]}: l2 Out{[}3{]}: {[}10, 11, 12, 13, 14, 15, 16{]}

    \begin{Verbatim}[commandchars=\\\{\}]
{\color{incolor}In [{\color{incolor}4}]:} \PY{n+nb}{print}\PY{p}{(}\PY{n}{l0}\PY{p}{)}
        \PY{n+nb}{print}\PY{p}{(}\PY{n}{l1}\PY{p}{)}
        \PY{n+nb}{print}\PY{p}{(}\PY{n}{l2}\PY{p}{)}
\end{Verbatim}


    \begin{Verbatim}[commandchars=\\\{\}]
[]
[1, 'abc', 5.7, [1, 3, 5]]
[10, 11, 12, 13, 14, 15, 16]

    \end{Verbatim}

    \subsection{1. {[}1{]}}\label{section}

Cree una lista que contenga los elementos 7, "xyz" y 2.7.

    \begin{Verbatim}[commandchars=\\\{\}]
{\color{incolor}In [{\color{incolor}5}]:} \PY{n}{primera\PYZus{}lista}\PY{o}{=}\PY{p}{[}\PY{l+m+mi}{7}\PY{p}{,}\PY{l+s+s2}{\PYZdq{}}\PY{l+s+s2}{xyz}\PY{l+s+s2}{\PYZdq{}}\PY{p}{,}\PY{l+m+mf}{2.7}\PY{p}{]}
\end{Verbatim}


    \subsection{2. {[}1{]}}\label{section}

Halle la longitud de la lista l1.

    \begin{Verbatim}[commandchars=\\\{\}]
{\color{incolor}In [{\color{incolor}8}]:} \PY{c+c1}{\PYZsh{}\PYZsh{} Longitud de L1}
        \PY{n}{longitud}\PY{o}{=}\PY{n+nb}{len}\PY{p}{(}\PY{n}{l1}\PY{p}{)}
        \PY{n+nb}{print}\PY{p}{(}\PY{l+s+s2}{\PYZdq{}}\PY{l+s+s2}{la longitud de l1 es:}\PY{l+s+s2}{\PYZdq{}}\PY{p}{,}\PY{n}{longitud}\PY{p}{)}
\end{Verbatim}


    \begin{Verbatim}[commandchars=\\\{\}]
la longitud de l1 es: 4

    \end{Verbatim}

    \subsection{3. {[}1{]}}\label{section}

Escriba expresiones para obtener el valor 5.7 de la lista l1 y para
obtener el valor 5 a partir del cuarto elemento de l1.

    \begin{Verbatim}[commandchars=\\\{\}]
{\color{incolor}In [{\color{incolor}17}]:} \PY{c+c1}{\PYZsh{}\PYZsh{}\PYZsh{} Obtener el valor 5.7}
         \PY{n}{posicion\PYZus{}1}\PY{o}{=}\PY{l+m+mi}{0}
         \PY{k}{for} \PY{n}{i} \PY{o+ow}{in} \PY{n+nb}{range}\PY{p}{(}\PY{l+m+mi}{0}\PY{p}{,}\PY{n+nb}{len}\PY{p}{(}\PY{n}{l1}\PY{p}{)}\PY{p}{)}\PY{p}{:}
             \PY{k}{if} \PY{n}{l1}\PY{p}{[}\PY{n}{i}\PY{p}{]}\PY{o}{==}\PY{l+m+mf}{5.7}\PY{p}{:}
                 \PY{n+nb}{print} \PY{p}{(}\PY{l+s+s2}{\PYZdq{}}\PY{l+s+s2}{El valor 5.7 esta en la posicion:}\PY{l+s+s2}{\PYZdq{}}\PY{p}{,} \PY{n}{i}\PY{p}{)}
                 \PY{n}{posicion\PYZus{}1}\PY{o}{=}\PY{n}{i}
                 
         \PY{c+c1}{\PYZsh{}\PYZsh{}\PYZsh{} obtener el valor de 5}
         \PY{n}{posicion\PYZus{}2}\PY{o}{=}\PY{l+m+mi}{0}
         \PY{k}{for} \PY{n}{j} \PY{o+ow}{in} \PY{n+nb}{range}\PY{p}{(}\PY{l+m+mi}{0}\PY{p}{,}\PY{n+nb}{len}\PY{p}{(}\PY{n}{l1}\PY{p}{)}\PY{p}{)}\PY{p}{:}
             \PY{k}{if} \PY{n}{j}\PY{o}{==}\PY{p}{(}\PY{n+nb}{len}\PY{p}{(}\PY{n}{l1}\PY{p}{)}\PY{o}{\PYZhy{}}\PY{l+m+mi}{1}\PY{p}{)}\PY{p}{:}
                 \PY{k}{for} \PY{n}{h} \PY{o+ow}{in} \PY{n+nb}{range}\PY{p}{(}\PY{l+m+mi}{0}\PY{p}{,}\PY{n+nb}{len}\PY{p}{(}\PY{n}{l1}\PY{p}{[}\PY{n}{j}\PY{p}{]}\PY{p}{)}\PY{p}{)}\PY{p}{:}
                     \PY{k}{if} \PY{n}{l1}\PY{p}{[}\PY{n}{j}\PY{p}{]}\PY{p}{[}\PY{n}{h}\PY{p}{]}\PY{o}{==}\PY{l+m+mi}{5}\PY{p}{:}
                          \PY{n+nb}{print} \PY{p}{(}\PY{l+s+s2}{\PYZdq{}}\PY{l+s+s2}{El valor 5 esta en la posicion:}\PY{l+s+s2}{\PYZdq{}}\PY{p}{,} \PY{n}{h}\PY{p}{,} \PY{l+s+s2}{\PYZdq{}}\PY{l+s+s2}{, de la lista de l1 que esta en la posicion:}\PY{l+s+s2}{\PYZdq{}}\PY{p}{,}\PY{n}{j}\PY{p}{)}
                          \PY{n}{posicion\PYZus{}2}\PY{o}{=}\PY{n}{h}
                         
\end{Verbatim}


    \begin{Verbatim}[commandchars=\\\{\}]
El valor 5.7 esta en la posicion: 2
El valor 5 esta en la posicion: 2 , de la lista de l1 que esta en la posicion: 3

    \end{Verbatim}

    \subsection{4. {[}1{]}}\label{section}

Prediga qué ocurrirá si se evalúa la expresión l1{[}4{]} y luego
pruébelo.

    \subsection{Rta:}\label{rta}

Yo predigo que el programa mostrará error dado que las posiciones de las
listas en python empiezan en cero, por ende el último elemento es
len(l1)-1 en este caso 3. Con 4 salimos del rango de posiciones de la
lista l1.

    \begin{Verbatim}[commandchars=\\\{\}]
{\color{incolor}In [{\color{incolor}18}]:} \PY{n}{l1}\PY{p}{[}\PY{l+m+mi}{4}\PY{p}{]}
\end{Verbatim}


    \begin{Verbatim}[commandchars=\\\{\}]

        ---------------------------------------------------------------------------

        IndexError                                Traceback (most recent call last)

        <ipython-input-18-7ffdcb2c9f2e> in <module>()
    ----> 1 l1[4]
    

        IndexError: list index out of range

    \end{Verbatim}

    \subsection{5. {[}1{]}}\label{section}

Prediga qué ocurrirá si se evalúa la expresión l2{[}-1{]} y luego
pruébelo.

    \subsection{Rta:}\label{rta}

Yo predigo que el programa mostrará el último elemento de la lista, en
este caso 16, dado que el -1 indica eso.

    \begin{Verbatim}[commandchars=\\\{\}]
{\color{incolor}In [{\color{incolor}21}]:} \PY{n}{l2}\PY{p}{[}\PY{o}{\PYZhy{}}\PY{l+m+mi}{1}\PY{p}{]}
\end{Verbatim}


\begin{Verbatim}[commandchars=\\\{\}]
{\color{outcolor}Out[{\color{outcolor}21}]:} 16
\end{Verbatim}
            
    \subsection{6. {[}1{]}}\label{section}

Escriba una expresión para cambiar el valor 3 en el cuarto elemento de
l1 a 15.0.

    \begin{Verbatim}[commandchars=\\\{\}]
{\color{incolor}In [{\color{incolor}39}]:} \PY{n+nb}{print}\PY{p}{(}\PY{l+s+s2}{\PYZdq{}}\PY{l+s+s2}{lista l1:}\PY{l+s+s2}{\PYZdq{}}\PY{p}{,} \PY{n}{l1}\PY{p}{)}
         \PY{k}{for} \PY{n}{j} \PY{o+ow}{in} \PY{n+nb}{range}\PY{p}{(}\PY{l+m+mi}{0}\PY{p}{,}\PY{n+nb}{len}\PY{p}{(}\PY{n}{l1}\PY{p}{)}\PY{p}{)}\PY{p}{:}
             \PY{k}{if} \PY{n}{j}\PY{o}{==}\PY{p}{(}\PY{n+nb}{len}\PY{p}{(}\PY{n}{l1}\PY{p}{)}\PY{o}{\PYZhy{}}\PY{l+m+mi}{1}\PY{p}{)}\PY{p}{:}
                 \PY{k}{for} \PY{n}{h} \PY{o+ow}{in} \PY{n+nb}{range}\PY{p}{(}\PY{l+m+mi}{0}\PY{p}{,}\PY{n+nb}{len}\PY{p}{(}\PY{n}{l1}\PY{p}{[}\PY{n}{j}\PY{p}{]}\PY{p}{)}\PY{p}{)}\PY{p}{:}
                     \PY{k}{if} \PY{n}{l1}\PY{p}{[}\PY{n}{j}\PY{p}{]}\PY{p}{[}\PY{n}{h}\PY{p}{]}\PY{o}{==}\PY{l+m+mi}{3}\PY{p}{:}
                         \PY{c+c1}{\PYZsh{}cambiamos}
                         \PY{n}{l1}\PY{p}{[}\PY{n}{j}\PY{p}{]}\PY{p}{[}\PY{n}{h}\PY{p}{]}\PY{o}{=}\PY{l+m+mf}{15.0}
         
         \PY{n+nb}{print}\PY{p}{(}\PY{l+s+s2}{\PYZdq{}}\PY{l+s+s2}{lista l1, con cambio en el ultimo elemento:}\PY{l+s+s2}{\PYZdq{}}\PY{p}{,} \PY{n}{l1}\PY{p}{)}
\end{Verbatim}


    \begin{Verbatim}[commandchars=\\\{\}]
lista l1: [1, 'abc', 5.7, [1, 3, 5]]
lista l1, con cambio en el ultimo elemento: [1, 'abc', 5.7, [1, 15.0, 5]]

    \end{Verbatim}

    \subsection{7. {[}1{]}}\label{section}

Escriba una expresión para crear un "slice" que contenga del segundo al
quinto elemento (inclusive) de la lista l2.

    \begin{Verbatim}[commandchars=\\\{\}]
{\color{incolor}In [{\color{incolor}45}]:} \PY{c+c1}{\PYZsh{} Segundo elemento seria 11 y el quinto seria 14}
         \PY{n}{slice\PYZus{}1}\PY{o}{=}\PY{n}{l2}\PY{p}{[}\PY{l+m+mi}{1}\PY{p}{:}\PY{l+m+mi}{5}\PY{p}{]}
         \PY{n+nb}{print}\PY{p}{(}\PY{n}{slice\PYZus{}1}\PY{p}{)}
\end{Verbatim}


    \begin{Verbatim}[commandchars=\\\{\}]
[11, 12, 13, 14]

    \end{Verbatim}

    \begin{Verbatim}[commandchars=\\\{\}]
{\color{incolor}In [{\color{incolor}51}]:} \PY{c+c1}{\PYZsh{}\PYZsh{}\PYZsh{} Ahora si es por posiciones (en caso tal de que no haya entendido bien la pregunta)}
         \PY{n}{slice\PYZus{}2}\PY{o}{=}\PY{n}{l2}\PY{p}{[}\PY{l+m+mi}{2}\PY{p}{:}\PY{l+m+mi}{5}\PY{p}{]}
         \PY{n+nb}{print}\PY{p}{(}\PY{n}{slice\PYZus{}2}\PY{p}{)}
\end{Verbatim}


    \begin{Verbatim}[commandchars=\\\{\}]
[12, 13, 14]

    \end{Verbatim}

    \subsection{8. {[}1{]}}\label{section}

Escriba una expresión para crear un "slice" que contenga los primeros
tres elementos de la lista l2.

    \begin{Verbatim}[commandchars=\\\{\}]
{\color{incolor}In [{\color{incolor}54}]:} \PY{n}{slice\PYZus{}3}\PY{o}{=}\PY{n}{l2}\PY{p}{[}\PY{p}{:}\PY{l+m+mi}{3}\PY{p}{]}
         \PY{n+nb}{print}\PY{p}{(}\PY{l+s+s2}{\PYZdq{}}\PY{l+s+s2}{lista l2:}\PY{l+s+s2}{\PYZdq{}}\PY{p}{,}\PY{n}{l2}\PY{p}{)}
         \PY{n+nb}{print}\PY{p}{(}\PY{l+s+s2}{\PYZdq{}}\PY{l+s+s2}{lista con slice:}\PY{l+s+s2}{\PYZdq{}}\PY{p}{,} \PY{n}{slice\PYZus{}3}\PY{p}{)}
\end{Verbatim}


    \begin{Verbatim}[commandchars=\\\{\}]
lista l2: [10, 11, 12, 13, 14, 15, 16]
lista con slice: [10, 11, 12]

    \end{Verbatim}

    \subsection{9. {[}1{]}}\label{section}

Escriba una expresión para crear un "slice" que contenga del segundo al
último elemento de la lista l2.

    \begin{Verbatim}[commandchars=\\\{\}]
{\color{incolor}In [{\color{incolor}56}]:} \PY{n}{slice\PYZus{}4}\PY{o}{=}\PY{n}{l2}\PY{p}{[}\PY{l+m+mi}{1}\PY{p}{:}\PY{p}{]}
         \PY{n+nb}{print}\PY{p}{(}\PY{l+s+s2}{\PYZdq{}}\PY{l+s+s2}{lista l2:}\PY{l+s+s2}{\PYZdq{}}\PY{p}{,}\PY{n}{l2}\PY{p}{)}
         \PY{n+nb}{print}\PY{p}{(}\PY{l+s+s2}{\PYZdq{}}\PY{l+s+s2}{lista con slice:}\PY{l+s+s2}{\PYZdq{}}\PY{p}{,} \PY{n}{slice\PYZus{}4}\PY{p}{)}
\end{Verbatim}


    \begin{Verbatim}[commandchars=\\\{\}]
lista l2: [10, 11, 12, 13, 14, 15, 16]
lista con slice: [11, 12, 13, 14, 15, 16]

    \end{Verbatim}

    \subsection{10. {[}1{]}}\label{section}

Escriba un código para añadir cuatro elementos a la lista l0 usando la
operación append y luego extraiga el tercer elemento (quítelo de la
lista). ¿Cuántos "appends" debe hacer?

    \begin{Verbatim}[commandchars=\\\{\}]
{\color{incolor}In [{\color{incolor}5}]:} \PY{k+kn}{import} \PY{n+nn}{random} \PY{k}{as} \PY{n+nn}{r}
        \PY{c+c1}{\PYZsh{}\PYZsh{}\PYZsh{} Agregare numeros aleatorios}
        
        \PY{n}{l0}\PY{o}{=}\PY{p}{[}\PY{p}{]}
        
        \PY{k}{for} \PY{n}{i} \PY{o+ow}{in} \PY{n+nb}{range}\PY{p}{(}\PY{l+m+mi}{0}\PY{p}{,}\PY{l+m+mi}{4}\PY{p}{)}\PY{p}{:}
            \PY{n}{b}\PY{o}{=}\PY{n}{r}\PY{o}{.}\PY{n}{randint}\PY{p}{(}\PY{l+m+mi}{0}\PY{p}{,}\PY{l+m+mi}{100}\PY{p}{)}
            \PY{n}{l0}\PY{o}{.}\PY{n}{append}\PY{p}{(}\PY{n}{b}\PY{p}{)}
            \PY{n+nb}{print}\PY{p}{(}\PY{n}{i}\PY{p}{)}
        \PY{n+nb}{print}\PY{p}{(}\PY{l+s+s2}{\PYZdq{}}\PY{l+s+s2}{lista l0 agregando cuatro elementos: }\PY{l+s+s2}{\PYZdq{}}\PY{p}{,} \PY{n}{l0}\PY{p}{)}
        
        \PY{c+c1}{\PYZsh{}\PYZsh{} Eliminar elemento}
        \PY{n}{l0}\PY{o}{.}\PY{n}{pop}\PY{p}{(}\PY{l+m+mi}{2}\PY{p}{)}
        \PY{n+nb}{print}\PY{p}{(}\PY{l+s+s2}{\PYZdq{}}\PY{l+s+s2}{lista l0 eliminando el tercer elemento: }\PY{l+s+s2}{\PYZdq{}}\PY{p}{,} \PY{n}{l0}\PY{p}{)}
        
        \PY{n+nb}{print}\PY{p}{(} \PY{l+s+s2}{\PYZdq{}}\PY{l+s+s2}{ cantidad de appends fueron: 4}\PY{l+s+s2}{\PYZdq{}}\PY{p}{)}
\end{Verbatim}


    \begin{Verbatim}[commandchars=\\\{\}]
0
1
2
3
lista l0 agregando cuatro elementos:  [60, 6, 1, 42]
lista l0 eliminando el tercer elemento:  [60, 6, 42]
 cantidad de appends fueron: 4

    \end{Verbatim}

    \subsection{11. {[}1{]}}\label{section}

Cree una nueva lista nl concatenando la nueva versión de l0 con l1, y
luego actualice un elemento cualquiera de nl. ¿Cambia alguna de las
listas l0 o l1 al ejecutar los anteriores comandos?

    \begin{Verbatim}[commandchars=\\\{\}]
{\color{incolor}In [{\color{incolor}7}]:} \PY{n}{n1}\PY{o}{=}\PY{n}{l0}\PY{o}{+}\PY{n}{l1}
        \PY{n+nb}{print}\PY{p}{(}\PY{l+s+s2}{\PYZdq{}}\PY{l+s+s2}{n1 concatenando l1 y l0:}\PY{l+s+s2}{\PYZdq{}}\PY{p}{,} \PY{n}{n1}\PY{p}{)}
        
        \PY{c+c1}{\PYZsh{} cambio el primero elemento de n1}
        \PY{n}{n1}\PY{p}{[}\PY{l+m+mi}{0}\PY{p}{]}\PY{o}{=}\PY{l+s+s2}{\PYZdq{}}\PY{l+s+s2}{hola}\PY{l+s+s2}{\PYZdq{}}
        \PY{n+nb}{print}\PY{p}{(}\PY{l+s+s2}{\PYZdq{}}\PY{l+s+s2}{n1 cambiando el primer elemento de }\PY{l+s+s2}{\PYZsq{}}\PY{l+s+s2}{1}\PY{l+s+s2}{\PYZsq{}}\PY{l+s+s2}{ a }\PY{l+s+s2}{\PYZsq{}}\PY{l+s+s2}{hola}\PY{l+s+s2}{\PYZsq{}}\PY{l+s+s2}{: }\PY{l+s+s2}{\PYZdq{}}\PY{p}{,} \PY{n}{n1}\PY{p}{)}
        
        \PY{n+nb}{print}\PY{p}{(}\PY{l+s+s2}{\PYZdq{}}\PY{l+s+s2}{l0:}\PY{l+s+s2}{\PYZdq{}}\PY{p}{,}\PY{n}{l0}\PY{p}{)}
        
        \PY{n+nb}{print}\PY{p}{(}\PY{l+s+s2}{\PYZdq{}}\PY{l+s+s2}{l1:}\PY{l+s+s2}{\PYZdq{}}\PY{p}{,}\PY{n}{l1}\PY{p}{)}
\end{Verbatim}


    \begin{Verbatim}[commandchars=\\\{\}]
n1 concatenando l1 y l0: [60, 6, 42, 1, 'abc', 5.7, [1, 3, 5]]
n1 cambiando el primer elemento de '1' a 'hola':  ['hola', 6, 42, 1, 'abc', 5.7, [1, 3, 5]]
l0: [60, 6, 42]
l1: [1, 'abc', 5.7, [1, 3, 5]]

    \end{Verbatim}

    \subsection{Rta:}\label{rta}

Ni l0, ni l1 cambian dado n1 es una especie de copia de l1+l0, esta
"copia" no cambia los valores originales de las listas.

    \subsection{12. {[}2{]}}\label{section}

Escriba un loop que compute una variable all\_pos cuyo valor sea True si
todos los elementos de la lista l3 son positivos y False en otro caso.

    \begin{Verbatim}[commandchars=\\\{\}]
{\color{incolor}In [{\color{incolor}1}]:} \PY{c+c1}{\PYZsh{}\PYZsh{}\PYZsh{} Construyo la lista l3 \PYZsh{}\PYZsh{}}
        
        \PY{n}{l3}\PY{o}{=}\PY{p}{[}\PY{l+m+mi}{1}\PY{p}{,}\PY{l+m+mi}{2}\PY{p}{,}\PY{l+m+mi}{3}\PY{p}{,}\PY{l+m+mi}{4}\PY{p}{,}\PY{l+m+mi}{5}\PY{p}{,}\PY{l+m+mi}{6}\PY{p}{,}\PY{l+m+mi}{8}\PY{p}{,}\PY{l+m+mi}{7}\PY{p}{,}\PY{l+m+mi}{9}\PY{p}{,}\PY{l+m+mi}{10}\PY{p}{,}\PY{l+m+mi}{0}\PY{p}{,}\PY{o}{\PYZhy{}}\PY{l+m+mi}{1}\PY{p}{,}\PY{o}{\PYZhy{}}\PY{l+m+mi}{2}\PY{p}{,}\PY{o}{\PYZhy{}}\PY{l+m+mi}{3}\PY{p}{,}\PY{o}{\PYZhy{}}\PY{l+m+mi}{4}\PY{p}{,}\PY{o}{\PYZhy{}}\PY{l+m+mi}{5}\PY{p}{,}\PY{o}{\PYZhy{}}\PY{l+m+mi}{7}\PY{p}{]}
        
        \PY{n}{all\PYZus{}pos}\PY{o}{=}\PY{k+kc}{True}
        \PY{k}{for} \PY{n}{i} \PY{o+ow}{in} \PY{n}{l3}\PY{p}{:}
            \PY{k}{if} \PY{n}{i}\PY{o}{\PYZlt{}}\PY{o}{=}\PY{l+m+mi}{0}\PY{p}{:}
                \PY{n}{all\PYZus{}pos}\PY{o}{=}\PY{k+kc}{False}
                \PY{k}{break}
        \PY{k}{if} \PY{n}{all\PYZus{}pos} \PY{o}{==}\PY{k+kc}{True}\PY{p}{:}
            \PY{n+nb}{print}\PY{p}{(}\PY{l+s+s2}{\PYZdq{}}\PY{l+s+s2}{La lista l3 solo tiene valores positivos}\PY{l+s+s2}{\PYZdq{}}\PY{p}{)}
        \PY{k}{else}\PY{p}{:}
            \PY{n+nb}{print}\PY{p}{(}\PY{l+s+s2}{\PYZdq{}}\PY{l+s+s2}{la lista l3 puede tener valores negativos o igual a cero}\PY{l+s+s2}{\PYZdq{}}\PY{p}{)}
                
\end{Verbatim}


    \begin{Verbatim}[commandchars=\\\{\}]
la lista l3 puede tener valores negativos o igual a cero

    \end{Verbatim}

    \subsection{13. {[}2{]}}\label{section}

Escriba un código para crear una nueva lista que contenga solo los
valores positivos de la lista l3.

    \begin{Verbatim}[commandchars=\\\{\}]
{\color{incolor}In [{\color{incolor}3}]:} \PY{c+c1}{\PYZsh{}\PYZsh{}\PYZsh{}\PYZsh{} lista solo valores positivos}
        
        \PY{n}{lista\PYZus{}positivos}\PY{o}{=}\PY{p}{[}\PY{p}{]}
        \PY{k}{for} \PY{n}{j} \PY{o+ow}{in} \PY{n}{l3}\PY{p}{:}
            \PY{k}{if} \PY{n}{j}\PY{o}{\PYZgt{}}\PY{l+m+mi}{0}\PY{p}{:}
                \PY{n}{lista\PYZus{}positivos}\PY{o}{.}\PY{n}{append}\PY{p}{(}\PY{n}{j}\PY{p}{)}
                
        \PY{n+nb}{print}\PY{p}{(}\PY{l+s+s2}{\PYZdq{}}\PY{l+s+s2}{la lista de número positvos de l3 es: }\PY{l+s+s2}{\PYZdq{}}\PY{p}{,} \PY{n}{lista\PYZus{}positivos}\PY{p}{)}
\end{Verbatim}


    \begin{Verbatim}[commandchars=\\\{\}]
la lista de número positvos de l3 es:  [1, 2, 3, 4, 5, 6, 8, 7, 9, 10]

    \end{Verbatim}

    \subsection{14. {[}2{]}}\label{section}

Escriba un código que use append para crear una nueva lista nl en la que
el i-ésimo elemento de nl tiene el valor True si el i-ésimo elemento de
l3 tiene un valor positivo y Falso en otro caso.

    \begin{Verbatim}[commandchars=\\\{\}]
{\color{incolor}In [{\color{incolor}10}]:} \PY{n}{n1}\PY{o}{=}\PY{p}{[}\PY{p}{]}
         \PY{k}{for} \PY{n}{j} \PY{o+ow}{in} \PY{n}{l3}\PY{p}{:}
             \PY{k}{if} \PY{n}{j} \PY{o}{\PYZgt{}}\PY{l+m+mi}{0}\PY{p}{:}
                 \PY{n}{n1}\PY{o}{.}\PY{n}{append}\PY{p}{(}\PY{k+kc}{True}\PY{p}{)}
             \PY{k}{else}\PY{p}{:}
                 \PY{n}{n1}\PY{o}{.}\PY{n}{append}\PY{p}{(}\PY{k+kc}{False}\PY{p}{)}
         
         \PY{n+nb}{print}\PY{p}{(}\PY{l+s+s2}{\PYZdq{}}\PY{l+s+s2}{l3:}\PY{l+s+s2}{\PYZdq{}}\PY{p}{,} \PY{n}{l3}\PY{p}{)}
         \PY{n+nb}{print}\PY{p}{(}\PY{l+s+s2}{\PYZdq{}}\PY{l+s+s2}{n1:}\PY{l+s+s2}{\PYZdq{}}\PY{p}{,} \PY{n}{n1}\PY{p}{)}
\end{Verbatim}


    \begin{Verbatim}[commandchars=\\\{\}]
l3: [1, 2, 3, 4, 5, 6, 8, 7, 9, 10, 0, -1, -2, -3, -4, -5, -7]
n1: [True, True, True, True, True, True, True, True, True, True, False, False, False, False, False, False, False]

    \end{Verbatim}

    \subsection{15. {[}3{]}}\label{section}

Escriba un código que use range, para crear una nueva lista nl en la que
el i-ésimo elemento de nl es True si el i-ésimo elemento de l3 es
positivo y False en otro caso.

\textbf{Pista:} Comience por crear una lista de longitud adecuada, con
False en cada elemento.

    \begin{Verbatim}[commandchars=\\\{\}]
{\color{incolor}In [{\color{incolor}12}]:} \PY{n}{n1}\PY{o}{=}\PY{p}{[}\PY{k+kc}{False}\PY{p}{]}\PY{o}{*}\PY{n+nb}{len}\PY{p}{(}\PY{n}{l3}\PY{p}{)}
         
         \PY{k}{for} \PY{n}{i} \PY{o+ow}{in} \PY{n+nb}{range}\PY{p}{(}\PY{l+m+mi}{0}\PY{p}{,}\PY{n+nb}{len}\PY{p}{(}\PY{n}{l3}\PY{p}{)}\PY{p}{)}\PY{p}{:}
             \PY{k}{if} \PY{n}{l3}\PY{p}{[}\PY{n}{i}\PY{p}{]}\PY{o}{\PYZgt{}}\PY{l+m+mi}{0}\PY{p}{:}
                 \PY{n}{n1}\PY{p}{[}\PY{n}{i}\PY{p}{]}\PY{o}{=}\PY{k+kc}{True}
         
         \PY{n+nb}{print}\PY{p}{(}\PY{l+s+s2}{\PYZdq{}}\PY{l+s+s2}{l3: }\PY{l+s+s2}{\PYZdq{}}\PY{p}{,} \PY{n}{l3}\PY{p}{)}
         \PY{n+nb}{print}\PY{p}{(}\PY{l+s+s2}{\PYZdq{}}\PY{l+s+s2}{n1: }\PY{l+s+s2}{\PYZdq{}}\PY{p}{,} \PY{n}{n1}\PY{p}{)}
\end{Verbatim}


    \begin{Verbatim}[commandchars=\\\{\}]
l3:  [1, 2, 3, 4, 5, 6, 8, 7, 9, 10, 0, -1, -2, -3, -4, -5, -7]
n1:  [True, True, True, True, True, True, True, True, True, True, False, False, False, False, False, False, False]

    \end{Verbatim}

    \subsection{16. {[}4{]}}\label{section}

En clase construimos una lista con 10000 números aleatorios entre 0 y 9,
a partir del siguiente código:

    \begin{Verbatim}[commandchars=\\\{\}]
{\color{incolor}In [{\color{incolor} }]:} \PY{c+c1}{\PYZsh{}import random}
        
        \PY{c+c1}{\PYZsh{}N = 10000  }
        \PY{c+c1}{\PYZsh{}random\PYZus{}numbers = []}
        \PY{c+c1}{\PYZsh{}for i in range(N):}
        \PY{c+c1}{\PYZsh{}    random\PYZus{}numbers.append(random.randint(0,9))}
\end{Verbatim}


    Y creamos un "contador" que calcula la frecuencia de ocurrencia de cada
número del 0 al 9, así:

    \begin{Verbatim}[commandchars=\\\{\}]
{\color{incolor}In [{\color{incolor} }]:} \PY{c+c1}{\PYZsh{} count = []}
        \PY{c+c1}{\PYZsh{} for x in range(0,10):}
        \PY{c+c1}{\PYZsh{}    count.append(random\PYZus{}numbers.count(x))}
\end{Verbatim}


    \begin{Verbatim}[commandchars=\\\{\}]
{\color{incolor}In [{\color{incolor}91}]:} \PY{k+kn}{import} \PY{n+nn}{random}
         
         \PY{n}{N} \PY{o}{=} \PY{l+m+mi}{10000}  
         \PY{n}{random\PYZus{}numbers} \PY{o}{=} \PY{p}{[}\PY{p}{]}
         \PY{k}{for} \PY{n}{i} \PY{o+ow}{in} \PY{n+nb}{range}\PY{p}{(}\PY{n}{N}\PY{p}{)}\PY{p}{:}
             \PY{n}{random\PYZus{}numbers}\PY{o}{.}\PY{n}{append}\PY{p}{(}\PY{n}{random}\PY{o}{.}\PY{n}{randint}\PY{p}{(}\PY{l+m+mi}{0}\PY{p}{,}\PY{l+m+mi}{9}\PY{p}{)}\PY{p}{)}
             
         \PY{n}{lista\PYZus{}contador}\PY{o}{=}\PY{p}{[}\PY{l+m+mi}{0}\PY{p}{]}\PY{o}{*}\PY{l+m+mi}{10}
         \PY{k}{for} \PY{n}{i} \PY{o+ow}{in} \PY{n}{random\PYZus{}numbers}\PY{p}{:}
             \PY{k}{if} \PY{n}{i}\PY{o}{==}\PY{l+m+mi}{0}\PY{p}{:}
                 \PY{n}{lista\PYZus{}contador}\PY{p}{[}\PY{n}{i}\PY{p}{]}\PY{o}{=}\PY{n}{lista\PYZus{}contador}\PY{p}{[}\PY{n}{i}\PY{p}{]}\PY{o}{+}\PY{l+m+mi}{1}
             \PY{k}{elif} \PY{n}{i}\PY{o}{==}\PY{l+m+mi}{1}\PY{p}{:}
                 \PY{n}{lista\PYZus{}contador}\PY{p}{[}\PY{n}{i}\PY{p}{]}\PY{o}{=}\PY{n}{lista\PYZus{}contador}\PY{p}{[}\PY{n}{i}\PY{p}{]}\PY{o}{+}\PY{l+m+mi}{1}
             \PY{k}{elif} \PY{n}{i}\PY{o}{==}\PY{l+m+mi}{2}\PY{p}{:}
                 \PY{n}{lista\PYZus{}contador}\PY{p}{[}\PY{n}{i}\PY{p}{]}\PY{o}{=}\PY{n}{lista\PYZus{}contador}\PY{p}{[}\PY{n}{i}\PY{p}{]}\PY{o}{+}\PY{l+m+mi}{1}
             \PY{k}{elif} \PY{n}{i}\PY{o}{==}\PY{l+m+mi}{3}\PY{p}{:}
                 \PY{n}{lista\PYZus{}contador}\PY{p}{[}\PY{n}{i}\PY{p}{]}\PY{o}{=}\PY{n}{lista\PYZus{}contador}\PY{p}{[}\PY{n}{i}\PY{p}{]}\PY{o}{+}\PY{l+m+mi}{1}
             \PY{k}{elif} \PY{n}{i}\PY{o}{==}\PY{l+m+mi}{4}\PY{p}{:}
                 \PY{n}{lista\PYZus{}contador}\PY{p}{[}\PY{n}{i}\PY{p}{]}\PY{o}{=}\PY{n}{lista\PYZus{}contador}\PY{p}{[}\PY{n}{i}\PY{p}{]}\PY{o}{+}\PY{l+m+mi}{1}
             \PY{k}{elif} \PY{n}{i}\PY{o}{==}\PY{l+m+mi}{5}\PY{p}{:}
                 \PY{n}{lista\PYZus{}contador}\PY{p}{[}\PY{n}{i}\PY{p}{]}\PY{o}{=}\PY{n}{lista\PYZus{}contador}\PY{p}{[}\PY{n}{i}\PY{p}{]}\PY{o}{+}\PY{l+m+mi}{1}
             \PY{k}{elif} \PY{n}{i}\PY{o}{==}\PY{l+m+mi}{6}\PY{p}{:}
                 \PY{n}{lista\PYZus{}contador}\PY{p}{[}\PY{n}{i}\PY{p}{]}\PY{o}{=}\PY{n}{lista\PYZus{}contador}\PY{p}{[}\PY{n}{i}\PY{p}{]}\PY{o}{+}\PY{l+m+mi}{1}
             \PY{k}{elif} \PY{n}{i}\PY{o}{==}\PY{l+m+mi}{7}\PY{p}{:}
                 \PY{n}{lista\PYZus{}contador}\PY{p}{[}\PY{n}{i}\PY{p}{]}\PY{o}{=}\PY{n}{lista\PYZus{}contador}\PY{p}{[}\PY{n}{i}\PY{p}{]}\PY{o}{+}\PY{l+m+mi}{1}
             \PY{k}{elif} \PY{n}{i}\PY{o}{==}\PY{l+m+mi}{8}\PY{p}{:}
                 \PY{n}{lista\PYZus{}contador}\PY{p}{[}\PY{n}{i}\PY{p}{]}\PY{o}{=}\PY{n}{lista\PYZus{}contador}\PY{p}{[}\PY{n}{i}\PY{p}{]}\PY{o}{+}\PY{l+m+mi}{1}
             \PY{k}{else}\PY{p}{:}
                 \PY{n}{lista\PYZus{}contador}\PY{p}{[}\PY{n}{i}\PY{p}{]}\PY{o}{=}\PY{n}{lista\PYZus{}contador}\PY{p}{[}\PY{n}{i}\PY{p}{]}\PY{o}{+}\PY{l+m+mi}{1}
         
         \PY{n}{completa}\PY{o}{=}\PY{p}{[}\PY{p}{]}
         \PY{k}{for} \PY{n}{j} \PY{o+ow}{in} \PY{n+nb}{range}\PY{p}{(}\PY{l+m+mi}{0}\PY{p}{,}\PY{l+m+mi}{10}\PY{p}{)}\PY{p}{:}
             \PY{n}{pareja}\PY{o}{=}\PY{p}{[}\PY{n+nb}{str}\PY{p}{(}\PY{n}{j}\PY{p}{)}\PY{o}{+}\PY{l+s+s2}{\PYZdq{}}\PY{l+s+s2}{:}\PY{l+s+s2}{\PYZdq{}}\PY{p}{,}\PY{n}{lista\PYZus{}contador}\PY{p}{[}\PY{n}{j}\PY{p}{]}\PY{p}{]}
             \PY{n}{completa}\PY{o}{.}\PY{n}{append}\PY{p}{(}\PY{n}{pareja}\PY{p}{)}
         \PY{n+nb}{print}\PY{p}{(}\PY{l+s+s2}{\PYZdq{}}\PY{l+s+s2}{La cantidad de cada uno de los numeros de 0 a 9 en la lista random numbers es: }\PY{l+s+s2}{\PYZdq{}}\PY{p}{,}\PY{n}{completa}\PY{p}{)}
\end{Verbatim}


    \begin{Verbatim}[commandchars=\\\{\}]
La cantidad de cada uno de los numeros de 0 a 9 en la lista random numbers es:  [['0:', 959], ['1:', 988], ['2:', 978], ['3:', 969], ['4:', 1029], ['5:', 996], ['6:', 997], ['7:', 1049], ['8:', 1004], ['9:', 1031]]

    \end{Verbatim}

    Cree un "contador" que haga lo mismo, pero sin hacer uso del método
"count". (De hecho, sin usar método alguno.)

    \textbf{Pistas:}

\begin{itemize}
\tightlist
\item
  Esto puede lograrse con un loop muy sencillo. Si su código es
  complejo, piense el problema de nuevo.
\item
  Es muy útil iniciar con una lista "vacía" de 10 elementos. Es decir,
  una lista con 10 ceros.
\end{itemize}

    \begin{center}\rule{0.5\linewidth}{\linethickness}\end{center}


    % Add a bibliography block to the postdoc
    
    
    
    \end{document}
