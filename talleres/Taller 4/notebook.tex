
% Default to the notebook output style

    


% Inherit from the specified cell style.




    
\documentclass[11pt]{article}

    
    
    \usepackage[T1]{fontenc}
    % Nicer default font (+ math font) than Computer Modern for most use cases
    \usepackage{mathpazo}

    % Basic figure setup, for now with no caption control since it's done
    % automatically by Pandoc (which extracts ![](path) syntax from Markdown).
    \usepackage{graphicx}
    % We will generate all images so they have a width \maxwidth. This means
    % that they will get their normal width if they fit onto the page, but
    % are scaled down if they would overflow the margins.
    \makeatletter
    \def\maxwidth{\ifdim\Gin@nat@width>\linewidth\linewidth
    \else\Gin@nat@width\fi}
    \makeatother
    \let\Oldincludegraphics\includegraphics
    % Set max figure width to be 80% of text width, for now hardcoded.
    \renewcommand{\includegraphics}[1]{\Oldincludegraphics[width=.8\maxwidth]{#1}}
    % Ensure that by default, figures have no caption (until we provide a
    % proper Figure object with a Caption API and a way to capture that
    % in the conversion process - todo).
    \usepackage{caption}
    \DeclareCaptionLabelFormat{nolabel}{}
    \captionsetup{labelformat=nolabel}

    \usepackage{adjustbox} % Used to constrain images to a maximum size 
    \usepackage{xcolor} % Allow colors to be defined
    \usepackage{enumerate} % Needed for markdown enumerations to work
    \usepackage{geometry} % Used to adjust the document margins
    \usepackage{amsmath} % Equations
    \usepackage{amssymb} % Equations
    \usepackage{textcomp} % defines textquotesingle
    % Hack from http://tex.stackexchange.com/a/47451/13684:
    \AtBeginDocument{%
        \def\PYZsq{\textquotesingle}% Upright quotes in Pygmentized code
    }
    \usepackage{upquote} % Upright quotes for verbatim code
    \usepackage{eurosym} % defines \euro
    \usepackage[mathletters]{ucs} % Extended unicode (utf-8) support
    \usepackage[utf8x]{inputenc} % Allow utf-8 characters in the tex document
    \usepackage{fancyvrb} % verbatim replacement that allows latex
    \usepackage{grffile} % extends the file name processing of package graphics 
                         % to support a larger range 
    % The hyperref package gives us a pdf with properly built
    % internal navigation ('pdf bookmarks' for the table of contents,
    % internal cross-reference links, web links for URLs, etc.)
    \usepackage{hyperref}
    \usepackage{longtable} % longtable support required by pandoc >1.10
    \usepackage{booktabs}  % table support for pandoc > 1.12.2
    \usepackage[inline]{enumitem} % IRkernel/repr support (it uses the enumerate* environment)
    \usepackage[normalem]{ulem} % ulem is needed to support strikethroughs (\sout)
                                % normalem makes italics be italics, not underlines
    

    
    
    % Colors for the hyperref package
    \definecolor{urlcolor}{rgb}{0,.145,.698}
    \definecolor{linkcolor}{rgb}{.71,0.21,0.01}
    \definecolor{citecolor}{rgb}{.12,.54,.11}

    % ANSI colors
    \definecolor{ansi-black}{HTML}{3E424D}
    \definecolor{ansi-black-intense}{HTML}{282C36}
    \definecolor{ansi-red}{HTML}{E75C58}
    \definecolor{ansi-red-intense}{HTML}{B22B31}
    \definecolor{ansi-green}{HTML}{00A250}
    \definecolor{ansi-green-intense}{HTML}{007427}
    \definecolor{ansi-yellow}{HTML}{DDB62B}
    \definecolor{ansi-yellow-intense}{HTML}{B27D12}
    \definecolor{ansi-blue}{HTML}{208FFB}
    \definecolor{ansi-blue-intense}{HTML}{0065CA}
    \definecolor{ansi-magenta}{HTML}{D160C4}
    \definecolor{ansi-magenta-intense}{HTML}{A03196}
    \definecolor{ansi-cyan}{HTML}{60C6C8}
    \definecolor{ansi-cyan-intense}{HTML}{258F8F}
    \definecolor{ansi-white}{HTML}{C5C1B4}
    \definecolor{ansi-white-intense}{HTML}{A1A6B2}

    % commands and environments needed by pandoc snippets
    % extracted from the output of `pandoc -s`
    \providecommand{\tightlist}{%
      \setlength{\itemsep}{0pt}\setlength{\parskip}{0pt}}
    \DefineVerbatimEnvironment{Highlighting}{Verbatim}{commandchars=\\\{\}}
    % Add ',fontsize=\small' for more characters per line
    \newenvironment{Shaded}{}{}
    \newcommand{\KeywordTok}[1]{\textcolor[rgb]{0.00,0.44,0.13}{\textbf{{#1}}}}
    \newcommand{\DataTypeTok}[1]{\textcolor[rgb]{0.56,0.13,0.00}{{#1}}}
    \newcommand{\DecValTok}[1]{\textcolor[rgb]{0.25,0.63,0.44}{{#1}}}
    \newcommand{\BaseNTok}[1]{\textcolor[rgb]{0.25,0.63,0.44}{{#1}}}
    \newcommand{\FloatTok}[1]{\textcolor[rgb]{0.25,0.63,0.44}{{#1}}}
    \newcommand{\CharTok}[1]{\textcolor[rgb]{0.25,0.44,0.63}{{#1}}}
    \newcommand{\StringTok}[1]{\textcolor[rgb]{0.25,0.44,0.63}{{#1}}}
    \newcommand{\CommentTok}[1]{\textcolor[rgb]{0.38,0.63,0.69}{\textit{{#1}}}}
    \newcommand{\OtherTok}[1]{\textcolor[rgb]{0.00,0.44,0.13}{{#1}}}
    \newcommand{\AlertTok}[1]{\textcolor[rgb]{1.00,0.00,0.00}{\textbf{{#1}}}}
    \newcommand{\FunctionTok}[1]{\textcolor[rgb]{0.02,0.16,0.49}{{#1}}}
    \newcommand{\RegionMarkerTok}[1]{{#1}}
    \newcommand{\ErrorTok}[1]{\textcolor[rgb]{1.00,0.00,0.00}{\textbf{{#1}}}}
    \newcommand{\NormalTok}[1]{{#1}}
    
    % Additional commands for more recent versions of Pandoc
    \newcommand{\ConstantTok}[1]{\textcolor[rgb]{0.53,0.00,0.00}{{#1}}}
    \newcommand{\SpecialCharTok}[1]{\textcolor[rgb]{0.25,0.44,0.63}{{#1}}}
    \newcommand{\VerbatimStringTok}[1]{\textcolor[rgb]{0.25,0.44,0.63}{{#1}}}
    \newcommand{\SpecialStringTok}[1]{\textcolor[rgb]{0.73,0.40,0.53}{{#1}}}
    \newcommand{\ImportTok}[1]{{#1}}
    \newcommand{\DocumentationTok}[1]{\textcolor[rgb]{0.73,0.13,0.13}{\textit{{#1}}}}
    \newcommand{\AnnotationTok}[1]{\textcolor[rgb]{0.38,0.63,0.69}{\textbf{\textit{{#1}}}}}
    \newcommand{\CommentVarTok}[1]{\textcolor[rgb]{0.38,0.63,0.69}{\textbf{\textit{{#1}}}}}
    \newcommand{\VariableTok}[1]{\textcolor[rgb]{0.10,0.09,0.49}{{#1}}}
    \newcommand{\ControlFlowTok}[1]{\textcolor[rgb]{0.00,0.44,0.13}{\textbf{{#1}}}}
    \newcommand{\OperatorTok}[1]{\textcolor[rgb]{0.40,0.40,0.40}{{#1}}}
    \newcommand{\BuiltInTok}[1]{{#1}}
    \newcommand{\ExtensionTok}[1]{{#1}}
    \newcommand{\PreprocessorTok}[1]{\textcolor[rgb]{0.74,0.48,0.00}{{#1}}}
    \newcommand{\AttributeTok}[1]{\textcolor[rgb]{0.49,0.56,0.16}{{#1}}}
    \newcommand{\InformationTok}[1]{\textcolor[rgb]{0.38,0.63,0.69}{\textbf{\textit{{#1}}}}}
    \newcommand{\WarningTok}[1]{\textcolor[rgb]{0.38,0.63,0.69}{\textbf{\textit{{#1}}}}}
    
    
    % Define a nice break command that doesn't care if a line doesn't already
    % exist.
    \def\br{\hspace*{\fill} \\* }
    % Math Jax compatability definitions
    \def\gt{>}
    \def\lt{<}
    % Document parameters
    \title{2019\_2\_mcpp\_taller\_4}
    
    
    

    % Pygments definitions
    
\makeatletter
\def\PY@reset{\let\PY@it=\relax \let\PY@bf=\relax%
    \let\PY@ul=\relax \let\PY@tc=\relax%
    \let\PY@bc=\relax \let\PY@ff=\relax}
\def\PY@tok#1{\csname PY@tok@#1\endcsname}
\def\PY@toks#1+{\ifx\relax#1\empty\else%
    \PY@tok{#1}\expandafter\PY@toks\fi}
\def\PY@do#1{\PY@bc{\PY@tc{\PY@ul{%
    \PY@it{\PY@bf{\PY@ff{#1}}}}}}}
\def\PY#1#2{\PY@reset\PY@toks#1+\relax+\PY@do{#2}}

\expandafter\def\csname PY@tok@w\endcsname{\def\PY@tc##1{\textcolor[rgb]{0.73,0.73,0.73}{##1}}}
\expandafter\def\csname PY@tok@c\endcsname{\let\PY@it=\textit\def\PY@tc##1{\textcolor[rgb]{0.25,0.50,0.50}{##1}}}
\expandafter\def\csname PY@tok@cp\endcsname{\def\PY@tc##1{\textcolor[rgb]{0.74,0.48,0.00}{##1}}}
\expandafter\def\csname PY@tok@k\endcsname{\let\PY@bf=\textbf\def\PY@tc##1{\textcolor[rgb]{0.00,0.50,0.00}{##1}}}
\expandafter\def\csname PY@tok@kp\endcsname{\def\PY@tc##1{\textcolor[rgb]{0.00,0.50,0.00}{##1}}}
\expandafter\def\csname PY@tok@kt\endcsname{\def\PY@tc##1{\textcolor[rgb]{0.69,0.00,0.25}{##1}}}
\expandafter\def\csname PY@tok@o\endcsname{\def\PY@tc##1{\textcolor[rgb]{0.40,0.40,0.40}{##1}}}
\expandafter\def\csname PY@tok@ow\endcsname{\let\PY@bf=\textbf\def\PY@tc##1{\textcolor[rgb]{0.67,0.13,1.00}{##1}}}
\expandafter\def\csname PY@tok@nb\endcsname{\def\PY@tc##1{\textcolor[rgb]{0.00,0.50,0.00}{##1}}}
\expandafter\def\csname PY@tok@nf\endcsname{\def\PY@tc##1{\textcolor[rgb]{0.00,0.00,1.00}{##1}}}
\expandafter\def\csname PY@tok@nc\endcsname{\let\PY@bf=\textbf\def\PY@tc##1{\textcolor[rgb]{0.00,0.00,1.00}{##1}}}
\expandafter\def\csname PY@tok@nn\endcsname{\let\PY@bf=\textbf\def\PY@tc##1{\textcolor[rgb]{0.00,0.00,1.00}{##1}}}
\expandafter\def\csname PY@tok@ne\endcsname{\let\PY@bf=\textbf\def\PY@tc##1{\textcolor[rgb]{0.82,0.25,0.23}{##1}}}
\expandafter\def\csname PY@tok@nv\endcsname{\def\PY@tc##1{\textcolor[rgb]{0.10,0.09,0.49}{##1}}}
\expandafter\def\csname PY@tok@no\endcsname{\def\PY@tc##1{\textcolor[rgb]{0.53,0.00,0.00}{##1}}}
\expandafter\def\csname PY@tok@nl\endcsname{\def\PY@tc##1{\textcolor[rgb]{0.63,0.63,0.00}{##1}}}
\expandafter\def\csname PY@tok@ni\endcsname{\let\PY@bf=\textbf\def\PY@tc##1{\textcolor[rgb]{0.60,0.60,0.60}{##1}}}
\expandafter\def\csname PY@tok@na\endcsname{\def\PY@tc##1{\textcolor[rgb]{0.49,0.56,0.16}{##1}}}
\expandafter\def\csname PY@tok@nt\endcsname{\let\PY@bf=\textbf\def\PY@tc##1{\textcolor[rgb]{0.00,0.50,0.00}{##1}}}
\expandafter\def\csname PY@tok@nd\endcsname{\def\PY@tc##1{\textcolor[rgb]{0.67,0.13,1.00}{##1}}}
\expandafter\def\csname PY@tok@s\endcsname{\def\PY@tc##1{\textcolor[rgb]{0.73,0.13,0.13}{##1}}}
\expandafter\def\csname PY@tok@sd\endcsname{\let\PY@it=\textit\def\PY@tc##1{\textcolor[rgb]{0.73,0.13,0.13}{##1}}}
\expandafter\def\csname PY@tok@si\endcsname{\let\PY@bf=\textbf\def\PY@tc##1{\textcolor[rgb]{0.73,0.40,0.53}{##1}}}
\expandafter\def\csname PY@tok@se\endcsname{\let\PY@bf=\textbf\def\PY@tc##1{\textcolor[rgb]{0.73,0.40,0.13}{##1}}}
\expandafter\def\csname PY@tok@sr\endcsname{\def\PY@tc##1{\textcolor[rgb]{0.73,0.40,0.53}{##1}}}
\expandafter\def\csname PY@tok@ss\endcsname{\def\PY@tc##1{\textcolor[rgb]{0.10,0.09,0.49}{##1}}}
\expandafter\def\csname PY@tok@sx\endcsname{\def\PY@tc##1{\textcolor[rgb]{0.00,0.50,0.00}{##1}}}
\expandafter\def\csname PY@tok@m\endcsname{\def\PY@tc##1{\textcolor[rgb]{0.40,0.40,0.40}{##1}}}
\expandafter\def\csname PY@tok@gh\endcsname{\let\PY@bf=\textbf\def\PY@tc##1{\textcolor[rgb]{0.00,0.00,0.50}{##1}}}
\expandafter\def\csname PY@tok@gu\endcsname{\let\PY@bf=\textbf\def\PY@tc##1{\textcolor[rgb]{0.50,0.00,0.50}{##1}}}
\expandafter\def\csname PY@tok@gd\endcsname{\def\PY@tc##1{\textcolor[rgb]{0.63,0.00,0.00}{##1}}}
\expandafter\def\csname PY@tok@gi\endcsname{\def\PY@tc##1{\textcolor[rgb]{0.00,0.63,0.00}{##1}}}
\expandafter\def\csname PY@tok@gr\endcsname{\def\PY@tc##1{\textcolor[rgb]{1.00,0.00,0.00}{##1}}}
\expandafter\def\csname PY@tok@ge\endcsname{\let\PY@it=\textit}
\expandafter\def\csname PY@tok@gs\endcsname{\let\PY@bf=\textbf}
\expandafter\def\csname PY@tok@gp\endcsname{\let\PY@bf=\textbf\def\PY@tc##1{\textcolor[rgb]{0.00,0.00,0.50}{##1}}}
\expandafter\def\csname PY@tok@go\endcsname{\def\PY@tc##1{\textcolor[rgb]{0.53,0.53,0.53}{##1}}}
\expandafter\def\csname PY@tok@gt\endcsname{\def\PY@tc##1{\textcolor[rgb]{0.00,0.27,0.87}{##1}}}
\expandafter\def\csname PY@tok@err\endcsname{\def\PY@bc##1{\setlength{\fboxsep}{0pt}\fcolorbox[rgb]{1.00,0.00,0.00}{1,1,1}{\strut ##1}}}
\expandafter\def\csname PY@tok@kc\endcsname{\let\PY@bf=\textbf\def\PY@tc##1{\textcolor[rgb]{0.00,0.50,0.00}{##1}}}
\expandafter\def\csname PY@tok@kd\endcsname{\let\PY@bf=\textbf\def\PY@tc##1{\textcolor[rgb]{0.00,0.50,0.00}{##1}}}
\expandafter\def\csname PY@tok@kn\endcsname{\let\PY@bf=\textbf\def\PY@tc##1{\textcolor[rgb]{0.00,0.50,0.00}{##1}}}
\expandafter\def\csname PY@tok@kr\endcsname{\let\PY@bf=\textbf\def\PY@tc##1{\textcolor[rgb]{0.00,0.50,0.00}{##1}}}
\expandafter\def\csname PY@tok@bp\endcsname{\def\PY@tc##1{\textcolor[rgb]{0.00,0.50,0.00}{##1}}}
\expandafter\def\csname PY@tok@fm\endcsname{\def\PY@tc##1{\textcolor[rgb]{0.00,0.00,1.00}{##1}}}
\expandafter\def\csname PY@tok@vc\endcsname{\def\PY@tc##1{\textcolor[rgb]{0.10,0.09,0.49}{##1}}}
\expandafter\def\csname PY@tok@vg\endcsname{\def\PY@tc##1{\textcolor[rgb]{0.10,0.09,0.49}{##1}}}
\expandafter\def\csname PY@tok@vi\endcsname{\def\PY@tc##1{\textcolor[rgb]{0.10,0.09,0.49}{##1}}}
\expandafter\def\csname PY@tok@vm\endcsname{\def\PY@tc##1{\textcolor[rgb]{0.10,0.09,0.49}{##1}}}
\expandafter\def\csname PY@tok@sa\endcsname{\def\PY@tc##1{\textcolor[rgb]{0.73,0.13,0.13}{##1}}}
\expandafter\def\csname PY@tok@sb\endcsname{\def\PY@tc##1{\textcolor[rgb]{0.73,0.13,0.13}{##1}}}
\expandafter\def\csname PY@tok@sc\endcsname{\def\PY@tc##1{\textcolor[rgb]{0.73,0.13,0.13}{##1}}}
\expandafter\def\csname PY@tok@dl\endcsname{\def\PY@tc##1{\textcolor[rgb]{0.73,0.13,0.13}{##1}}}
\expandafter\def\csname PY@tok@s2\endcsname{\def\PY@tc##1{\textcolor[rgb]{0.73,0.13,0.13}{##1}}}
\expandafter\def\csname PY@tok@sh\endcsname{\def\PY@tc##1{\textcolor[rgb]{0.73,0.13,0.13}{##1}}}
\expandafter\def\csname PY@tok@s1\endcsname{\def\PY@tc##1{\textcolor[rgb]{0.73,0.13,0.13}{##1}}}
\expandafter\def\csname PY@tok@mb\endcsname{\def\PY@tc##1{\textcolor[rgb]{0.40,0.40,0.40}{##1}}}
\expandafter\def\csname PY@tok@mf\endcsname{\def\PY@tc##1{\textcolor[rgb]{0.40,0.40,0.40}{##1}}}
\expandafter\def\csname PY@tok@mh\endcsname{\def\PY@tc##1{\textcolor[rgb]{0.40,0.40,0.40}{##1}}}
\expandafter\def\csname PY@tok@mi\endcsname{\def\PY@tc##1{\textcolor[rgb]{0.40,0.40,0.40}{##1}}}
\expandafter\def\csname PY@tok@il\endcsname{\def\PY@tc##1{\textcolor[rgb]{0.40,0.40,0.40}{##1}}}
\expandafter\def\csname PY@tok@mo\endcsname{\def\PY@tc##1{\textcolor[rgb]{0.40,0.40,0.40}{##1}}}
\expandafter\def\csname PY@tok@ch\endcsname{\let\PY@it=\textit\def\PY@tc##1{\textcolor[rgb]{0.25,0.50,0.50}{##1}}}
\expandafter\def\csname PY@tok@cm\endcsname{\let\PY@it=\textit\def\PY@tc##1{\textcolor[rgb]{0.25,0.50,0.50}{##1}}}
\expandafter\def\csname PY@tok@cpf\endcsname{\let\PY@it=\textit\def\PY@tc##1{\textcolor[rgb]{0.25,0.50,0.50}{##1}}}
\expandafter\def\csname PY@tok@c1\endcsname{\let\PY@it=\textit\def\PY@tc##1{\textcolor[rgb]{0.25,0.50,0.50}{##1}}}
\expandafter\def\csname PY@tok@cs\endcsname{\let\PY@it=\textit\def\PY@tc##1{\textcolor[rgb]{0.25,0.50,0.50}{##1}}}

\def\PYZbs{\char`\\}
\def\PYZus{\char`\_}
\def\PYZob{\char`\{}
\def\PYZcb{\char`\}}
\def\PYZca{\char`\^}
\def\PYZam{\char`\&}
\def\PYZlt{\char`\<}
\def\PYZgt{\char`\>}
\def\PYZsh{\char`\#}
\def\PYZpc{\char`\%}
\def\PYZdl{\char`\$}
\def\PYZhy{\char`\-}
\def\PYZsq{\char`\'}
\def\PYZdq{\char`\"}
\def\PYZti{\char`\~}
% for compatibility with earlier versions
\def\PYZat{@}
\def\PYZlb{[}
\def\PYZrb{]}
\makeatother


    % Exact colors from NB
    \definecolor{incolor}{rgb}{0.0, 0.0, 0.5}
    \definecolor{outcolor}{rgb}{0.545, 0.0, 0.0}



    
    % Prevent overflowing lines due to hard-to-break entities
    \sloppy 
    % Setup hyperref package
    \hypersetup{
      breaklinks=true,  % so long urls are correctly broken across lines
      colorlinks=true,
      urlcolor=urlcolor,
      linkcolor=linkcolor,
      citecolor=citecolor,
      }
    % Slightly bigger margins than the latex defaults
    
    \geometry{verbose,tmargin=1in,bmargin=1in,lmargin=1in,rmargin=1in}
    
    

    \begin{document}
    
    
    \maketitle
    
    

    
    \section{Taller 4}\label{taller-4}

Métodos Computacionales para Políticas Públicas - URosario

\textbf{Entrega: viernes 30-ago-2019 11:59 PM}

    \textbf{Julián Santiago Ramírez} julians.ramirez@urosario.edu.co

    \subsection{Instrucciones:}\label{instrucciones}

\begin{itemize}
\tightlist
\item
  Guarde una copia de este \emph{Jupyter Notebook} en su computador,
  idealmente en una carpeta destinada al material del curso.
\item
  Modifique el nombre del archivo del \emph{notebook}, agregando al
  final un guión inferior y su nombre y apellido, separados estos
  últimos por otro guión inferior. Por ejemplo, mi \emph{notebook} se
  llamaría: mcpp\_taller4\_santiago\_matallana
\item
  Marque el \emph{notebook} con su nombre y e-mail en el bloque verde
  arriba. Reemplace el texto "{[}Su nombre acá{]}" con su nombre y
  apellido. Similar para su e-mail.
\item
  Desarrolle la totalidad del taller sobre este \emph{notebook},
  insertando las celdas que sea necesario debajo de cada pregunta. Haga
  buen uso de las celdas para código y de las celdas tipo
  \emph{markdown} según el caso.
\item
  Recuerde salvar periódicamente sus avances.
\item
  Cuando termine el taller:

  \begin{enumerate}
  \def\labelenumi{\arabic{enumi}.}
  \tightlist
  \item
    Descárguelo en PDF.
  \item
    Suba los dos archivos (.pdf y .ipynb) a su repositorio en GitHub
    antes de la fecha y hora límites.
  \end{enumerate}
\end{itemize}

(Todos los ejercicios tienen el mismo valor.)

    \begin{center}\rule{0.5\linewidth}{\linethickness}\end{center}

    \subsection{Zelle, Exercises 6.8 (p.
159):}\label{zelle-exercises-6.8-p.-159}

\begin{itemize}
\tightlist
\item
  True/False: 1-10
\item
  Multiple choice: 2, 3, 6, 7, 10
\item
  Programming Exercises: 1, 3, 4, 11, 12, 13
\end{itemize}

    \subsection{True/False 1-10}\label{truefalse-1-10}

    \paragraph{1) Programmers rarely define their own
functions.}\label{programmers-rarely-define-their-own-functions.}

Rta: F

    \paragraph{2) A function may only be called at one place in a
program.}\label{a-function-may-only-be-called-at-one-place-in-a-program.}

Rta: F

    \paragraph{3) Information can be passed into a function through
parameters}\label{information-can-be-passed-into-a-function-through-parameters}

Rta: T

    \paragraph{4) Every Python function returns some
value.}\label{every-python-function-returns-some-value.}

Rta: T. Aclaración: El libro menciona que en caso de que dentro de la
función no se defina un return o algun print, la función retornaría
"None". Este "none" no es visto por el usuario, ni es almacenado en
alguna variable.

    \paragraph{5) In Python, some parameters are passed by
reference.}\label{in-python-some-parameters-are-passed-by-reference.}

Rta: F

    \paragraph{6) In Python, a function can return only one
value}\label{in-python-a-function-can-return-only-one-value}

Rta: F

    \paragraph{7) Python functions can never modify a
parameter.}\label{python-functions-can-never-modify-a-parameter.}

Rta: T

    \paragraph{8) One reason to use functions is to reduce code
duplication.}\label{one-reason-to-use-functions-is-to-reduce-code-duplication.}

Rta: T

    \paragraph{9.)Variables defined in a function are local to that
function}\label{variables-defined-in-a-function-are-local-to-that-function}

Rta: T

    \paragraph{10) It's a bad idea to define new functions if it makes a
program
longer.}\label{its-a-bad-idea-to-define-new-functions-if-it-makes-a-program-longer.}

Rta: F

    \subsection{Multiple choice}\label{multiple-choice}

    \paragraph{2. A Python function definition begins
with}\label{a-python-function-definition-begins-with}

\begin{enumerate}
\def\labelenumi{\alph{enumi})}
\tightlist
\item
  def
\item
  define
\item
  function
\item
  defun
\end{enumerate}

Rta: a

    \paragraph{3. A function can send output back to the program with
a(n)}\label{a-function-can-send-output-back-to-the-program-with-an}

\begin{enumerate}
\def\labelenumi{\alph{enumi})}
\tightlist
\item
  return b) print c) assignment d) SASE
\end{enumerate}

Rta: a

    \paragraph{6. In Python, actual parameters are passed to
functions}\label{in-python-actual-parameters-are-passed-to-functions}

\begin{enumerate}
\def\labelenumi{\alph{enumi})}
\tightlist
\item
  by value b) by reference c) at random d) by networking
\end{enumerate}

Rta: a

    \paragraph{7. Which of the following is not a reason to use
functions?}\label{which-of-the-following-is-not-a-reason-to-use-functions}

\begin{enumerate}
\def\labelenumi{\alph{enumi})}
\item
  to reduce code duplication
\item
  to make a program more modular
\item
  to make a program more self-documenting
\item
  to demonstrate intellectual superiority
\end{enumerate}

Rta: d

    \paragraph{10. A function can modify the value of an actual parameter
only if
it's}\label{a-function-can-modify-the-value-of-an-actual-parameter-only-if-its}

\begin{enumerate}
\def\labelenumi{\alph{enumi})}
\tightlist
\item
  mutable b) a list c) passed by reference d) a variable
\end{enumerate}

Rta: a

    \subsection{Programming exercises}\label{programming-exercises}

    \subsubsection{1. Write a program to print the lyrics of the song ``Old
MacDonald.'' Your program should print the lyrics for five different
animals, similar to the example verse
below.}\label{write-a-program-to-print-the-lyrics-of-the-song-old-macdonald.-your-program-should-print-the-lyrics-for-five-different-animals-similar-to-the-example-verse-below.}

    \begin{Verbatim}[commandchars=\\\{\}]
{\color{incolor}In [{\color{incolor}4}]:} \PY{k}{def} \PY{n+nf}{principal}\PY{p}{(}\PY{p}{)}\PY{p}{:}
            \PY{n+nb}{print}\PY{p}{(}\PY{l+s+s2}{\PYZdq{}}\PY{l+s+s2}{Old McDonald had a farm, Ee\PYZhy{}igh, Ee\PYZhy{}igh, Oh!}\PY{l+s+s2}{\PYZdq{}}\PY{p}{)}
        
        \PY{k}{def} \PY{n+nf}{parte\PYZus{}animal}\PY{p}{(}\PY{n}{animal}\PY{p}{,}\PY{n}{sonido}\PY{p}{)}\PY{p}{:}
            \PY{n}{palabra\PYZus{}1}\PY{o}{=}\PY{l+s+s2}{\PYZdq{}}\PY{l+s+s2}{And on that farm he had a }\PY{l+s+s2}{\PYZdq{}}\PY{o}{+} \PY{n}{animal}\PY{o}{+}\PY{l+s+s2}{\PYZdq{}}\PY{l+s+s2}{,Ee\PYZhy{}igh, Ee\PYZhy{}igh, Oh!}\PY{l+s+s2}{\PYZdq{}} 
            \PY{n}{palabra\PYZus{}2}\PY{o}{=}\PY{l+s+s2}{\PYZdq{}}\PY{l+s+s2}{With a }\PY{l+s+s2}{\PYZdq{}}\PY{o}{+} \PY{n}{sonido}\PY{o}{+}\PY{l+s+s2}{\PYZdq{}}\PY{l+s+s2}{,}\PY{l+s+s2}{\PYZdq{}}\PY{o}{+}\PY{n}{sonido}\PY{o}{+} \PY{l+s+s2}{\PYZdq{}}\PY{l+s+s2}{ here and a }\PY{l+s+s2}{\PYZdq{}}\PY{o}{+} \PY{n}{sonido}\PY{o}{+}\PY{l+s+s2}{\PYZdq{}}\PY{l+s+s2}{,}\PY{l+s+s2}{\PYZdq{}}\PY{o}{+}\PY{n}{sonido}\PY{o}{+}\PY{l+s+s2}{\PYZdq{}}\PY{l+s+s2}{ there}\PY{l+s+s2}{\PYZdq{}}
            \PY{n}{palabra\PYZus{}3}\PY{o}{=}\PY{l+s+s2}{\PYZdq{}}\PY{l+s+s2}{Here a }\PY{l+s+s2}{\PYZdq{}}\PY{o}{+} \PY{n}{sonido}\PY{o}{+}\PY{l+s+s2}{\PYZdq{}}\PY{l+s+s2}{,}\PY{l+s+s2}{\PYZdq{}}\PY{o}{+}\PY{l+s+s2}{\PYZdq{}}\PY{l+s+s2}{ there a }\PY{l+s+s2}{\PYZdq{}}\PY{o}{+} \PY{n}{sonido}\PY{o}{+}\PY{l+s+s2}{\PYZdq{}}\PY{l+s+s2}{, everywhere a }\PY{l+s+s2}{\PYZdq{}}\PY{o}{+}\PY{n}{sonido}\PY{o}{+}\PY{l+s+s2}{\PYZdq{}}\PY{l+s+s2}{,}\PY{l+s+s2}{\PYZdq{}}\PY{o}{+}\PY{n}{sonido}
            \PY{n+nb}{print}\PY{p}{(}\PY{n}{palabra\PYZus{}1}\PY{p}{)}
            \PY{n+nb}{print}\PY{p}{(}\PY{n}{palabra\PYZus{}2}\PY{p}{)}
            \PY{n+nb}{print}\PY{p}{(}\PY{n}{palabra\PYZus{}3}\PY{p}{)}
            
        \PY{k}{def} \PY{n+nf}{total}\PY{p}{(}\PY{n}{animales}\PY{p}{)}\PY{p}{:}
            \PY{n}{principal}\PY{p}{(}\PY{p}{)}
            \PY{k}{for} \PY{n}{i} \PY{o+ow}{in} \PY{n+nb}{range}\PY{p}{(}\PY{l+m+mi}{0}\PY{p}{,}\PY{n+nb}{len}\PY{p}{(}\PY{n}{animales}\PY{p}{)}\PY{p}{)}\PY{p}{:}
                \PY{n}{parte\PYZus{}animal}\PY{p}{(}\PY{n}{animales}\PY{p}{[}\PY{n}{i}\PY{p}{]}\PY{p}{[}\PY{l+m+mi}{0}\PY{p}{]}\PY{p}{,}\PY{n}{animales}\PY{p}{[}\PY{n}{i}\PY{p}{]}\PY{p}{[}\PY{l+m+mi}{1}\PY{p}{]}\PY{p}{)}
                \PY{n}{principal}\PY{p}{(}\PY{p}{)}
            
        \PY{n}{lista\PYZus{}a}\PY{o}{=}\PY{p}{[}\PY{p}{[}\PY{l+s+s2}{\PYZdq{}}\PY{l+s+s2}{caballo}\PY{l+s+s2}{\PYZdq{}}\PY{p}{,}\PY{l+s+s2}{\PYZdq{}}\PY{l+s+s2}{brrrrr}\PY{l+s+s2}{\PYZdq{}}\PY{p}{]}\PY{p}{,}\PY{p}{[}\PY{l+s+s2}{\PYZdq{}}\PY{l+s+s2}{vaca}\PY{l+s+s2}{\PYZdq{}}\PY{p}{,}\PY{l+s+s2}{\PYZdq{}}\PY{l+s+s2}{moo}\PY{l+s+s2}{\PYZdq{}}\PY{p}{]}\PY{p}{,}\PY{p}{[}\PY{l+s+s2}{\PYZdq{}}\PY{l+s+s2}{perro}\PY{l+s+s2}{\PYZdq{}}\PY{p}{,}\PY{l+s+s2}{\PYZdq{}}\PY{l+s+s2}{wof}\PY{l+s+s2}{\PYZdq{}}\PY{p}{]}\PY{p}{,}\PY{p}{[}\PY{l+s+s2}{\PYZdq{}}\PY{l+s+s2}{gato}\PY{l+s+s2}{\PYZdq{}}\PY{p}{,}\PY{l+s+s2}{\PYZdq{}}\PY{l+s+s2}{miau}\PY{l+s+s2}{\PYZdq{}}\PY{p}{]}\PY{p}{,}\PY{p}{[}\PY{l+s+s2}{\PYZdq{}}\PY{l+s+s2}{pollo}\PY{l+s+s2}{\PYZdq{}}\PY{p}{,}\PY{l+s+s2}{\PYZdq{}}\PY{l+s+s2}{pio}\PY{l+s+s2}{\PYZdq{}}\PY{p}{]}\PY{p}{]}  
        \PY{n}{total}\PY{p}{(}\PY{n}{lista\PYZus{}a}\PY{p}{)}
\end{Verbatim}


    \begin{Verbatim}[commandchars=\\\{\}]
Old McDonald had a farm, Ee-igh, Ee-igh, Oh!
And on that farm he had a caballo,Ee-igh, Ee-igh, Oh!
With a brrrrr,brrrrr here and a brrrrr,brrrrr there
Here a brrrrr, there a brrrrr, everywhere a brrrrr,brrrrr
Old McDonald had a farm, Ee-igh, Ee-igh, Oh!
And on that farm he had a vaca,Ee-igh, Ee-igh, Oh!
With a moo,moo here and a moo,moo there
Here a moo, there a moo, everywhere a moo,moo
Old McDonald had a farm, Ee-igh, Ee-igh, Oh!
And on that farm he had a perro,Ee-igh, Ee-igh, Oh!
With a wof,wof here and a wof,wof there
Here a wof, there a wof, everywhere a wof,wof
Old McDonald had a farm, Ee-igh, Ee-igh, Oh!
And on that farm he had a gato,Ee-igh, Ee-igh, Oh!
With a miau,miau here and a miau,miau there
Here a miau, there a miau, everywhere a miau,miau
Old McDonald had a farm, Ee-igh, Ee-igh, Oh!
And on that farm he had a pollo,Ee-igh, Ee-igh, Oh!
With a pio,pio here and a pio,pio there
Here a pio, there a pio, everywhere a pio,pio
Old McDonald had a farm, Ee-igh, Ee-igh, Oh!

    \end{Verbatim}

    \subsubsection{3. Write definitions for these functions:
sphereArea(radius) Returns the surface area of a sphere having the given
radius. sphereVolume(radius) Returns the volume of a sphere having the
given radius. Use your functions to solve Programming Exercise 1 from
Chapter
3.}\label{write-definitions-for-these-functions-spherearearadius-returns-the-surface-area-of-a-sphere-having-the-given-radius.-spherevolumeradius-returns-the-volume-of-a-sphere-having-the-given-radius.-use-your-functions-to-solve-programming-exercise-1-from-chapter-3.}

    \begin{Verbatim}[commandchars=\\\{\}]
{\color{incolor}In [{\color{incolor}10}]:} \PY{c+c1}{\PYZsh{} Variabl global}
         \PY{n}{pi}\PY{o}{=}\PY{l+m+mf}{3.14159265359}
         
         \PY{k}{def} \PY{n+nf}{sphereArea}\PY{p}{(}\PY{n}{radius}\PY{p}{)}\PY{p}{:}
             \PY{c+c1}{\PYZsh{} Formula del área}
             \PY{n}{area} \PY{o}{=} \PY{l+m+mi}{4}\PY{o}{*}\PY{n}{pi}\PY{o}{*}\PY{n}{radius}\PY{o}{*}\PY{o}{*}\PY{l+m+mi}{2} 
             \PY{k}{return} \PY{n}{area}
         
         \PY{k}{def} \PY{n+nf}{sphereVolume}\PY{p}{(}\PY{n}{radius}\PY{p}{)}\PY{p}{:}
             \PY{c+c1}{\PYZsh{} Formula del volumen}
             \PY{n}{volumen} \PY{o}{=} \PY{p}{(}\PY{l+m+mi}{4}\PY{o}{/}\PY{l+m+mi}{3}\PY{o}{*}\PY{n}{pi}\PY{o}{*}\PY{n}{radius}\PY{o}{*}\PY{o}{*}\PY{l+m+mi}{3}\PY{p}{)} 
             \PY{k}{return} \PY{n}{volumen}
         
         \PY{k}{def} \PY{n+nf}{principal}\PY{p}{(}\PY{p}{)}\PY{p}{:}
             \PY{n+nb}{print}\PY{p}{(}\PY{l+s+s2}{\PYZdq{}}\PY{l+s+s2}{Calculamos el área y volumen de una esfera, con un radio que da el usuario}\PY{l+s+s2}{\PYZdq{}}\PY{p}{)}
             \PY{n}{radio} \PY{o}{=} \PY{n+nb}{float}\PY{p}{(}\PY{n+nb}{input}\PY{p}{(}\PY{l+s+s2}{\PYZdq{}}\PY{l+s+s2}{Radio de la esfera}\PY{l+s+s2}{\PYZdq{}}\PY{p}{)}\PY{p}{)}
             \PY{n}{vol} \PY{o}{=} \PY{n}{sphereVolume}\PY{p}{(}\PY{n}{radio}\PY{p}{)}
             \PY{n}{area} \PY{o}{=} \PY{n}{sphereArea}\PY{p}{(}\PY{n}{radio}\PY{p}{)}
             \PY{n+nb}{print}\PY{p}{(}\PY{l+s+s2}{\PYZdq{}}\PY{l+s+s2}{Volumen de la esfera: }\PY{l+s+s2}{\PYZdq{}}\PY{o}{+} \PY{n+nb}{str}\PY{p}{(}\PY{n}{vol}\PY{p}{)} \PY{o}{+}\PY{l+s+s1}{\PYZsq{}}\PY{l+s+se}{\PYZbs{}n}\PY{l+s+s1}{\PYZsq{}}\PY{o}{+} \PY{l+s+s2}{\PYZdq{}}\PY{l+s+s2}{Area de la esfera :}\PY{l+s+s2}{\PYZdq{}}\PY{o}{+} \PY{n+nb}{str}\PY{p}{(}\PY{n}{area}\PY{p}{)}\PY{p}{)}
         
         \PY{n}{principal}\PY{p}{(}\PY{p}{)}
\end{Verbatim}


    \begin{Verbatim}[commandchars=\\\{\}]
Calculamos el área y volumen de una esfera, con un radio que da el usuario
Radio de la esfera2
Volumen de la esfera: 33.510321638293334
Area de la esfera :50.26548245744

    \end{Verbatim}

    \subsubsection{4. Write definitions for the following two functions:
sumN(n) returns the sum of the first n natural numbers. sumNCubes(n)
returns the sum of the cubes of the first n natural
numbers.}\label{write-definitions-for-the-following-two-functions-sumnn-returns-the-sum-of-the-first-n-natural-numbers.-sumncubesn-returns-the-sum-of-the-cubes-of-the-first-n-natural-numbers.}

    \begin{Verbatim}[commandchars=\\\{\}]
{\color{incolor}In [{\color{incolor}18}]:} \PY{k}{def} \PY{n+nf}{sumN}\PY{p}{(}\PY{n}{n}\PY{p}{)}\PY{p}{:}
             \PY{n}{final}\PY{o}{=}\PY{n}{n}\PY{o}{+}\PY{l+m+mi}{1}
             \PY{n}{tot}\PY{o}{=}\PY{n+nb}{sum}\PY{p}{(}\PY{n+nb}{range}\PY{p}{(}\PY{l+m+mi}{0}\PY{p}{,}\PY{n}{final}\PY{p}{)}\PY{p}{)}
             \PY{k}{return} \PY{n}{tot}
         
         \PY{k}{def} \PY{n+nf}{sumNCubes}\PY{p}{(}\PY{n}{n}\PY{p}{)}\PY{p}{:}
             \PY{n}{final}\PY{o}{=}\PY{n}{n}\PY{o}{+}\PY{l+m+mi}{1}
             \PY{n}{lista}\PY{o}{=}\PY{p}{[}\PY{p}{]}
             \PY{k}{for} \PY{n}{i} \PY{o+ow}{in} \PY{n+nb}{range}\PY{p}{(}\PY{l+m+mi}{0}\PY{p}{,}\PY{n}{final}\PY{p}{)}\PY{p}{:}
                 \PY{n}{lista}\PY{o}{.}\PY{n}{append}\PY{p}{(}\PY{n}{i}\PY{o}{*}\PY{o}{*}\PY{l+m+mi}{3}\PY{p}{)}
             \PY{n}{tot}\PY{o}{=}\PY{n+nb}{sum}\PY{p}{(}\PY{n}{lista}\PY{p}{)}
             \PY{n+nb}{print}\PY{p}{(}\PY{l+s+s2}{\PYZdq{}}\PY{l+s+s2}{lista de cubos:}\PY{l+s+s2}{\PYZdq{}}\PY{p}{,}\PY{n}{lista}\PY{p}{)}
             \PY{k}{return} \PY{n}{tot}
         
         \PY{k}{def} \PY{n+nf}{prin}\PY{p}{(}\PY{n}{n}\PY{p}{)}\PY{p}{:}
             \PY{n}{su} \PY{o}{=} \PY{n}{sumN}\PY{p}{(}\PY{n}{n}\PY{p}{)}
             \PY{n}{cubos} \PY{o}{=} \PY{n}{sumNCubes}\PY{p}{(}\PY{n}{n}\PY{p}{)}
             \PY{n+nb}{print}\PY{p}{(}\PY{l+s+s2}{\PYZdq{}}\PY{l+s+s2}{La suma de todos los numeros hasta }\PY{l+s+s2}{\PYZdq{}}\PY{o}{+}\PY{n+nb}{str}\PY{p}{(}\PY{n}{n}\PY{p}{)}\PY{o}{+}\PY{l+s+s2}{\PYZdq{}}\PY{l+s+s2}{ es:}\PY{l+s+s2}{\PYZdq{}}\PY{p}{,} \PY{n}{su}\PY{p}{)}
             \PY{n+nb}{print}\PY{p}{(}\PY{l+s+s2}{\PYZdq{}}\PY{l+s+s2}{La suma de todos los cubos hasta }\PY{l+s+s2}{\PYZdq{}}\PY{o}{+}\PY{n+nb}{str}\PY{p}{(}\PY{n}{n}\PY{p}{)}\PY{o}{+}\PY{l+s+s2}{\PYZdq{}}\PY{l+s+s2}{ es: }\PY{l+s+s2}{\PYZdq{}}\PY{p}{,} \PY{n}{cubos}\PY{p}{)}
         
         \PY{n}{prin}\PY{p}{(}\PY{l+m+mi}{7}\PY{p}{)}
\end{Verbatim}


    \begin{Verbatim}[commandchars=\\\{\}]
lista de cubos: [0, 1, 8, 27, 64, 125, 216, 343]
La suma de todos los numeros hasta 7 es: 28
La suma de todos los cubos hasta 7 es:  784

    \end{Verbatim}

    \subsubsection{11. Write and test a function to meet this specification.
squareEach(nums) nums is a list of numbers. Modifies the list by
squaring each
entry.}\label{write-and-test-a-function-to-meet-this-specification.-squareeachnums-nums-is-a-list-of-numbers.-modifies-the-list-by-squaring-each-entry.}

    \begin{Verbatim}[commandchars=\\\{\}]
{\color{incolor}In [{\color{incolor}22}]:} \PY{k}{def} \PY{n+nf}{squareEach}\PY{p}{(}\PY{n}{nums}\PY{p}{)}\PY{p}{:}
             \PY{k}{for} \PY{n}{i} \PY{o+ow}{in} \PY{n+nb}{range}\PY{p}{(}\PY{l+m+mi}{0}\PY{p}{,}\PY{n+nb}{len}\PY{p}{(}\PY{n}{nums}\PY{p}{)}\PY{p}{)}\PY{p}{:}
                 \PY{n}{nums}\PY{p}{[}\PY{n}{i}\PY{p}{]} \PY{o}{=} \PY{n}{nums}\PY{p}{[}\PY{n}{i}\PY{p}{]}\PY{o}{*}\PY{o}{*}\PY{l+m+mi}{2}
         
         \PY{k}{def} \PY{n+nf}{principal}\PY{p}{(}\PY{n}{lista}\PY{p}{)}\PY{p}{:}
             \PY{n+nb}{print}\PY{p}{(}\PY{l+s+s2}{\PYZdq{}}\PY{l+s+s2}{lista:}\PY{l+s+s2}{\PYZdq{}}\PY{p}{)}
             \PY{n+nb}{print}\PY{p}{(}\PY{n}{lista}\PY{p}{)}
             \PY{n}{squareEach}\PY{p}{(}\PY{n}{lista}\PY{p}{)}
             \PY{n+nb}{print}\PY{p}{(}\PY{l+s+s2}{\PYZdq{}}\PY{l+s+s2}{Lista modificada:}\PY{l+s+s2}{\PYZdq{}}\PY{p}{)}
             \PY{n+nb}{print}\PY{p}{(}\PY{n}{lista}\PY{p}{)}
         
         \PY{n}{lista}\PY{o}{=}\PY{p}{[}\PY{l+m+mi}{2}\PY{p}{,}\PY{l+m+mi}{3}\PY{p}{,}\PY{l+m+mi}{4}\PY{p}{,}\PY{l+m+mi}{5}\PY{p}{,}\PY{l+m+mi}{6}\PY{p}{,}\PY{l+m+mi}{7}\PY{p}{]}
         \PY{n}{principal}\PY{p}{(}\PY{n}{lista}\PY{p}{)}
\end{Verbatim}


    \begin{Verbatim}[commandchars=\\\{\}]
lista:
[2, 3, 4, 5, 6, 7]
Lista modificada:
[4, 9, 16, 25, 36, 49]

    \end{Verbatim}

    \subsubsection{12. Write and test a function to meet this specification.
sumList(nums) nums is a list of numbers. Returns the sum of the numbers
in the
list.}\label{write-and-test-a-function-to-meet-this-specification.-sumlistnums-nums-is-a-list-of-numbers.-returns-the-sum-of-the-numbers-in-the-list.}

    \begin{Verbatim}[commandchars=\\\{\}]
{\color{incolor}In [{\color{incolor}25}]:} \PY{k}{def} \PY{n+nf}{sumList}\PY{p}{(}\PY{n}{nums}\PY{p}{)}\PY{p}{:}
             \PY{k}{return} \PY{n+nb}{sum}\PY{p}{(}\PY{n}{nums}\PY{p}{)}
         
         \PY{k}{def} \PY{n+nf}{principal}\PY{p}{(}\PY{n}{lista}\PY{p}{)}\PY{p}{:}
             \PY{n+nb}{print}\PY{p}{(}\PY{l+s+s2}{\PYZdq{}}\PY{l+s+s2}{Lista: }\PY{l+s+s2}{\PYZdq{}}\PY{p}{,}\PY{n}{lista}\PY{p}{)}
             \PY{n+nb}{print}\PY{p}{(}\PY{l+s+s2}{\PYZdq{}}\PY{l+s+s2}{La suma de la lista: }\PY{l+s+s2}{\PYZdq{}}\PY{p}{)}
             \PY{n}{suma}\PY{o}{=}\PY{n}{sumList}\PY{p}{(}\PY{n}{lista}\PY{p}{)}
             \PY{n+nb}{print}\PY{p}{(}\PY{n}{suma}\PY{p}{)}
             \PY{k}{return} \PY{n}{suma}
         
         \PY{n}{lista}\PY{o}{=}\PY{p}{[}\PY{l+m+mi}{1}\PY{p}{,}\PY{l+m+mi}{2}\PY{p}{,}\PY{l+m+mi}{3}\PY{p}{,}\PY{l+m+mi}{4}\PY{p}{,}\PY{l+m+mi}{5}\PY{p}{,}\PY{l+m+mi}{6}\PY{p}{,}\PY{l+m+mi}{7}\PY{p}{,}\PY{l+m+mi}{8}\PY{p}{,}\PY{l+m+mi}{9}\PY{p}{]}
         \PY{n}{a}\PY{o}{=}\PY{n}{principal}\PY{p}{(}\PY{n}{lista}\PY{p}{)}
\end{Verbatim}


    \begin{Verbatim}[commandchars=\\\{\}]
Lista:  [1, 2, 3, 4, 5, 6, 7, 8, 9]
La suma de la lista: 
45

    \end{Verbatim}

    \subsubsection{13. Write and test a function to meet this specification.
toNumbers(strList) strList is a list of strings, each of which
represents a number. Modi- fies each entry in the list by converting it
to a
number.}\label{write-and-test-a-function-to-meet-this-specification.-tonumbersstrlist-strlist-is-a-list-of-strings-each-of-which-represents-a-number.-modi--fies-each-entry-in-the-list-by-converting-it-to-a-number.}

    \begin{Verbatim}[commandchars=\\\{\}]
{\color{incolor}In [{\color{incolor}28}]:} \PY{k}{def} \PY{n+nf}{toNumbers}\PY{p}{(}\PY{n}{strList}\PY{p}{)}\PY{p}{:}
             \PY{k}{for} \PY{n}{i} \PY{o+ow}{in} \PY{n+nb}{range}\PY{p}{(}\PY{l+m+mi}{0}\PY{p}{,}\PY{n+nb}{len}\PY{p}{(}\PY{n}{strList}\PY{p}{)}\PY{p}{)}\PY{p}{:}
                 \PY{n}{strList}\PY{p}{[}\PY{n}{i}\PY{p}{]} \PY{o}{=} \PY{n+nb}{int}\PY{p}{(}\PY{n}{strList}\PY{p}{[}\PY{n}{i}\PY{p}{]}\PY{p}{)}
         
         \PY{k}{def} \PY{n+nf}{principal}\PY{p}{(}\PY{n}{lista}\PY{p}{)}\PY{p}{:}
             \PY{n+nb}{print}\PY{p}{(}\PY{l+s+s2}{\PYZdq{}}\PY{l+s+s2}{Lista de numeros en string: }\PY{l+s+s2}{\PYZdq{}}\PY{p}{)}
             \PY{n+nb}{print}\PY{p}{(}\PY{n}{lista}\PY{p}{)}
             \PY{n}{toNumbers}\PY{p}{(}\PY{n}{lista}\PY{p}{)}
             \PY{n+nb}{print}\PY{p}{(}\PY{l+s+s2}{\PYZdq{}}\PY{l+s+s2}{Lista modificada: }\PY{l+s+s2}{\PYZdq{}}\PY{p}{)}
             \PY{n+nb}{print}\PY{p}{(}\PY{n}{lista}\PY{p}{)}
         
         \PY{n}{lista}\PY{o}{=}\PY{p}{[}\PY{l+s+s2}{\PYZdq{}}\PY{l+s+s2}{1}\PY{l+s+s2}{\PYZdq{}}\PY{p}{,}\PY{l+s+s2}{\PYZdq{}}\PY{l+s+s2}{2}\PY{l+s+s2}{\PYZdq{}}\PY{p}{,}\PY{l+s+s2}{\PYZdq{}}\PY{l+s+s2}{3}\PY{l+s+s2}{\PYZdq{}}\PY{p}{,}\PY{l+s+s2}{\PYZdq{}}\PY{l+s+s2}{4}\PY{l+s+s2}{\PYZdq{}}\PY{p}{,}\PY{l+s+s2}{\PYZdq{}}\PY{l+s+s2}{5}\PY{l+s+s2}{\PYZdq{}}\PY{p}{]}
         \PY{n}{principal}\PY{p}{(}\PY{n}{lista}\PY{p}{)}
\end{Verbatim}


    \begin{Verbatim}[commandchars=\\\{\}]
Lista de numeros en string: 
['1', '2', '3', '4', '5']
Lista modificada: 
[1, 2, 3, 4, 5]

    \end{Verbatim}


    % Add a bibliography block to the postdoc
    
    
    
    \end{document}
