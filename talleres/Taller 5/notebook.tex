
% Default to the notebook output style

    


% Inherit from the specified cell style.




    
\documentclass[11pt]{article}

    
    
    \usepackage[T1]{fontenc}
    % Nicer default font (+ math font) than Computer Modern for most use cases
    \usepackage{mathpazo}

    % Basic figure setup, for now with no caption control since it's done
    % automatically by Pandoc (which extracts ![](path) syntax from Markdown).
    \usepackage{graphicx}
    % We will generate all images so they have a width \maxwidth. This means
    % that they will get their normal width if they fit onto the page, but
    % are scaled down if they would overflow the margins.
    \makeatletter
    \def\maxwidth{\ifdim\Gin@nat@width>\linewidth\linewidth
    \else\Gin@nat@width\fi}
    \makeatother
    \let\Oldincludegraphics\includegraphics
    % Set max figure width to be 80% of text width, for now hardcoded.
    \renewcommand{\includegraphics}[1]{\Oldincludegraphics[width=.8\maxwidth]{#1}}
    % Ensure that by default, figures have no caption (until we provide a
    % proper Figure object with a Caption API and a way to capture that
    % in the conversion process - todo).
    \usepackage{caption}
    \DeclareCaptionLabelFormat{nolabel}{}
    \captionsetup{labelformat=nolabel}

    \usepackage{adjustbox} % Used to constrain images to a maximum size 
    \usepackage{xcolor} % Allow colors to be defined
    \usepackage{enumerate} % Needed for markdown enumerations to work
    \usepackage{geometry} % Used to adjust the document margins
    \usepackage{amsmath} % Equations
    \usepackage{amssymb} % Equations
    \usepackage{textcomp} % defines textquotesingle
    % Hack from http://tex.stackexchange.com/a/47451/13684:
    \AtBeginDocument{%
        \def\PYZsq{\textquotesingle}% Upright quotes in Pygmentized code
    }
    \usepackage{upquote} % Upright quotes for verbatim code
    \usepackage{eurosym} % defines \euro
    \usepackage[mathletters]{ucs} % Extended unicode (utf-8) support
    \usepackage[utf8x]{inputenc} % Allow utf-8 characters in the tex document
    \usepackage{fancyvrb} % verbatim replacement that allows latex
    \usepackage{grffile} % extends the file name processing of package graphics 
                         % to support a larger range 
    % The hyperref package gives us a pdf with properly built
    % internal navigation ('pdf bookmarks' for the table of contents,
    % internal cross-reference links, web links for URLs, etc.)
    \usepackage{hyperref}
    \usepackage{longtable} % longtable support required by pandoc >1.10
    \usepackage{booktabs}  % table support for pandoc > 1.12.2
    \usepackage[inline]{enumitem} % IRkernel/repr support (it uses the enumerate* environment)
    \usepackage[normalem]{ulem} % ulem is needed to support strikethroughs (\sout)
                                % normalem makes italics be italics, not underlines
    

    
    
    % Colors for the hyperref package
    \definecolor{urlcolor}{rgb}{0,.145,.698}
    \definecolor{linkcolor}{rgb}{.71,0.21,0.01}
    \definecolor{citecolor}{rgb}{.12,.54,.11}

    % ANSI colors
    \definecolor{ansi-black}{HTML}{3E424D}
    \definecolor{ansi-black-intense}{HTML}{282C36}
    \definecolor{ansi-red}{HTML}{E75C58}
    \definecolor{ansi-red-intense}{HTML}{B22B31}
    \definecolor{ansi-green}{HTML}{00A250}
    \definecolor{ansi-green-intense}{HTML}{007427}
    \definecolor{ansi-yellow}{HTML}{DDB62B}
    \definecolor{ansi-yellow-intense}{HTML}{B27D12}
    \definecolor{ansi-blue}{HTML}{208FFB}
    \definecolor{ansi-blue-intense}{HTML}{0065CA}
    \definecolor{ansi-magenta}{HTML}{D160C4}
    \definecolor{ansi-magenta-intense}{HTML}{A03196}
    \definecolor{ansi-cyan}{HTML}{60C6C8}
    \definecolor{ansi-cyan-intense}{HTML}{258F8F}
    \definecolor{ansi-white}{HTML}{C5C1B4}
    \definecolor{ansi-white-intense}{HTML}{A1A6B2}

    % commands and environments needed by pandoc snippets
    % extracted from the output of `pandoc -s`
    \providecommand{\tightlist}{%
      \setlength{\itemsep}{0pt}\setlength{\parskip}{0pt}}
    \DefineVerbatimEnvironment{Highlighting}{Verbatim}{commandchars=\\\{\}}
    % Add ',fontsize=\small' for more characters per line
    \newenvironment{Shaded}{}{}
    \newcommand{\KeywordTok}[1]{\textcolor[rgb]{0.00,0.44,0.13}{\textbf{{#1}}}}
    \newcommand{\DataTypeTok}[1]{\textcolor[rgb]{0.56,0.13,0.00}{{#1}}}
    \newcommand{\DecValTok}[1]{\textcolor[rgb]{0.25,0.63,0.44}{{#1}}}
    \newcommand{\BaseNTok}[1]{\textcolor[rgb]{0.25,0.63,0.44}{{#1}}}
    \newcommand{\FloatTok}[1]{\textcolor[rgb]{0.25,0.63,0.44}{{#1}}}
    \newcommand{\CharTok}[1]{\textcolor[rgb]{0.25,0.44,0.63}{{#1}}}
    \newcommand{\StringTok}[1]{\textcolor[rgb]{0.25,0.44,0.63}{{#1}}}
    \newcommand{\CommentTok}[1]{\textcolor[rgb]{0.38,0.63,0.69}{\textit{{#1}}}}
    \newcommand{\OtherTok}[1]{\textcolor[rgb]{0.00,0.44,0.13}{{#1}}}
    \newcommand{\AlertTok}[1]{\textcolor[rgb]{1.00,0.00,0.00}{\textbf{{#1}}}}
    \newcommand{\FunctionTok}[1]{\textcolor[rgb]{0.02,0.16,0.49}{{#1}}}
    \newcommand{\RegionMarkerTok}[1]{{#1}}
    \newcommand{\ErrorTok}[1]{\textcolor[rgb]{1.00,0.00,0.00}{\textbf{{#1}}}}
    \newcommand{\NormalTok}[1]{{#1}}
    
    % Additional commands for more recent versions of Pandoc
    \newcommand{\ConstantTok}[1]{\textcolor[rgb]{0.53,0.00,0.00}{{#1}}}
    \newcommand{\SpecialCharTok}[1]{\textcolor[rgb]{0.25,0.44,0.63}{{#1}}}
    \newcommand{\VerbatimStringTok}[1]{\textcolor[rgb]{0.25,0.44,0.63}{{#1}}}
    \newcommand{\SpecialStringTok}[1]{\textcolor[rgb]{0.73,0.40,0.53}{{#1}}}
    \newcommand{\ImportTok}[1]{{#1}}
    \newcommand{\DocumentationTok}[1]{\textcolor[rgb]{0.73,0.13,0.13}{\textit{{#1}}}}
    \newcommand{\AnnotationTok}[1]{\textcolor[rgb]{0.38,0.63,0.69}{\textbf{\textit{{#1}}}}}
    \newcommand{\CommentVarTok}[1]{\textcolor[rgb]{0.38,0.63,0.69}{\textbf{\textit{{#1}}}}}
    \newcommand{\VariableTok}[1]{\textcolor[rgb]{0.10,0.09,0.49}{{#1}}}
    \newcommand{\ControlFlowTok}[1]{\textcolor[rgb]{0.00,0.44,0.13}{\textbf{{#1}}}}
    \newcommand{\OperatorTok}[1]{\textcolor[rgb]{0.40,0.40,0.40}{{#1}}}
    \newcommand{\BuiltInTok}[1]{{#1}}
    \newcommand{\ExtensionTok}[1]{{#1}}
    \newcommand{\PreprocessorTok}[1]{\textcolor[rgb]{0.74,0.48,0.00}{{#1}}}
    \newcommand{\AttributeTok}[1]{\textcolor[rgb]{0.49,0.56,0.16}{{#1}}}
    \newcommand{\InformationTok}[1]{\textcolor[rgb]{0.38,0.63,0.69}{\textbf{\textit{{#1}}}}}
    \newcommand{\WarningTok}[1]{\textcolor[rgb]{0.38,0.63,0.69}{\textbf{\textit{{#1}}}}}
    
    
    % Define a nice break command that doesn't care if a line doesn't already
    % exist.
    \def\br{\hspace*{\fill} \\* }
    % Math Jax compatability definitions
    \def\gt{>}
    \def\lt{<}
    % Document parameters
    \title{mcpp\_taller5\_julian\_ramirez}
    
    
    

    % Pygments definitions
    
\makeatletter
\def\PY@reset{\let\PY@it=\relax \let\PY@bf=\relax%
    \let\PY@ul=\relax \let\PY@tc=\relax%
    \let\PY@bc=\relax \let\PY@ff=\relax}
\def\PY@tok#1{\csname PY@tok@#1\endcsname}
\def\PY@toks#1+{\ifx\relax#1\empty\else%
    \PY@tok{#1}\expandafter\PY@toks\fi}
\def\PY@do#1{\PY@bc{\PY@tc{\PY@ul{%
    \PY@it{\PY@bf{\PY@ff{#1}}}}}}}
\def\PY#1#2{\PY@reset\PY@toks#1+\relax+\PY@do{#2}}

\expandafter\def\csname PY@tok@w\endcsname{\def\PY@tc##1{\textcolor[rgb]{0.73,0.73,0.73}{##1}}}
\expandafter\def\csname PY@tok@c\endcsname{\let\PY@it=\textit\def\PY@tc##1{\textcolor[rgb]{0.25,0.50,0.50}{##1}}}
\expandafter\def\csname PY@tok@cp\endcsname{\def\PY@tc##1{\textcolor[rgb]{0.74,0.48,0.00}{##1}}}
\expandafter\def\csname PY@tok@k\endcsname{\let\PY@bf=\textbf\def\PY@tc##1{\textcolor[rgb]{0.00,0.50,0.00}{##1}}}
\expandafter\def\csname PY@tok@kp\endcsname{\def\PY@tc##1{\textcolor[rgb]{0.00,0.50,0.00}{##1}}}
\expandafter\def\csname PY@tok@kt\endcsname{\def\PY@tc##1{\textcolor[rgb]{0.69,0.00,0.25}{##1}}}
\expandafter\def\csname PY@tok@o\endcsname{\def\PY@tc##1{\textcolor[rgb]{0.40,0.40,0.40}{##1}}}
\expandafter\def\csname PY@tok@ow\endcsname{\let\PY@bf=\textbf\def\PY@tc##1{\textcolor[rgb]{0.67,0.13,1.00}{##1}}}
\expandafter\def\csname PY@tok@nb\endcsname{\def\PY@tc##1{\textcolor[rgb]{0.00,0.50,0.00}{##1}}}
\expandafter\def\csname PY@tok@nf\endcsname{\def\PY@tc##1{\textcolor[rgb]{0.00,0.00,1.00}{##1}}}
\expandafter\def\csname PY@tok@nc\endcsname{\let\PY@bf=\textbf\def\PY@tc##1{\textcolor[rgb]{0.00,0.00,1.00}{##1}}}
\expandafter\def\csname PY@tok@nn\endcsname{\let\PY@bf=\textbf\def\PY@tc##1{\textcolor[rgb]{0.00,0.00,1.00}{##1}}}
\expandafter\def\csname PY@tok@ne\endcsname{\let\PY@bf=\textbf\def\PY@tc##1{\textcolor[rgb]{0.82,0.25,0.23}{##1}}}
\expandafter\def\csname PY@tok@nv\endcsname{\def\PY@tc##1{\textcolor[rgb]{0.10,0.09,0.49}{##1}}}
\expandafter\def\csname PY@tok@no\endcsname{\def\PY@tc##1{\textcolor[rgb]{0.53,0.00,0.00}{##1}}}
\expandafter\def\csname PY@tok@nl\endcsname{\def\PY@tc##1{\textcolor[rgb]{0.63,0.63,0.00}{##1}}}
\expandafter\def\csname PY@tok@ni\endcsname{\let\PY@bf=\textbf\def\PY@tc##1{\textcolor[rgb]{0.60,0.60,0.60}{##1}}}
\expandafter\def\csname PY@tok@na\endcsname{\def\PY@tc##1{\textcolor[rgb]{0.49,0.56,0.16}{##1}}}
\expandafter\def\csname PY@tok@nt\endcsname{\let\PY@bf=\textbf\def\PY@tc##1{\textcolor[rgb]{0.00,0.50,0.00}{##1}}}
\expandafter\def\csname PY@tok@nd\endcsname{\def\PY@tc##1{\textcolor[rgb]{0.67,0.13,1.00}{##1}}}
\expandafter\def\csname PY@tok@s\endcsname{\def\PY@tc##1{\textcolor[rgb]{0.73,0.13,0.13}{##1}}}
\expandafter\def\csname PY@tok@sd\endcsname{\let\PY@it=\textit\def\PY@tc##1{\textcolor[rgb]{0.73,0.13,0.13}{##1}}}
\expandafter\def\csname PY@tok@si\endcsname{\let\PY@bf=\textbf\def\PY@tc##1{\textcolor[rgb]{0.73,0.40,0.53}{##1}}}
\expandafter\def\csname PY@tok@se\endcsname{\let\PY@bf=\textbf\def\PY@tc##1{\textcolor[rgb]{0.73,0.40,0.13}{##1}}}
\expandafter\def\csname PY@tok@sr\endcsname{\def\PY@tc##1{\textcolor[rgb]{0.73,0.40,0.53}{##1}}}
\expandafter\def\csname PY@tok@ss\endcsname{\def\PY@tc##1{\textcolor[rgb]{0.10,0.09,0.49}{##1}}}
\expandafter\def\csname PY@tok@sx\endcsname{\def\PY@tc##1{\textcolor[rgb]{0.00,0.50,0.00}{##1}}}
\expandafter\def\csname PY@tok@m\endcsname{\def\PY@tc##1{\textcolor[rgb]{0.40,0.40,0.40}{##1}}}
\expandafter\def\csname PY@tok@gh\endcsname{\let\PY@bf=\textbf\def\PY@tc##1{\textcolor[rgb]{0.00,0.00,0.50}{##1}}}
\expandafter\def\csname PY@tok@gu\endcsname{\let\PY@bf=\textbf\def\PY@tc##1{\textcolor[rgb]{0.50,0.00,0.50}{##1}}}
\expandafter\def\csname PY@tok@gd\endcsname{\def\PY@tc##1{\textcolor[rgb]{0.63,0.00,0.00}{##1}}}
\expandafter\def\csname PY@tok@gi\endcsname{\def\PY@tc##1{\textcolor[rgb]{0.00,0.63,0.00}{##1}}}
\expandafter\def\csname PY@tok@gr\endcsname{\def\PY@tc##1{\textcolor[rgb]{1.00,0.00,0.00}{##1}}}
\expandafter\def\csname PY@tok@ge\endcsname{\let\PY@it=\textit}
\expandafter\def\csname PY@tok@gs\endcsname{\let\PY@bf=\textbf}
\expandafter\def\csname PY@tok@gp\endcsname{\let\PY@bf=\textbf\def\PY@tc##1{\textcolor[rgb]{0.00,0.00,0.50}{##1}}}
\expandafter\def\csname PY@tok@go\endcsname{\def\PY@tc##1{\textcolor[rgb]{0.53,0.53,0.53}{##1}}}
\expandafter\def\csname PY@tok@gt\endcsname{\def\PY@tc##1{\textcolor[rgb]{0.00,0.27,0.87}{##1}}}
\expandafter\def\csname PY@tok@err\endcsname{\def\PY@bc##1{\setlength{\fboxsep}{0pt}\fcolorbox[rgb]{1.00,0.00,0.00}{1,1,1}{\strut ##1}}}
\expandafter\def\csname PY@tok@kc\endcsname{\let\PY@bf=\textbf\def\PY@tc##1{\textcolor[rgb]{0.00,0.50,0.00}{##1}}}
\expandafter\def\csname PY@tok@kd\endcsname{\let\PY@bf=\textbf\def\PY@tc##1{\textcolor[rgb]{0.00,0.50,0.00}{##1}}}
\expandafter\def\csname PY@tok@kn\endcsname{\let\PY@bf=\textbf\def\PY@tc##1{\textcolor[rgb]{0.00,0.50,0.00}{##1}}}
\expandafter\def\csname PY@tok@kr\endcsname{\let\PY@bf=\textbf\def\PY@tc##1{\textcolor[rgb]{0.00,0.50,0.00}{##1}}}
\expandafter\def\csname PY@tok@bp\endcsname{\def\PY@tc##1{\textcolor[rgb]{0.00,0.50,0.00}{##1}}}
\expandafter\def\csname PY@tok@fm\endcsname{\def\PY@tc##1{\textcolor[rgb]{0.00,0.00,1.00}{##1}}}
\expandafter\def\csname PY@tok@vc\endcsname{\def\PY@tc##1{\textcolor[rgb]{0.10,0.09,0.49}{##1}}}
\expandafter\def\csname PY@tok@vg\endcsname{\def\PY@tc##1{\textcolor[rgb]{0.10,0.09,0.49}{##1}}}
\expandafter\def\csname PY@tok@vi\endcsname{\def\PY@tc##1{\textcolor[rgb]{0.10,0.09,0.49}{##1}}}
\expandafter\def\csname PY@tok@vm\endcsname{\def\PY@tc##1{\textcolor[rgb]{0.10,0.09,0.49}{##1}}}
\expandafter\def\csname PY@tok@sa\endcsname{\def\PY@tc##1{\textcolor[rgb]{0.73,0.13,0.13}{##1}}}
\expandafter\def\csname PY@tok@sb\endcsname{\def\PY@tc##1{\textcolor[rgb]{0.73,0.13,0.13}{##1}}}
\expandafter\def\csname PY@tok@sc\endcsname{\def\PY@tc##1{\textcolor[rgb]{0.73,0.13,0.13}{##1}}}
\expandafter\def\csname PY@tok@dl\endcsname{\def\PY@tc##1{\textcolor[rgb]{0.73,0.13,0.13}{##1}}}
\expandafter\def\csname PY@tok@s2\endcsname{\def\PY@tc##1{\textcolor[rgb]{0.73,0.13,0.13}{##1}}}
\expandafter\def\csname PY@tok@sh\endcsname{\def\PY@tc##1{\textcolor[rgb]{0.73,0.13,0.13}{##1}}}
\expandafter\def\csname PY@tok@s1\endcsname{\def\PY@tc##1{\textcolor[rgb]{0.73,0.13,0.13}{##1}}}
\expandafter\def\csname PY@tok@mb\endcsname{\def\PY@tc##1{\textcolor[rgb]{0.40,0.40,0.40}{##1}}}
\expandafter\def\csname PY@tok@mf\endcsname{\def\PY@tc##1{\textcolor[rgb]{0.40,0.40,0.40}{##1}}}
\expandafter\def\csname PY@tok@mh\endcsname{\def\PY@tc##1{\textcolor[rgb]{0.40,0.40,0.40}{##1}}}
\expandafter\def\csname PY@tok@mi\endcsname{\def\PY@tc##1{\textcolor[rgb]{0.40,0.40,0.40}{##1}}}
\expandafter\def\csname PY@tok@il\endcsname{\def\PY@tc##1{\textcolor[rgb]{0.40,0.40,0.40}{##1}}}
\expandafter\def\csname PY@tok@mo\endcsname{\def\PY@tc##1{\textcolor[rgb]{0.40,0.40,0.40}{##1}}}
\expandafter\def\csname PY@tok@ch\endcsname{\let\PY@it=\textit\def\PY@tc##1{\textcolor[rgb]{0.25,0.50,0.50}{##1}}}
\expandafter\def\csname PY@tok@cm\endcsname{\let\PY@it=\textit\def\PY@tc##1{\textcolor[rgb]{0.25,0.50,0.50}{##1}}}
\expandafter\def\csname PY@tok@cpf\endcsname{\let\PY@it=\textit\def\PY@tc##1{\textcolor[rgb]{0.25,0.50,0.50}{##1}}}
\expandafter\def\csname PY@tok@c1\endcsname{\let\PY@it=\textit\def\PY@tc##1{\textcolor[rgb]{0.25,0.50,0.50}{##1}}}
\expandafter\def\csname PY@tok@cs\endcsname{\let\PY@it=\textit\def\PY@tc##1{\textcolor[rgb]{0.25,0.50,0.50}{##1}}}

\def\PYZbs{\char`\\}
\def\PYZus{\char`\_}
\def\PYZob{\char`\{}
\def\PYZcb{\char`\}}
\def\PYZca{\char`\^}
\def\PYZam{\char`\&}
\def\PYZlt{\char`\<}
\def\PYZgt{\char`\>}
\def\PYZsh{\char`\#}
\def\PYZpc{\char`\%}
\def\PYZdl{\char`\$}
\def\PYZhy{\char`\-}
\def\PYZsq{\char`\'}
\def\PYZdq{\char`\"}
\def\PYZti{\char`\~}
% for compatibility with earlier versions
\def\PYZat{@}
\def\PYZlb{[}
\def\PYZrb{]}
\makeatother


    % Exact colors from NB
    \definecolor{incolor}{rgb}{0.0, 0.0, 0.5}
    \definecolor{outcolor}{rgb}{0.545, 0.0, 0.0}



    
    % Prevent overflowing lines due to hard-to-break entities
    \sloppy 
    % Setup hyperref package
    \hypersetup{
      breaklinks=true,  % so long urls are correctly broken across lines
      colorlinks=true,
      urlcolor=urlcolor,
      linkcolor=linkcolor,
      citecolor=citecolor,
      }
    % Slightly bigger margins than the latex defaults
    
    \geometry{verbose,tmargin=1in,bmargin=1in,lmargin=1in,rmargin=1in}
    
    

    \begin{document}
    
    
    \maketitle
    
    

    
    \section{Taller 5}\label{taller-5}

Métodos Computacionales para Políticas Públicas - URosario

\textbf{Entrega: viernes 6-sep-2019 11:59 PM}

    \textbf{Julián Santiago Ramírez} julians.ramirez@urosario.edu.co

    \subsection{Instrucciones:}\label{instrucciones}

\begin{itemize}
\tightlist
\item
  Guarde una copia de este \emph{Jupyter Notebook} en su computador,
  idealmente en una carpeta destinada al material del curso.
\item
  Modifique el nombre del archivo del \emph{notebook}, agregando al
  final un guión inferior y su nombre y apellido, separados estos
  últimos por otro guión inferior. Por ejemplo, mi \emph{notebook} se
  llamaría: mcpp\_taller5\_santiago\_matallana
\item
  Marque el \emph{notebook} con su nombre y e-mail en el bloque verde
  arriba. Reemplace el texto "{[}Su nombre acá{]}" con su nombre y
  apellido. Similar para su e-mail.
\item
  Desarrolle la totalidad del taller sobre este \emph{notebook},
  insertando las celdas que sea necesario debajo de cada pregunta. Haga
  buen uso de las celdas para código y de las celdas tipo
  \emph{markdown} según el caso.
\item
  Recuerde salvar periódicamente sus avances.
\item
  Cuando termine el taller:

  \begin{enumerate}
  \def\labelenumi{\arabic{enumi}.}
  \tightlist
  \item
    Descárguelo en PDF. Si tiene algún problema con la conversión,
    descárguelo en HTML.
  \item
    Suba los dos archivos (.pdf -o .html- y .ipynb) a su repositorio en
    GitHub antes de la fecha y hora límites.
  \end{enumerate}
\end{itemize}

(Todos los ejercicios tienen el mismo valor.)

    \begin{center}\rule{0.5\linewidth}{\linethickness}\end{center}

    \subsubsection{1}\label{section}

Escríba una función que ordene (de forma ascedente y descendente) un
diccionario según sus valores.

    \begin{Verbatim}[commandchars=\\\{\}]
{\color{incolor}In [{\color{incolor}115}]:} \PY{k}{def} \PY{n+nf}{ordenar}\PY{p}{(}\PY{n}{dicc}\PY{p}{,}\PY{n}{forma}\PY{p}{)}\PY{p}{:}
              \PY{k}{if} \PY{n}{forma}\PY{o}{==}\PY{l+s+s2}{\PYZdq{}}\PY{l+s+s2}{ascendente}\PY{l+s+s2}{\PYZdq{}}\PY{p}{:}
                  \PY{c+c1}{\PYZsh{}\PYZsh{} Creamos una lista con los items del diccionario}
                  \PY{n}{b}\PY{o}{=}\PY{n+nb}{list}\PY{p}{(}\PY{n}{dicc}\PY{o}{.}\PY{n}{items}\PY{p}{(}\PY{p}{)}\PY{p}{)}
                  \PY{c+c1}{\PYZsh{}\PYZsh{} Dado que b es una lista podemos usar el metodo \PYZdq{}sort\PYZdq{}}
                  \PY{c+c1}{\PYZsh{}\PYZsh{} Este metodo requiere que se le diga que se quiere organizar en la lista, eso se logra con la palabra \PYZdq{}key\PYZdq{}}
                  \PY{c+c1}{\PYZsh{}\PYZsh{} Luego seleccionamos solo los valores y esto lo logramos con una funcion lambda que escoge el valor }
                  \PY{n}{b}\PY{o}{.}\PY{n}{sort}\PY{p}{(}\PY{n}{key}\PY{o}{=}\PY{k}{lambda} \PY{n}{x}\PY{p}{:}\PY{n}{x}\PY{p}{[}\PY{l+m+mi}{1}\PY{p}{]}\PY{p}{)}
                  \PY{n}{diccionario\PYZus{}nuevo}\PY{o}{=}\PY{p}{\PYZob{}}\PY{p}{\PYZcb{}}
                  \PY{k}{for} \PY{n}{i} \PY{o+ow}{in} \PY{n}{b}\PY{p}{:}
                      \PY{n}{diccionario\PYZus{}nuevo}\PY{p}{[}\PY{n}{i}\PY{p}{[}\PY{l+m+mi}{0}\PY{p}{]}\PY{p}{]}\PY{o}{=}\PY{n}{i}\PY{p}{[}\PY{l+m+mi}{1}\PY{p}{]}
                  \PY{k}{return} \PY{n}{diccionario\PYZus{}nuevo}
              \PY{k}{elif} \PY{n}{forma}\PY{o}{==}\PY{l+s+s2}{\PYZdq{}}\PY{l+s+s2}{descendente}\PY{l+s+s2}{\PYZdq{}}\PY{p}{:}
                  \PY{n}{b}\PY{o}{=}\PY{n+nb}{list}\PY{p}{(}\PY{n}{dicc}\PY{o}{.}\PY{n}{items}\PY{p}{(}\PY{p}{)}\PY{p}{)}
                  \PY{c+c1}{\PYZsh{}\PYZsh{} Dado que b es una lista podemos usar el metodo \PYZdq{}sort\PYZdq{}}
                  \PY{c+c1}{\PYZsh{}\PYZsh{} Este metodo requiere que se le diga que se quiere organizar en la lista, eso se logra con la palabra \PYZdq{}key\PYZdq{}}
                  \PY{c+c1}{\PYZsh{}\PYZsh{} Luego seleccionamos solo los valores y esto lo logramos con una funcion lambda que escoge el valor }
                  \PY{n}{b}\PY{o}{.}\PY{n}{sort}\PY{p}{(}\PY{n}{key}\PY{o}{=}\PY{k}{lambda} \PY{n}{x}\PY{p}{:}\PY{n}{x}\PY{p}{[}\PY{l+m+mi}{1}\PY{p}{]}\PY{p}{,}\PY{n}{reverse}\PY{o}{=}\PY{k+kc}{True}\PY{p}{)}
                  \PY{n}{diccionario\PYZus{}nuevo}\PY{o}{=}\PY{p}{\PYZob{}}\PY{p}{\PYZcb{}}
                  \PY{k}{for} \PY{n}{i} \PY{o+ow}{in} \PY{n}{b}\PY{p}{:}
                      \PY{n}{diccionario\PYZus{}nuevo}\PY{p}{[}\PY{n}{i}\PY{p}{[}\PY{l+m+mi}{0}\PY{p}{]}\PY{p}{]}\PY{o}{=}\PY{n}{i}\PY{p}{[}\PY{l+m+mi}{1}\PY{p}{]}
                  \PY{k}{return} \PY{n}{diccionario\PYZus{}nuevo}
          
              
          \PY{n}{dicc}\PY{o}{=}\PY{p}{\PYZob{}}\PY{l+s+s2}{\PYZdq{}}\PY{l+s+s2}{a}\PY{l+s+s2}{\PYZdq{}}\PY{p}{:}\PY{l+m+mi}{4}\PY{p}{,}\PY{l+s+s2}{\PYZdq{}}\PY{l+s+s2}{b}\PY{l+s+s2}{\PYZdq{}}\PY{p}{:}\PY{l+m+mi}{3}\PY{p}{,}\PY{l+s+s2}{\PYZdq{}}\PY{l+s+s2}{c}\PY{l+s+s2}{\PYZdq{}}\PY{p}{:}\PY{l+m+mi}{9}\PY{p}{,}\PY{l+s+s2}{\PYZdq{}}\PY{l+s+s2}{d}\PY{l+s+s2}{\PYZdq{}}\PY{p}{:}\PY{l+m+mi}{5}\PY{p}{,}\PY{l+s+s2}{\PYZdq{}}\PY{l+s+s2}{e}\PY{l+s+s2}{\PYZdq{}}\PY{p}{:}\PY{l+m+mi}{8}\PY{p}{,}\PY{l+s+s2}{\PYZdq{}}\PY{l+s+s2}{f}\PY{l+s+s2}{\PYZdq{}}\PY{p}{:}\PY{l+m+mi}{1}\PY{p}{\PYZcb{}}   
          \PY{n}{nuevo}\PY{o}{=}\PY{n}{ordenar}\PY{p}{(}\PY{n}{dicc}\PY{p}{,}\PY{l+s+s2}{\PYZdq{}}\PY{l+s+s2}{ascendente}\PY{l+s+s2}{\PYZdq{}}\PY{p}{)}
          \PY{n+nb}{print} \PY{p}{(}\PY{l+s+s2}{\PYZdq{}}\PY{l+s+s2}{diccionario principal: }\PY{l+s+s2}{\PYZdq{}}\PY{p}{,} \PY{n}{dicc}\PY{p}{)}
          \PY{n+nb}{print} \PY{p}{(}\PY{l+s+s2}{\PYZdq{}}\PY{l+s+s2}{diccionario organizado: }\PY{l+s+s2}{\PYZdq{}}\PY{p}{,} \PY{n}{nuevo}\PY{p}{)}
          \PY{n}{nuevo}\PY{o}{=}\PY{n}{ordenar}\PY{p}{(}\PY{n}{dicc}\PY{p}{,}\PY{l+s+s2}{\PYZdq{}}\PY{l+s+s2}{descendente}\PY{l+s+s2}{\PYZdq{}}\PY{p}{)}
          \PY{n+nb}{print} \PY{p}{(}\PY{l+s+s2}{\PYZdq{}}\PY{l+s+s2}{diccionario principal: }\PY{l+s+s2}{\PYZdq{}}\PY{p}{,} \PY{n}{dicc}\PY{p}{)}
          \PY{n+nb}{print} \PY{p}{(}\PY{l+s+s2}{\PYZdq{}}\PY{l+s+s2}{diccionario organizado: }\PY{l+s+s2}{\PYZdq{}}\PY{p}{,} \PY{n}{nuevo}\PY{p}{)}
\end{Verbatim}


    \begin{Verbatim}[commandchars=\\\{\}]
diccionario principal:  \{'a': 4, 'b': 3, 'c': 9, 'd': 5, 'e': 8, 'f': 1\}
diccionario organizado:  \{'f': 1, 'b': 3, 'a': 4, 'd': 5, 'e': 8, 'c': 9\}
diccionario principal:  \{'a': 4, 'b': 3, 'c': 9, 'd': 5, 'e': 8, 'f': 1\}
diccionario organizado:  \{'c': 9, 'e': 8, 'd': 5, 'a': 4, 'b': 3, 'f': 1\}

    \end{Verbatim}

    \begin{Verbatim}[commandchars=\\\{\}]
{\color{incolor}In [{\color{incolor}108}]:} \PY{n}{lista}\PY{o}{=}\PY{p}{[}\PY{l+m+mi}{1}\PY{p}{,}\PY{l+m+mi}{2}\PY{p}{,}\PY{l+m+mi}{3}\PY{p}{,}\PY{l+m+mi}{4}\PY{p}{,}\PY{l+m+mi}{5}\PY{p}{,}\PY{l+m+mi}{6}\PY{p}{]}
          \PY{n}{lista}\PY{o}{.}\PY{n}{reverse}\PY{p}{(}\PY{p}{)}
\end{Verbatim}


    \subsubsection{2}\label{section}

Escriba una función que agregue una llave a un diccionario.

    \begin{Verbatim}[commandchars=\\\{\}]
{\color{incolor}In [{\color{incolor}46}]:} \PY{c+c1}{\PYZsh{}\PYZsh{}\PYZsh{} El programa no especifica que valor debo agregar con la llave por ende la agrego con el valor 0}
         
         \PY{k}{def} \PY{n+nf}{agregar}\PY{p}{(}\PY{n}{dicc}\PY{p}{,}\PY{n}{llave}\PY{p}{)}\PY{p}{:}
             \PY{n}{dicc}\PY{p}{[}\PY{n}{llave}\PY{p}{]}\PY{o}{=}\PY{l+m+mi}{0}
             \PY{k}{return} \PY{n}{dicc}
         
         \PY{n}{dicc}\PY{o}{=}\PY{p}{\PYZob{}}\PY{l+s+s2}{\PYZdq{}}\PY{l+s+s2}{a}\PY{l+s+s2}{\PYZdq{}}\PY{p}{:}\PY{l+m+mi}{4}\PY{p}{,}\PY{l+s+s2}{\PYZdq{}}\PY{l+s+s2}{b}\PY{l+s+s2}{\PYZdq{}}\PY{p}{:}\PY{l+m+mi}{3}\PY{p}{,}\PY{l+s+s2}{\PYZdq{}}\PY{l+s+s2}{c}\PY{l+s+s2}{\PYZdq{}}\PY{p}{:}\PY{l+m+mi}{9}\PY{p}{,}\PY{l+s+s2}{\PYZdq{}}\PY{l+s+s2}{d}\PY{l+s+s2}{\PYZdq{}}\PY{p}{:}\PY{l+m+mi}{5}\PY{p}{,}\PY{l+s+s2}{\PYZdq{}}\PY{l+s+s2}{e}\PY{l+s+s2}{\PYZdq{}}\PY{p}{:}\PY{l+m+mi}{8}\PY{p}{,}\PY{l+s+s2}{\PYZdq{}}\PY{l+s+s2}{f}\PY{l+s+s2}{\PYZdq{}}\PY{p}{:}\PY{l+m+mi}{1}\PY{p}{\PYZcb{}}
         \PY{n}{dicc}\PY{o}{=}\PY{n}{agregar}\PY{p}{(}\PY{n}{dicc}\PY{p}{,}\PY{l+s+s2}{\PYZdq{}}\PY{l+s+s2}{y}\PY{l+s+s2}{\PYZdq{}}\PY{p}{)}
         \PY{n+nb}{print} \PY{p}{(}\PY{l+s+s2}{\PYZdq{}}\PY{l+s+s2}{diccionario principal: }\PY{l+s+s2}{\PYZdq{}}\PY{p}{,} \PY{n}{dicc}\PY{p}{)}
\end{Verbatim}


    \begin{Verbatim}[commandchars=\\\{\}]
diccionario principal:  \{'a': 4, 'b': 3, 'c': 9, 'd': 5, 'e': 8, 'f': 1, 'y': 0\}

    \end{Verbatim}

    \subsubsection{3}\label{section}

Escriba un programa que concatene los siguientes tres diccionarios en
uno nuevo:

    dicc1 = \{1:10, 2:20\} dicc2 = \{3:30, 4:40\} dicc3 = \{5:50,6:60\}
Resultado esperado: \{1: 10, 2: 20, 3: 30, 4: 40, 5: 50, 6: 60\}

    \begin{Verbatim}[commandchars=\\\{\}]
{\color{incolor}In [{\color{incolor}53}]:} \PY{n}{dicc1} \PY{o}{=} \PY{p}{\PYZob{}}\PY{l+m+mi}{1}\PY{p}{:}\PY{l+m+mi}{10}\PY{p}{,} \PY{l+m+mi}{2}\PY{p}{:}\PY{l+m+mi}{20}\PY{p}{\PYZcb{}}
         \PY{n}{dicc2} \PY{o}{=} \PY{p}{\PYZob{}}\PY{l+m+mi}{3}\PY{p}{:}\PY{l+m+mi}{30}\PY{p}{,} \PY{l+m+mi}{4}\PY{p}{:}\PY{l+m+mi}{40}\PY{p}{\PYZcb{}}
         \PY{n}{dicc3} \PY{o}{=} \PY{p}{\PYZob{}}\PY{l+m+mi}{5}\PY{p}{:}\PY{l+m+mi}{50}\PY{p}{,}\PY{l+m+mi}{6}\PY{p}{:}\PY{l+m+mi}{60}\PY{p}{\PYZcb{}}
         
         \PY{n}{d1}\PY{o}{=}\PY{n+nb}{list}\PY{p}{(}\PY{n}{dicc1}\PY{o}{.}\PY{n}{items}\PY{p}{(}\PY{p}{)}\PY{p}{)}
         \PY{n}{d2}\PY{o}{=}\PY{n+nb}{list}\PY{p}{(}\PY{n}{dicc2}\PY{o}{.}\PY{n}{items}\PY{p}{(}\PY{p}{)}\PY{p}{)}
         \PY{n}{d3}\PY{o}{=}\PY{n+nb}{list}\PY{p}{(}\PY{n}{dicc3}\PY{o}{.}\PY{n}{items}\PY{p}{(}\PY{p}{)}\PY{p}{)}
         
         \PY{n}{total}\PY{o}{=}\PY{n}{d1}\PY{o}{+}\PY{n}{d2}\PY{o}{+}\PY{n}{d3}
         
         \PY{n}{diccionario\PYZus{}nuevo}\PY{o}{=}\PY{p}{\PYZob{}}\PY{p}{\PYZcb{}}
         \PY{k}{for} \PY{n}{i} \PY{o+ow}{in} \PY{n}{total}\PY{p}{:}
             \PY{n}{diccionario\PYZus{}nuevo}\PY{p}{[}\PY{n}{i}\PY{p}{[}\PY{l+m+mi}{0}\PY{p}{]}\PY{p}{]}\PY{o}{=}\PY{n}{i}\PY{p}{[}\PY{l+m+mi}{1}\PY{p}{]}
             
         \PY{n+nb}{print}\PY{p}{(}\PY{l+s+s2}{\PYZdq{}}\PY{l+s+s2}{diccionario concatenado: }\PY{l+s+s2}{\PYZdq{}}\PY{p}{,}\PY{n}{diccionario\PYZus{}nuevo}\PY{p}{)}
\end{Verbatim}


    \begin{Verbatim}[commandchars=\\\{\}]
diccionario concatenado:  \{1: 10, 2: 20, 3: 30, 4: 40, 5: 50, 6: 60\}

    \end{Verbatim}

    \subsubsection{4}\label{section}

Escriba una función que verifique si una determinada llave existe o no
en un diccionario.

    \begin{Verbatim}[commandchars=\\\{\}]
{\color{incolor}In [{\color{incolor}63}]:} \PY{k}{def} \PY{n+nf}{existencia}\PY{p}{(}\PY{n}{dicc}\PY{p}{,}\PY{n}{llave}\PY{p}{)}\PY{p}{:}
             \PY{n}{lista}\PY{o}{=}\PY{n+nb}{list}\PY{p}{(}\PY{n}{dicc}\PY{o}{.}\PY{n}{items}\PY{p}{(}\PY{p}{)}\PY{p}{)}
             \PY{n}{booll}\PY{o}{=}\PY{k+kc}{False}
             \PY{k}{for} \PY{n}{i} \PY{o+ow}{in} \PY{n}{lista}\PY{p}{:}
                 \PY{k}{if} \PY{n}{i}\PY{p}{[}\PY{l+m+mi}{0}\PY{p}{]}\PY{o}{==}\PY{n}{llave}\PY{p}{:}
                     \PY{n}{booll}\PY{o}{=}\PY{k+kc}{True}
                     \PY{n+nb}{print}\PY{p}{(}\PY{l+s+s2}{\PYZdq{}}\PY{l+s+s2}{la llave }\PY{l+s+s2}{\PYZdq{}}\PY{o}{+}\PY{l+s+s2}{\PYZdq{}}\PY{l+s+s2}{\PYZsq{}}\PY{l+s+s2}{\PYZsq{}}\PY{l+s+s2}{\PYZdq{}}\PY{o}{+}\PY{n+nb}{str}\PY{p}{(}\PY{n}{llave}\PY{p}{)}\PY{o}{+}\PY{l+s+s2}{\PYZdq{}}\PY{l+s+s2}{\PYZsq{}}\PY{l+s+s2}{\PYZsq{}}\PY{l+s+s2}{\PYZdq{}}\PY{o}{+}\PY{l+s+s2}{\PYZdq{}}\PY{l+s+s2}{ esta en el diccionario}\PY{l+s+s2}{\PYZdq{}}\PY{p}{)}
                     \PY{k}{break}
             \PY{k}{if} \PY{n}{booll}\PY{o}{==}\PY{k+kc}{False}\PY{p}{:}
                 \PY{n+nb}{print}\PY{p}{(}\PY{l+s+s2}{\PYZdq{}}\PY{l+s+s2}{\PYZsq{}}\PY{l+s+s2}{\PYZsq{}}\PY{l+s+s2}{\PYZdq{}}\PY{o}{+}\PY{n+nb}{str}\PY{p}{(}\PY{n}{llave}\PY{p}{)}\PY{o}{+}\PY{l+s+s2}{\PYZdq{}}\PY{l+s+s2}{\PYZsq{}}\PY{l+s+s2}{\PYZsq{}}\PY{l+s+s2}{\PYZdq{}}\PY{o}{+}\PY{l+s+s2}{\PYZdq{}}\PY{l+s+s2}{No esta en el diccionario}\PY{l+s+s2}{\PYZdq{}}\PY{p}{)}
                 
         \PY{n}{dicc}\PY{o}{=}\PY{p}{\PYZob{}}\PY{l+s+s2}{\PYZdq{}}\PY{l+s+s2}{a}\PY{l+s+s2}{\PYZdq{}}\PY{p}{:}\PY{l+m+mi}{4}\PY{p}{,}\PY{l+s+s2}{\PYZdq{}}\PY{l+s+s2}{b}\PY{l+s+s2}{\PYZdq{}}\PY{p}{:}\PY{l+m+mi}{3}\PY{p}{,}\PY{l+s+s2}{\PYZdq{}}\PY{l+s+s2}{c}\PY{l+s+s2}{\PYZdq{}}\PY{p}{:}\PY{l+m+mi}{9}\PY{p}{,}\PY{l+s+s2}{\PYZdq{}}\PY{l+s+s2}{d}\PY{l+s+s2}{\PYZdq{}}\PY{p}{:}\PY{l+m+mi}{5}\PY{p}{,}\PY{l+s+s2}{\PYZdq{}}\PY{l+s+s2}{e}\PY{l+s+s2}{\PYZdq{}}\PY{p}{:}\PY{l+m+mi}{8}\PY{p}{,}\PY{l+s+s2}{\PYZdq{}}\PY{l+s+s2}{f}\PY{l+s+s2}{\PYZdq{}}\PY{p}{:}\PY{l+m+mi}{1}\PY{p}{\PYZcb{}}
         \PY{n}{existencia}\PY{p}{(}\PY{n}{dicc}\PY{p}{,}\PY{l+s+s2}{\PYZdq{}}\PY{l+s+s2}{a}\PY{l+s+s2}{\PYZdq{}}\PY{p}{)}
         \PY{n}{existencia}\PY{p}{(}\PY{n}{dicc}\PY{p}{,}\PY{l+s+s2}{\PYZdq{}}\PY{l+s+s2}{k}\PY{l+s+s2}{\PYZdq{}}\PY{p}{)}
\end{Verbatim}


    \begin{Verbatim}[commandchars=\\\{\}]
la llave ''a'' esta en el diccionario
''k''No esta en el diccionario

    \end{Verbatim}

    \subsubsection{5}\label{section}

Escriba una función que imprima todos los pares (llave, valor) de un
diccionario.

    \begin{Verbatim}[commandchars=\\\{\}]
{\color{incolor}In [{\color{incolor}65}]:} \PY{k}{def} \PY{n+nf}{imprimir} \PY{p}{(}\PY{n}{dicc}\PY{p}{)}\PY{p}{:}
             \PY{k}{for} \PY{n}{k}\PY{p}{,}\PY{n}{v} \PY{o+ow}{in} \PY{n}{dicc}\PY{o}{.}\PY{n}{items}\PY{p}{(}\PY{p}{)}\PY{p}{:}
                 \PY{n}{tup}\PY{o}{=}\PY{p}{(}\PY{n}{k}\PY{p}{,}\PY{n}{v}\PY{p}{)}
                 \PY{n+nb}{print}\PY{p}{(}\PY{n}{tup}\PY{p}{)}
         \PY{n}{dicc}\PY{o}{=}\PY{p}{\PYZob{}}\PY{l+s+s2}{\PYZdq{}}\PY{l+s+s2}{a}\PY{l+s+s2}{\PYZdq{}}\PY{p}{:}\PY{l+m+mi}{4}\PY{p}{,}\PY{l+s+s2}{\PYZdq{}}\PY{l+s+s2}{b}\PY{l+s+s2}{\PYZdq{}}\PY{p}{:}\PY{l+m+mi}{3}\PY{p}{,}\PY{l+s+s2}{\PYZdq{}}\PY{l+s+s2}{c}\PY{l+s+s2}{\PYZdq{}}\PY{p}{:}\PY{l+m+mi}{9}\PY{p}{,}\PY{l+s+s2}{\PYZdq{}}\PY{l+s+s2}{d}\PY{l+s+s2}{\PYZdq{}}\PY{p}{:}\PY{l+m+mi}{5}\PY{p}{,}\PY{l+s+s2}{\PYZdq{}}\PY{l+s+s2}{e}\PY{l+s+s2}{\PYZdq{}}\PY{p}{:}\PY{l+m+mi}{8}\PY{p}{,}\PY{l+s+s2}{\PYZdq{}}\PY{l+s+s2}{f}\PY{l+s+s2}{\PYZdq{}}\PY{p}{:}\PY{l+m+mi}{1}\PY{p}{\PYZcb{}}
         \PY{n}{imprimir}\PY{p}{(}\PY{n}{dicc}\PY{p}{)}
\end{Verbatim}


    \begin{Verbatim}[commandchars=\\\{\}]
('a', 4)
('b', 3)
('c', 9)
('d', 5)
('e', 8)
('f', 1)

    \end{Verbatim}

    \subsubsection{6}\label{section}

Escriba una función que genere un diccionario con los números enteros
entre 1 y n en la forma (x: x**2).

    \begin{Verbatim}[commandchars=\\\{\}]
{\color{incolor}In [{\color{incolor}71}]:} \PY{k}{def} \PY{n+nf}{generar}\PY{p}{(}\PY{n}{n}\PY{p}{)}\PY{p}{:}
             \PY{n}{diccionario}\PY{o}{=}\PY{p}{\PYZob{}}\PY{p}{\PYZcb{}}
             \PY{k}{for} \PY{n}{i} \PY{o+ow}{in} \PY{n+nb}{range}\PY{p}{(}\PY{l+m+mi}{1}\PY{p}{,}\PY{n}{n}\PY{o}{+}\PY{l+m+mi}{1}\PY{p}{)}\PY{p}{:}
                 \PY{n}{valor}\PY{o}{=}\PY{n}{i}\PY{o}{*}\PY{o}{*}\PY{l+m+mi}{2}
                 \PY{n}{diccionario}\PY{p}{[}\PY{n}{i}\PY{p}{]}\PY{o}{=}\PY{n}{valor}
             \PY{k}{return} \PY{n}{diccionario}
         
         \PY{n}{d}\PY{o}{=}\PY{n}{generar}\PY{p}{(}\PY{l+m+mi}{10}\PY{p}{)}
         \PY{n+nb}{print}\PY{p}{(}\PY{l+s+s2}{\PYZdq{}}\PY{l+s+s2}{diccionario generado: }\PY{l+s+s2}{\PYZdq{}}\PY{p}{,}\PY{n}{d}\PY{p}{)}
\end{Verbatim}


    \begin{Verbatim}[commandchars=\\\{\}]
diccionario generado:  \{1: 1, 2: 4, 3: 9, 4: 16, 5: 25, 6: 36, 7: 49, 8: 64, 9: 81, 10: 100\}

    \end{Verbatim}

    \subsubsection{7}\label{section}

Escriba una función que sume todas las llaves de un diccionario. (Asuma
que son números.)

    \begin{Verbatim}[commandchars=\\\{\}]
{\color{incolor}In [{\color{incolor}75}]:} \PY{k}{def} \PY{n+nf}{suma}\PY{p}{(}\PY{n}{dicc}\PY{p}{)}\PY{p}{:}
             \PY{n}{lla}\PY{o}{=}\PY{n+nb}{list}\PY{p}{(}\PY{n}{dicc}\PY{o}{.}\PY{n}{keys}\PY{p}{(}\PY{p}{)}\PY{p}{)}
             \PY{n}{x}\PY{o}{=}\PY{n+nb}{sum}\PY{p}{(}\PY{n}{lla}\PY{p}{)}
             \PY{n+nb}{print}\PY{p}{(}\PY{l+s+s2}{\PYZdq{}}\PY{l+s+s2}{La suma de las llaves del dicccionario es: }\PY{l+s+s2}{\PYZdq{}}\PY{p}{,}\PY{n}{x}\PY{p}{)}
         
         \PY{n}{d}\PY{o}{=}\PY{p}{\PYZob{}}\PY{l+m+mi}{10}\PY{p}{:} \PY{l+m+mi}{10}\PY{p}{,} \PY{l+m+mi}{2}\PY{p}{:} \PY{l+m+mi}{20}\PY{p}{,} \PY{l+m+mi}{3}\PY{p}{:} \PY{l+m+mi}{30}\PY{p}{,} \PY{l+m+mi}{4}\PY{p}{:} \PY{l+m+mi}{40}\PY{p}{,} \PY{l+m+mi}{5}\PY{p}{:} \PY{l+m+mi}{50}\PY{p}{,} \PY{l+m+mi}{6}\PY{p}{:} \PY{l+m+mi}{60}\PY{p}{\PYZcb{}}
         \PY{n}{suma}\PY{p}{(}\PY{n}{d}\PY{p}{)}
\end{Verbatim}


    \begin{Verbatim}[commandchars=\\\{\}]
La suma de las llaves del dicccionario es:  30

    \end{Verbatim}

    \subsubsection{8}\label{section}

Escriba una función que sume todos los valores de un diccionario. (Asuma
que son números.)

    \begin{Verbatim}[commandchars=\\\{\}]
{\color{incolor}In [{\color{incolor}76}]:} \PY{k}{def} \PY{n+nf}{suma\PYZus{}valores}\PY{p}{(}\PY{n}{dicc}\PY{p}{)}\PY{p}{:}
             \PY{n}{val}\PY{o}{=}\PY{n+nb}{list}\PY{p}{(}\PY{n}{dicc}\PY{o}{.}\PY{n}{values}\PY{p}{(}\PY{p}{)}\PY{p}{)}
             \PY{n}{x}\PY{o}{=}\PY{n+nb}{sum}\PY{p}{(}\PY{n}{val}\PY{p}{)}
             \PY{n+nb}{print}\PY{p}{(}\PY{l+s+s2}{\PYZdq{}}\PY{l+s+s2}{La suma de los valores del dicccionario es: }\PY{l+s+s2}{\PYZdq{}}\PY{p}{,}\PY{n}{val}\PY{p}{)}
         
         \PY{n}{d}\PY{o}{=}\PY{p}{\PYZob{}}\PY{l+m+mi}{10}\PY{p}{:} \PY{l+m+mi}{10}\PY{p}{,} \PY{l+m+mi}{2}\PY{p}{:} \PY{l+m+mi}{20}\PY{p}{,} \PY{l+m+mi}{3}\PY{p}{:} \PY{l+m+mi}{30}\PY{p}{,} \PY{l+m+mi}{4}\PY{p}{:} \PY{l+m+mi}{40}\PY{p}{,} \PY{l+m+mi}{5}\PY{p}{:} \PY{l+m+mi}{50}\PY{p}{,} \PY{l+m+mi}{6}\PY{p}{:} \PY{l+m+mi}{60}\PY{p}{\PYZcb{}}
         \PY{n}{suma\PYZus{}valores}\PY{p}{(}\PY{n}{d}\PY{p}{)}
\end{Verbatim}


    \begin{Verbatim}[commandchars=\\\{\}]
La suma de los valores del dicccionario es:  [10, 20, 30, 40, 50, 60]

    \end{Verbatim}

    \subsubsection{9}\label{section}

Escriba una función que sume todos los ítems de un diccionario. (Asuma
que son números.)

    \begin{Verbatim}[commandchars=\\\{\}]
{\color{incolor}In [{\color{incolor}82}]:} \PY{k}{def} \PY{n+nf}{suma\PYZus{}total}\PY{p}{(}\PY{n}{dicc}\PY{p}{)}\PY{p}{:}
             \PY{n}{lista}\PY{o}{=}\PY{n+nb}{list}\PY{p}{(}\PY{n}{dicc}\PY{o}{.}\PY{n}{items}\PY{p}{(}\PY{p}{)}\PY{p}{)}
             \PY{n}{total}\PY{o}{=}\PY{l+m+mi}{0}
             \PY{k}{for} \PY{n}{i} \PY{o+ow}{in} \PY{n}{lista}\PY{p}{:}
                 \PY{n}{s}\PY{o}{=}\PY{n+nb}{sum}\PY{p}{(}\PY{n}{i}\PY{p}{)}
                 \PY{n}{total}\PY{o}{=}\PY{n}{total}\PY{o}{+}\PY{n}{s}
             \PY{n+nb}{print}\PY{p}{(}\PY{l+s+s2}{\PYZdq{}}\PY{l+s+s2}{Suma de llaves y valores: }\PY{l+s+s2}{\PYZdq{}}\PY{p}{,} \PY{n}{total}\PY{p}{)}
         
         \PY{n}{d}\PY{o}{=}\PY{p}{\PYZob{}}\PY{l+m+mi}{10}\PY{p}{:} \PY{l+m+mi}{10}\PY{p}{,} \PY{l+m+mi}{2}\PY{p}{:} \PY{l+m+mi}{20}\PY{p}{,} \PY{l+m+mi}{3}\PY{p}{:} \PY{l+m+mi}{30}\PY{p}{,} \PY{l+m+mi}{4}\PY{p}{:} \PY{l+m+mi}{40}\PY{p}{,} \PY{l+m+mi}{5}\PY{p}{:} \PY{l+m+mi}{50}\PY{p}{,} \PY{l+m+mi}{6}\PY{p}{:} \PY{l+m+mi}{60}\PY{p}{\PYZcb{}}
         \PY{n}{suma\PYZus{}total}\PY{p}{(}\PY{n}{d}\PY{p}{)}
\end{Verbatim}


    \begin{Verbatim}[commandchars=\\\{\}]
Suma de llaves y valores:  240

    \end{Verbatim}

    \subsubsection{10}\label{section}

Escriba una función que tome dos listas y las mapee a un diccionario por
pares. (El primer elemento de la primera lista es la primera llave del
diccionario, el primer elemento de la segunda lista es el valor de la
primera llave del diccionario, etc.)

    \begin{Verbatim}[commandchars=\\\{\}]
{\color{incolor}In [{\color{incolor}84}]:} \PY{k}{def} \PY{n+nf}{crear\PYZus{}diccionario}\PY{p}{(}\PY{n}{lista1}\PY{p}{,}\PY{n}{lista2}\PY{p}{)}\PY{p}{:}
             \PY{n}{diccionario}\PY{o}{=}\PY{p}{\PYZob{}}\PY{p}{\PYZcb{}}
             \PY{c+c1}{\PYZsh{}\PYZsh{}\PYZsh{} En principio asumire que es necesario que ambas listas tengan el mismo tamaño y en caso de que no muestre un mensaje }
             \PY{c+c1}{\PYZsh{}\PYZsh{}\PYZsh{} ... diciendo que tienen tamaños diferentes.}
             \PY{k}{if} \PY{n+nb}{len}\PY{p}{(}\PY{n}{lista1}\PY{p}{)}\PY{o}{==}\PY{n+nb}{len}\PY{p}{(}\PY{n}{lista2}\PY{p}{)}\PY{p}{:}
                 \PY{k}{for} \PY{n}{i} \PY{o+ow}{in} \PY{n+nb}{range}\PY{p}{(}\PY{l+m+mi}{0}\PY{p}{,}\PY{n+nb}{len}\PY{p}{(}\PY{n}{lista1}\PY{p}{)}\PY{p}{)}\PY{p}{:}
                     \PY{n}{diccionario}\PY{p}{[}\PY{n}{lista1}\PY{p}{[}\PY{n}{i}\PY{p}{]}\PY{p}{]}\PY{o}{=}\PY{n}{lista2}\PY{p}{[}\PY{n}{i}\PY{p}{]}
                 \PY{n+nb}{print}\PY{p}{(}\PY{l+s+s2}{\PYZdq{}}\PY{l+s+s2}{Diccionario resultante: }\PY{l+s+s2}{\PYZdq{}}\PY{p}{,} \PY{n}{diccionario}\PY{p}{)}
             \PY{k}{else}\PY{p}{:}
                 \PY{n+nb}{print}\PY{p}{(}\PY{l+s+s2}{\PYZdq{}}\PY{l+s+s2}{Es necesario que el tamaño de las dos listas sea igual}\PY{l+s+s2}{\PYZdq{}}\PY{p}{)}
                 
                 
                 
         \PY{n}{nombres}\PY{o}{=}\PY{p}{[}\PY{l+s+s2}{\PYZdq{}}\PY{l+s+s2}{julian}\PY{l+s+s2}{\PYZdq{}}\PY{p}{,}\PY{l+s+s2}{\PYZdq{}}\PY{l+s+s2}{pablo}\PY{l+s+s2}{\PYZdq{}}\PY{p}{,}\PY{l+s+s2}{\PYZdq{}}\PY{l+s+s2}{juana}\PY{l+s+s2}{\PYZdq{}}\PY{p}{,}\PY{l+s+s2}{\PYZdq{}}\PY{l+s+s2}{laura}\PY{l+s+s2}{\PYZdq{}}\PY{p}{]}
         \PY{n}{edad}\PY{o}{=}\PY{p}{[}\PY{l+m+mi}{20}\PY{p}{,}\PY{l+m+mi}{10}\PY{p}{,}\PY{l+m+mi}{15}\PY{p}{,}\PY{l+m+mi}{12}\PY{p}{]}
         \PY{n}{crear\PYZus{}diccionario}\PY{p}{(}\PY{n}{nombres}\PY{p}{,}\PY{n}{edad}\PY{p}{)}
\end{Verbatim}


    \begin{Verbatim}[commandchars=\\\{\}]
Diccionario resultante:  \{'julian': 20, 'pablo': 10, 'juana': 15, 'laura': 12\}

    \end{Verbatim}

    \subsubsection{11}\label{section}

Escriba una función que elimine una llave de un diccionario.

    \begin{Verbatim}[commandchars=\\\{\}]
{\color{incolor}In [{\color{incolor}87}]:} \PY{k}{def} \PY{n+nf}{eliminar}\PY{p}{(}\PY{n}{dicc}\PY{p}{,}\PY{n}{llave}\PY{p}{)}\PY{p}{:}
             \PY{n}{llav}\PY{o}{=}\PY{n+nb}{list}\PY{p}{(}\PY{n}{dicc}\PY{o}{.}\PY{n}{keys}\PY{p}{(}\PY{p}{)}\PY{p}{)}
             \PY{n}{u}\PY{o}{=}\PY{k+kc}{False}
             \PY{k}{for} \PY{n}{i} \PY{o+ow}{in} \PY{n}{llav}\PY{p}{:}
                 \PY{k}{if} \PY{n}{i}\PY{o}{==}\PY{n}{llave}\PY{p}{:}
                     \PY{n}{u}\PY{o}{=}\PY{k+kc}{True}
                     \PY{k}{del} \PY{n}{dicc}\PY{p}{[}\PY{n}{llave}\PY{p}{]}
                     \PY{k}{return} \PY{n}{dicc}
                     \PY{k}{break}
             \PY{k}{if} \PY{n}{u}\PY{o}{==}\PY{k+kc}{False}\PY{p}{:}
                 \PY{n+nb}{print}\PY{p}{(}\PY{l+s+s2}{\PYZdq{}}\PY{l+s+s2}{La llave }\PY{l+s+s2}{\PYZdq{}}\PY{o}{+}\PY{n+nb}{str}\PY{p}{(}\PY{n}{llave}\PY{p}{)}\PY{o}{+}\PY{l+s+s2}{\PYZdq{}}\PY{l+s+s2}{ no fue encontrada en el diccionario}\PY{l+s+s2}{\PYZdq{}}\PY{p}{)}
         
         \PY{n}{d}\PY{o}{=}\PY{p}{\PYZob{}}\PY{l+s+s1}{\PYZsq{}}\PY{l+s+s1}{julian}\PY{l+s+s1}{\PYZsq{}}\PY{p}{:} \PY{l+m+mi}{20}\PY{p}{,} \PY{l+s+s1}{\PYZsq{}}\PY{l+s+s1}{pablo}\PY{l+s+s1}{\PYZsq{}}\PY{p}{:} \PY{l+m+mi}{10}\PY{p}{,} \PY{l+s+s1}{\PYZsq{}}\PY{l+s+s1}{juana}\PY{l+s+s1}{\PYZsq{}}\PY{p}{:} \PY{l+m+mi}{15}\PY{p}{,} \PY{l+s+s1}{\PYZsq{}}\PY{l+s+s1}{laura}\PY{l+s+s1}{\PYZsq{}}\PY{p}{:} \PY{l+m+mi}{12}\PY{p}{\PYZcb{}}
         \PY{n}{dic}\PY{o}{=}\PY{n}{eliminar}\PY{p}{(}\PY{n}{d}\PY{p}{,}\PY{l+s+s2}{\PYZdq{}}\PY{l+s+s2}{julian}\PY{l+s+s2}{\PYZdq{}}\PY{p}{)}
         \PY{n+nb}{print}\PY{p}{(}\PY{l+s+s2}{\PYZdq{}}\PY{l+s+s2}{nuevo diccionario: }\PY{l+s+s2}{\PYZdq{}}\PY{p}{,} \PY{n}{dic}\PY{p}{)}
         \PY{n}{a}\PY{o}{=}\PY{n}{eliminar}\PY{p}{(}\PY{n}{d}\PY{p}{,}\PY{l+s+s2}{\PYZdq{}}\PY{l+s+s2}{ana}\PY{l+s+s2}{\PYZdq{}}\PY{p}{)}
\end{Verbatim}


    \begin{Verbatim}[commandchars=\\\{\}]
nuevo diccionario:  \{'pablo': 10, 'juana': 15, 'laura': 12\}
La llave ana no fue encontrada en el diccionario

    \end{Verbatim}

    \subsubsection{12}\label{section}

Escriba una función que arroje los valores mínimo y máximo de un
diccionario.

    \begin{Verbatim}[commandchars=\\\{\}]
{\color{incolor}In [{\color{incolor}93}]:} \PY{k}{def} \PY{n+nf}{encontrar}\PY{p}{(}\PY{n}{dicc}\PY{p}{)}\PY{p}{:}
             \PY{c+c1}{\PYZsh{}\PYZsh{} Asumo que todos los valores en el diccionario son del mismo tipo}
             \PY{n}{val}\PY{o}{=}\PY{n+nb}{list}\PY{p}{(}\PY{n}{dicc}\PY{o}{.}\PY{n}{values}\PY{p}{(}\PY{p}{)}\PY{p}{)}
             \PY{n}{minimo}\PY{o}{=}\PY{n+nb}{min}\PY{p}{(}\PY{n}{val}\PY{p}{)}
             \PY{n}{maximo}\PY{o}{=}\PY{n+nb}{max}\PY{p}{(}\PY{n}{val}\PY{p}{)}
             \PY{n+nb}{print}\PY{p}{(}\PY{l+s+s2}{\PYZdq{}}\PY{l+s+s2}{Valor minimo: }\PY{l+s+s2}{\PYZdq{}}\PY{p}{,} \PY{n}{minimo}\PY{p}{)}
             \PY{n+nb}{print}\PY{p}{(}\PY{l+s+s2}{\PYZdq{}}\PY{l+s+s2}{valor maximo: }\PY{l+s+s2}{\PYZdq{}}\PY{p}{,} \PY{n}{maximo}\PY{p}{)}
         
         \PY{n}{d}\PY{o}{=}\PY{p}{\PYZob{}}\PY{l+s+s1}{\PYZsq{}}\PY{l+s+s1}{julian}\PY{l+s+s1}{\PYZsq{}}\PY{p}{:} \PY{l+m+mi}{20}\PY{p}{,} \PY{l+s+s1}{\PYZsq{}}\PY{l+s+s1}{pablo}\PY{l+s+s1}{\PYZsq{}}\PY{p}{:} \PY{l+m+mi}{10}\PY{p}{,} \PY{l+s+s1}{\PYZsq{}}\PY{l+s+s1}{juana}\PY{l+s+s1}{\PYZsq{}}\PY{p}{:} \PY{l+m+mi}{15}\PY{p}{,} \PY{l+s+s1}{\PYZsq{}}\PY{l+s+s1}{laura}\PY{l+s+s1}{\PYZsq{}}\PY{p}{:} \PY{l+m+mi}{12}\PY{p}{\PYZcb{}}
         \PY{n}{encontrar}\PY{p}{(}\PY{n}{d}\PY{p}{)}
             
             
\end{Verbatim}


    \begin{Verbatim}[commandchars=\\\{\}]
Valor minimo:  10
valor maximo:  20

    \end{Verbatim}

    \subsubsection{13}\label{section}

    sentence = "the quick brown fox jumps over the lazy dog" words =
sentence.split() word\_lengths = {[}{]} for word in words: if word !=
"the": word\_lengths.append(len(word))

Simplifique el código anterior combinando las líneas 3 a 6 usando list
comprehension. Su código final deberá entonces tener tres líneas.

    \begin{Verbatim}[commandchars=\\\{\}]
{\color{incolor}In [{\color{incolor}102}]:} \PY{n}{sentence} \PY{o}{=} \PY{l+s+s2}{\PYZdq{}}\PY{l+s+s2}{the quick brown fox jumps over the lazy dog}\PY{l+s+s2}{\PYZdq{}}
          \PY{n}{words} \PY{o}{=} \PY{n}{sentence}\PY{o}{.}\PY{n}{split}\PY{p}{(}\PY{p}{)} 
          \PY{n}{word\PYZus{}lengths\PYZus{}mio} \PY{o}{=} \PY{p}{[}\PY{n+nb}{len}\PY{p}{(}\PY{n}{x}\PY{p}{)} \PY{k}{for} \PY{n}{x} \PY{o+ow}{in} \PY{n}{words} \PY{k}{if} \PY{n}{x}\PY{o}{!=}\PY{l+s+s2}{\PYZdq{}}\PY{l+s+s2}{the}\PY{l+s+s2}{\PYZdq{}}\PY{p}{]}
          \PY{c+c1}{\PYZsh{}\PYZsh{}\PYZsh{}\PYZsh{}\PYZsh{}\PYZsh{}\PYZsh{}\PYZsh{}\PYZsh{}\PYZsh{}\PYZsh{}\PYZsh{}\PYZsh{}\PYZsh{}\PYZsh{}\PYZsh{}\PYZsh{}\PYZsh{}\PYZsh{}\PYZsh{} prueba de que esta bien}
          \PY{n+nb}{print}\PY{p}{(}\PY{n}{word\PYZus{}lengths\PYZus{}mio}\PY{p}{)}
\end{Verbatim}


    \begin{Verbatim}[commandchars=\\\{\}]
[5, 5, 3, 5, 4, 4, 3]

    \end{Verbatim}

    \subsubsection{14}\label{section}

    Escriba UNA línea de código que tome la lista a y arroje una nueva lista
con solo los elementos pares de a.

    \begin{Verbatim}[commandchars=\\\{\}]
{\color{incolor}In [{\color{incolor}103}]:} \PY{c+c1}{\PYZsh{}\PYZsh{} En realidad dos por que necesito escribir la lista a}
          \PY{n}{a}\PY{o}{=}\PY{p}{[}\PY{l+m+mi}{1}\PY{p}{,}\PY{l+m+mi}{2}\PY{p}{,}\PY{l+m+mi}{3}\PY{p}{,}\PY{l+m+mi}{4}\PY{p}{,}\PY{l+m+mi}{5}\PY{p}{,}\PY{l+m+mi}{6}\PY{p}{,}\PY{l+m+mi}{7}\PY{p}{,}\PY{l+m+mi}{8}\PY{p}{,}\PY{l+m+mi}{9}\PY{p}{,}\PY{l+m+mi}{20}\PY{p}{,}\PY{l+m+mi}{22}\PY{p}{,}\PY{l+m+mi}{44}\PY{p}{]}
          \PY{n+nb}{print}\PY{p}{(}\PY{p}{[}\PY{n}{x} \PY{k}{for} \PY{n}{x} \PY{o+ow}{in} \PY{n}{a} \PY{k}{if} \PY{n}{x}\PY{o}{\PYZpc{}}\PY{k}{2}==0])
\end{Verbatim}


    \begin{Verbatim}[commandchars=\\\{\}]
[2, 4, 6, 8, 20, 22, 44]

    \end{Verbatim}

    \subsubsection{15}\label{section}

    Escriba UNA línea de código que tome la lista a del ejercicio 14 y
multiplique todos sus valores.

    \begin{Verbatim}[commandchars=\\\{\}]
{\color{incolor}In [{\color{incolor}105}]:} \PY{k+kn}{from} \PY{n+nn}{functools} \PY{k}{import} \PY{n}{reduce}
\end{Verbatim}


    \begin{Verbatim}[commandchars=\\\{\}]
{\color{incolor}In [{\color{incolor}106}]:} \PY{n}{reduce}\PY{p}{(}\PY{k}{lambda} \PY{n}{x}\PY{p}{,}\PY{n}{y}\PY{p}{:} \PY{n}{x}\PY{o}{*}\PY{n}{y}\PY{p}{,} \PY{n}{a}\PY{p}{)}
\end{Verbatim}


\begin{Verbatim}[commandchars=\\\{\}]
{\color{outcolor}Out[{\color{outcolor}106}]:} 7025356800
\end{Verbatim}
            
    \subsubsection{16}\label{section}

    Usando "list comprehension", cree una lista con las 36 combinaciones de
un par de dados, como tuplas: {[}(1,1), (1,2),...,(6,6){]}.

    \begin{Verbatim}[commandchars=\\\{\}]
{\color{incolor}In [{\color{incolor}107}]:} \PY{p}{[}\PY{p}{(}\PY{n}{x}\PY{p}{,}\PY{n}{y}\PY{p}{)} \PY{k}{for} \PY{n}{x} \PY{o+ow}{in} \PY{n+nb}{range}\PY{p}{(}\PY{l+m+mi}{1}\PY{p}{,}\PY{l+m+mi}{7}\PY{p}{)} \PY{k}{for} \PY{n}{y} \PY{o+ow}{in} \PY{n+nb}{range}\PY{p}{(}\PY{l+m+mi}{1}\PY{p}{,}\PY{l+m+mi}{7}\PY{p}{)}\PY{p}{]}
\end{Verbatim}


\begin{Verbatim}[commandchars=\\\{\}]
{\color{outcolor}Out[{\color{outcolor}107}]:} [(1, 1),
           (1, 2),
           (1, 3),
           (1, 4),
           (1, 5),
           (1, 6),
           (2, 1),
           (2, 2),
           (2, 3),
           (2, 4),
           (2, 5),
           (2, 6),
           (3, 1),
           (3, 2),
           (3, 3),
           (3, 4),
           (3, 5),
           (3, 6),
           (4, 1),
           (4, 2),
           (4, 3),
           (4, 4),
           (4, 5),
           (4, 6),
           (5, 1),
           (5, 2),
           (5, 3),
           (5, 4),
           (5, 5),
           (5, 6),
           (6, 1),
           (6, 2),
           (6, 3),
           (6, 4),
           (6, 5),
           (6, 6)]
\end{Verbatim}
            
    \begin{center}\rule{0.5\linewidth}{\linethickness}\end{center}


    % Add a bibliography block to the postdoc
    
    
    
    \end{document}
