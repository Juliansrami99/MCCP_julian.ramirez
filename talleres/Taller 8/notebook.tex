
% Default to the notebook output style

    


% Inherit from the specified cell style.




    
\documentclass[11pt]{article}

    
    
    \usepackage[T1]{fontenc}
    % Nicer default font (+ math font) than Computer Modern for most use cases
    \usepackage{mathpazo}

    % Basic figure setup, for now with no caption control since it's done
    % automatically by Pandoc (which extracts ![](path) syntax from Markdown).
    \usepackage{graphicx}
    % We will generate all images so they have a width \maxwidth. This means
    % that they will get their normal width if they fit onto the page, but
    % are scaled down if they would overflow the margins.
    \makeatletter
    \def\maxwidth{\ifdim\Gin@nat@width>\linewidth\linewidth
    \else\Gin@nat@width\fi}
    \makeatother
    \let\Oldincludegraphics\includegraphics
    % Set max figure width to be 80% of text width, for now hardcoded.
    \renewcommand{\includegraphics}[1]{\Oldincludegraphics[width=.8\maxwidth]{#1}}
    % Ensure that by default, figures have no caption (until we provide a
    % proper Figure object with a Caption API and a way to capture that
    % in the conversion process - todo).
    \usepackage{caption}
    \DeclareCaptionLabelFormat{nolabel}{}
    \captionsetup{labelformat=nolabel}

    \usepackage{adjustbox} % Used to constrain images to a maximum size 
    \usepackage{xcolor} % Allow colors to be defined
    \usepackage{enumerate} % Needed for markdown enumerations to work
    \usepackage{geometry} % Used to adjust the document margins
    \usepackage{amsmath} % Equations
    \usepackage{amssymb} % Equations
    \usepackage{textcomp} % defines textquotesingle
    % Hack from http://tex.stackexchange.com/a/47451/13684:
    \AtBeginDocument{%
        \def\PYZsq{\textquotesingle}% Upright quotes in Pygmentized code
    }
    \usepackage{upquote} % Upright quotes for verbatim code
    \usepackage{eurosym} % defines \euro
    \usepackage[mathletters]{ucs} % Extended unicode (utf-8) support
    \usepackage[utf8x]{inputenc} % Allow utf-8 characters in the tex document
    \usepackage{fancyvrb} % verbatim replacement that allows latex
    \usepackage{grffile} % extends the file name processing of package graphics 
                         % to support a larger range 
    % The hyperref package gives us a pdf with properly built
    % internal navigation ('pdf bookmarks' for the table of contents,
    % internal cross-reference links, web links for URLs, etc.)
    \usepackage{hyperref}
    \usepackage{longtable} % longtable support required by pandoc >1.10
    \usepackage{booktabs}  % table support for pandoc > 1.12.2
    \usepackage[inline]{enumitem} % IRkernel/repr support (it uses the enumerate* environment)
    \usepackage[normalem]{ulem} % ulem is needed to support strikethroughs (\sout)
                                % normalem makes italics be italics, not underlines
    

    
    
    % Colors for the hyperref package
    \definecolor{urlcolor}{rgb}{0,.145,.698}
    \definecolor{linkcolor}{rgb}{.71,0.21,0.01}
    \definecolor{citecolor}{rgb}{.12,.54,.11}

    % ANSI colors
    \definecolor{ansi-black}{HTML}{3E424D}
    \definecolor{ansi-black-intense}{HTML}{282C36}
    \definecolor{ansi-red}{HTML}{E75C58}
    \definecolor{ansi-red-intense}{HTML}{B22B31}
    \definecolor{ansi-green}{HTML}{00A250}
    \definecolor{ansi-green-intense}{HTML}{007427}
    \definecolor{ansi-yellow}{HTML}{DDB62B}
    \definecolor{ansi-yellow-intense}{HTML}{B27D12}
    \definecolor{ansi-blue}{HTML}{208FFB}
    \definecolor{ansi-blue-intense}{HTML}{0065CA}
    \definecolor{ansi-magenta}{HTML}{D160C4}
    \definecolor{ansi-magenta-intense}{HTML}{A03196}
    \definecolor{ansi-cyan}{HTML}{60C6C8}
    \definecolor{ansi-cyan-intense}{HTML}{258F8F}
    \definecolor{ansi-white}{HTML}{C5C1B4}
    \definecolor{ansi-white-intense}{HTML}{A1A6B2}

    % commands and environments needed by pandoc snippets
    % extracted from the output of `pandoc -s`
    \providecommand{\tightlist}{%
      \setlength{\itemsep}{0pt}\setlength{\parskip}{0pt}}
    \DefineVerbatimEnvironment{Highlighting}{Verbatim}{commandchars=\\\{\}}
    % Add ',fontsize=\small' for more characters per line
    \newenvironment{Shaded}{}{}
    \newcommand{\KeywordTok}[1]{\textcolor[rgb]{0.00,0.44,0.13}{\textbf{{#1}}}}
    \newcommand{\DataTypeTok}[1]{\textcolor[rgb]{0.56,0.13,0.00}{{#1}}}
    \newcommand{\DecValTok}[1]{\textcolor[rgb]{0.25,0.63,0.44}{{#1}}}
    \newcommand{\BaseNTok}[1]{\textcolor[rgb]{0.25,0.63,0.44}{{#1}}}
    \newcommand{\FloatTok}[1]{\textcolor[rgb]{0.25,0.63,0.44}{{#1}}}
    \newcommand{\CharTok}[1]{\textcolor[rgb]{0.25,0.44,0.63}{{#1}}}
    \newcommand{\StringTok}[1]{\textcolor[rgb]{0.25,0.44,0.63}{{#1}}}
    \newcommand{\CommentTok}[1]{\textcolor[rgb]{0.38,0.63,0.69}{\textit{{#1}}}}
    \newcommand{\OtherTok}[1]{\textcolor[rgb]{0.00,0.44,0.13}{{#1}}}
    \newcommand{\AlertTok}[1]{\textcolor[rgb]{1.00,0.00,0.00}{\textbf{{#1}}}}
    \newcommand{\FunctionTok}[1]{\textcolor[rgb]{0.02,0.16,0.49}{{#1}}}
    \newcommand{\RegionMarkerTok}[1]{{#1}}
    \newcommand{\ErrorTok}[1]{\textcolor[rgb]{1.00,0.00,0.00}{\textbf{{#1}}}}
    \newcommand{\NormalTok}[1]{{#1}}
    
    % Additional commands for more recent versions of Pandoc
    \newcommand{\ConstantTok}[1]{\textcolor[rgb]{0.53,0.00,0.00}{{#1}}}
    \newcommand{\SpecialCharTok}[1]{\textcolor[rgb]{0.25,0.44,0.63}{{#1}}}
    \newcommand{\VerbatimStringTok}[1]{\textcolor[rgb]{0.25,0.44,0.63}{{#1}}}
    \newcommand{\SpecialStringTok}[1]{\textcolor[rgb]{0.73,0.40,0.53}{{#1}}}
    \newcommand{\ImportTok}[1]{{#1}}
    \newcommand{\DocumentationTok}[1]{\textcolor[rgb]{0.73,0.13,0.13}{\textit{{#1}}}}
    \newcommand{\AnnotationTok}[1]{\textcolor[rgb]{0.38,0.63,0.69}{\textbf{\textit{{#1}}}}}
    \newcommand{\CommentVarTok}[1]{\textcolor[rgb]{0.38,0.63,0.69}{\textbf{\textit{{#1}}}}}
    \newcommand{\VariableTok}[1]{\textcolor[rgb]{0.10,0.09,0.49}{{#1}}}
    \newcommand{\ControlFlowTok}[1]{\textcolor[rgb]{0.00,0.44,0.13}{\textbf{{#1}}}}
    \newcommand{\OperatorTok}[1]{\textcolor[rgb]{0.40,0.40,0.40}{{#1}}}
    \newcommand{\BuiltInTok}[1]{{#1}}
    \newcommand{\ExtensionTok}[1]{{#1}}
    \newcommand{\PreprocessorTok}[1]{\textcolor[rgb]{0.74,0.48,0.00}{{#1}}}
    \newcommand{\AttributeTok}[1]{\textcolor[rgb]{0.49,0.56,0.16}{{#1}}}
    \newcommand{\InformationTok}[1]{\textcolor[rgb]{0.38,0.63,0.69}{\textbf{\textit{{#1}}}}}
    \newcommand{\WarningTok}[1]{\textcolor[rgb]{0.38,0.63,0.69}{\textbf{\textit{{#1}}}}}
    
    
    % Define a nice break command that doesn't care if a line doesn't already
    % exist.
    \def\br{\hspace*{\fill} \\* }
    % Math Jax compatability definitions
    \def\gt{>}
    \def\lt{<}
    % Document parameters
    \title{mcpp\_taller8\_Julian\_Ramirez}
    
    
    

    % Pygments definitions
    
\makeatletter
\def\PY@reset{\let\PY@it=\relax \let\PY@bf=\relax%
    \let\PY@ul=\relax \let\PY@tc=\relax%
    \let\PY@bc=\relax \let\PY@ff=\relax}
\def\PY@tok#1{\csname PY@tok@#1\endcsname}
\def\PY@toks#1+{\ifx\relax#1\empty\else%
    \PY@tok{#1}\expandafter\PY@toks\fi}
\def\PY@do#1{\PY@bc{\PY@tc{\PY@ul{%
    \PY@it{\PY@bf{\PY@ff{#1}}}}}}}
\def\PY#1#2{\PY@reset\PY@toks#1+\relax+\PY@do{#2}}

\expandafter\def\csname PY@tok@w\endcsname{\def\PY@tc##1{\textcolor[rgb]{0.73,0.73,0.73}{##1}}}
\expandafter\def\csname PY@tok@c\endcsname{\let\PY@it=\textit\def\PY@tc##1{\textcolor[rgb]{0.25,0.50,0.50}{##1}}}
\expandafter\def\csname PY@tok@cp\endcsname{\def\PY@tc##1{\textcolor[rgb]{0.74,0.48,0.00}{##1}}}
\expandafter\def\csname PY@tok@k\endcsname{\let\PY@bf=\textbf\def\PY@tc##1{\textcolor[rgb]{0.00,0.50,0.00}{##1}}}
\expandafter\def\csname PY@tok@kp\endcsname{\def\PY@tc##1{\textcolor[rgb]{0.00,0.50,0.00}{##1}}}
\expandafter\def\csname PY@tok@kt\endcsname{\def\PY@tc##1{\textcolor[rgb]{0.69,0.00,0.25}{##1}}}
\expandafter\def\csname PY@tok@o\endcsname{\def\PY@tc##1{\textcolor[rgb]{0.40,0.40,0.40}{##1}}}
\expandafter\def\csname PY@tok@ow\endcsname{\let\PY@bf=\textbf\def\PY@tc##1{\textcolor[rgb]{0.67,0.13,1.00}{##1}}}
\expandafter\def\csname PY@tok@nb\endcsname{\def\PY@tc##1{\textcolor[rgb]{0.00,0.50,0.00}{##1}}}
\expandafter\def\csname PY@tok@nf\endcsname{\def\PY@tc##1{\textcolor[rgb]{0.00,0.00,1.00}{##1}}}
\expandafter\def\csname PY@tok@nc\endcsname{\let\PY@bf=\textbf\def\PY@tc##1{\textcolor[rgb]{0.00,0.00,1.00}{##1}}}
\expandafter\def\csname PY@tok@nn\endcsname{\let\PY@bf=\textbf\def\PY@tc##1{\textcolor[rgb]{0.00,0.00,1.00}{##1}}}
\expandafter\def\csname PY@tok@ne\endcsname{\let\PY@bf=\textbf\def\PY@tc##1{\textcolor[rgb]{0.82,0.25,0.23}{##1}}}
\expandafter\def\csname PY@tok@nv\endcsname{\def\PY@tc##1{\textcolor[rgb]{0.10,0.09,0.49}{##1}}}
\expandafter\def\csname PY@tok@no\endcsname{\def\PY@tc##1{\textcolor[rgb]{0.53,0.00,0.00}{##1}}}
\expandafter\def\csname PY@tok@nl\endcsname{\def\PY@tc##1{\textcolor[rgb]{0.63,0.63,0.00}{##1}}}
\expandafter\def\csname PY@tok@ni\endcsname{\let\PY@bf=\textbf\def\PY@tc##1{\textcolor[rgb]{0.60,0.60,0.60}{##1}}}
\expandafter\def\csname PY@tok@na\endcsname{\def\PY@tc##1{\textcolor[rgb]{0.49,0.56,0.16}{##1}}}
\expandafter\def\csname PY@tok@nt\endcsname{\let\PY@bf=\textbf\def\PY@tc##1{\textcolor[rgb]{0.00,0.50,0.00}{##1}}}
\expandafter\def\csname PY@tok@nd\endcsname{\def\PY@tc##1{\textcolor[rgb]{0.67,0.13,1.00}{##1}}}
\expandafter\def\csname PY@tok@s\endcsname{\def\PY@tc##1{\textcolor[rgb]{0.73,0.13,0.13}{##1}}}
\expandafter\def\csname PY@tok@sd\endcsname{\let\PY@it=\textit\def\PY@tc##1{\textcolor[rgb]{0.73,0.13,0.13}{##1}}}
\expandafter\def\csname PY@tok@si\endcsname{\let\PY@bf=\textbf\def\PY@tc##1{\textcolor[rgb]{0.73,0.40,0.53}{##1}}}
\expandafter\def\csname PY@tok@se\endcsname{\let\PY@bf=\textbf\def\PY@tc##1{\textcolor[rgb]{0.73,0.40,0.13}{##1}}}
\expandafter\def\csname PY@tok@sr\endcsname{\def\PY@tc##1{\textcolor[rgb]{0.73,0.40,0.53}{##1}}}
\expandafter\def\csname PY@tok@ss\endcsname{\def\PY@tc##1{\textcolor[rgb]{0.10,0.09,0.49}{##1}}}
\expandafter\def\csname PY@tok@sx\endcsname{\def\PY@tc##1{\textcolor[rgb]{0.00,0.50,0.00}{##1}}}
\expandafter\def\csname PY@tok@m\endcsname{\def\PY@tc##1{\textcolor[rgb]{0.40,0.40,0.40}{##1}}}
\expandafter\def\csname PY@tok@gh\endcsname{\let\PY@bf=\textbf\def\PY@tc##1{\textcolor[rgb]{0.00,0.00,0.50}{##1}}}
\expandafter\def\csname PY@tok@gu\endcsname{\let\PY@bf=\textbf\def\PY@tc##1{\textcolor[rgb]{0.50,0.00,0.50}{##1}}}
\expandafter\def\csname PY@tok@gd\endcsname{\def\PY@tc##1{\textcolor[rgb]{0.63,0.00,0.00}{##1}}}
\expandafter\def\csname PY@tok@gi\endcsname{\def\PY@tc##1{\textcolor[rgb]{0.00,0.63,0.00}{##1}}}
\expandafter\def\csname PY@tok@gr\endcsname{\def\PY@tc##1{\textcolor[rgb]{1.00,0.00,0.00}{##1}}}
\expandafter\def\csname PY@tok@ge\endcsname{\let\PY@it=\textit}
\expandafter\def\csname PY@tok@gs\endcsname{\let\PY@bf=\textbf}
\expandafter\def\csname PY@tok@gp\endcsname{\let\PY@bf=\textbf\def\PY@tc##1{\textcolor[rgb]{0.00,0.00,0.50}{##1}}}
\expandafter\def\csname PY@tok@go\endcsname{\def\PY@tc##1{\textcolor[rgb]{0.53,0.53,0.53}{##1}}}
\expandafter\def\csname PY@tok@gt\endcsname{\def\PY@tc##1{\textcolor[rgb]{0.00,0.27,0.87}{##1}}}
\expandafter\def\csname PY@tok@err\endcsname{\def\PY@bc##1{\setlength{\fboxsep}{0pt}\fcolorbox[rgb]{1.00,0.00,0.00}{1,1,1}{\strut ##1}}}
\expandafter\def\csname PY@tok@kc\endcsname{\let\PY@bf=\textbf\def\PY@tc##1{\textcolor[rgb]{0.00,0.50,0.00}{##1}}}
\expandafter\def\csname PY@tok@kd\endcsname{\let\PY@bf=\textbf\def\PY@tc##1{\textcolor[rgb]{0.00,0.50,0.00}{##1}}}
\expandafter\def\csname PY@tok@kn\endcsname{\let\PY@bf=\textbf\def\PY@tc##1{\textcolor[rgb]{0.00,0.50,0.00}{##1}}}
\expandafter\def\csname PY@tok@kr\endcsname{\let\PY@bf=\textbf\def\PY@tc##1{\textcolor[rgb]{0.00,0.50,0.00}{##1}}}
\expandafter\def\csname PY@tok@bp\endcsname{\def\PY@tc##1{\textcolor[rgb]{0.00,0.50,0.00}{##1}}}
\expandafter\def\csname PY@tok@fm\endcsname{\def\PY@tc##1{\textcolor[rgb]{0.00,0.00,1.00}{##1}}}
\expandafter\def\csname PY@tok@vc\endcsname{\def\PY@tc##1{\textcolor[rgb]{0.10,0.09,0.49}{##1}}}
\expandafter\def\csname PY@tok@vg\endcsname{\def\PY@tc##1{\textcolor[rgb]{0.10,0.09,0.49}{##1}}}
\expandafter\def\csname PY@tok@vi\endcsname{\def\PY@tc##1{\textcolor[rgb]{0.10,0.09,0.49}{##1}}}
\expandafter\def\csname PY@tok@vm\endcsname{\def\PY@tc##1{\textcolor[rgb]{0.10,0.09,0.49}{##1}}}
\expandafter\def\csname PY@tok@sa\endcsname{\def\PY@tc##1{\textcolor[rgb]{0.73,0.13,0.13}{##1}}}
\expandafter\def\csname PY@tok@sb\endcsname{\def\PY@tc##1{\textcolor[rgb]{0.73,0.13,0.13}{##1}}}
\expandafter\def\csname PY@tok@sc\endcsname{\def\PY@tc##1{\textcolor[rgb]{0.73,0.13,0.13}{##1}}}
\expandafter\def\csname PY@tok@dl\endcsname{\def\PY@tc##1{\textcolor[rgb]{0.73,0.13,0.13}{##1}}}
\expandafter\def\csname PY@tok@s2\endcsname{\def\PY@tc##1{\textcolor[rgb]{0.73,0.13,0.13}{##1}}}
\expandafter\def\csname PY@tok@sh\endcsname{\def\PY@tc##1{\textcolor[rgb]{0.73,0.13,0.13}{##1}}}
\expandafter\def\csname PY@tok@s1\endcsname{\def\PY@tc##1{\textcolor[rgb]{0.73,0.13,0.13}{##1}}}
\expandafter\def\csname PY@tok@mb\endcsname{\def\PY@tc##1{\textcolor[rgb]{0.40,0.40,0.40}{##1}}}
\expandafter\def\csname PY@tok@mf\endcsname{\def\PY@tc##1{\textcolor[rgb]{0.40,0.40,0.40}{##1}}}
\expandafter\def\csname PY@tok@mh\endcsname{\def\PY@tc##1{\textcolor[rgb]{0.40,0.40,0.40}{##1}}}
\expandafter\def\csname PY@tok@mi\endcsname{\def\PY@tc##1{\textcolor[rgb]{0.40,0.40,0.40}{##1}}}
\expandafter\def\csname PY@tok@il\endcsname{\def\PY@tc##1{\textcolor[rgb]{0.40,0.40,0.40}{##1}}}
\expandafter\def\csname PY@tok@mo\endcsname{\def\PY@tc##1{\textcolor[rgb]{0.40,0.40,0.40}{##1}}}
\expandafter\def\csname PY@tok@ch\endcsname{\let\PY@it=\textit\def\PY@tc##1{\textcolor[rgb]{0.25,0.50,0.50}{##1}}}
\expandafter\def\csname PY@tok@cm\endcsname{\let\PY@it=\textit\def\PY@tc##1{\textcolor[rgb]{0.25,0.50,0.50}{##1}}}
\expandafter\def\csname PY@tok@cpf\endcsname{\let\PY@it=\textit\def\PY@tc##1{\textcolor[rgb]{0.25,0.50,0.50}{##1}}}
\expandafter\def\csname PY@tok@c1\endcsname{\let\PY@it=\textit\def\PY@tc##1{\textcolor[rgb]{0.25,0.50,0.50}{##1}}}
\expandafter\def\csname PY@tok@cs\endcsname{\let\PY@it=\textit\def\PY@tc##1{\textcolor[rgb]{0.25,0.50,0.50}{##1}}}

\def\PYZbs{\char`\\}
\def\PYZus{\char`\_}
\def\PYZob{\char`\{}
\def\PYZcb{\char`\}}
\def\PYZca{\char`\^}
\def\PYZam{\char`\&}
\def\PYZlt{\char`\<}
\def\PYZgt{\char`\>}
\def\PYZsh{\char`\#}
\def\PYZpc{\char`\%}
\def\PYZdl{\char`\$}
\def\PYZhy{\char`\-}
\def\PYZsq{\char`\'}
\def\PYZdq{\char`\"}
\def\PYZti{\char`\~}
% for compatibility with earlier versions
\def\PYZat{@}
\def\PYZlb{[}
\def\PYZrb{]}
\makeatother


    % Exact colors from NB
    \definecolor{incolor}{rgb}{0.0, 0.0, 0.5}
    \definecolor{outcolor}{rgb}{0.545, 0.0, 0.0}



    
    % Prevent overflowing lines due to hard-to-break entities
    \sloppy 
    % Setup hyperref package
    \hypersetup{
      breaklinks=true,  % so long urls are correctly broken across lines
      colorlinks=true,
      urlcolor=urlcolor,
      linkcolor=linkcolor,
      citecolor=citecolor,
      }
    % Slightly bigger margins than the latex defaults
    
    \geometry{verbose,tmargin=1in,bmargin=1in,lmargin=1in,rmargin=1in}
    
    

    \begin{document}
    
    
    \maketitle
    
    

    
    \section{Taller 8}\label{taller-8}

Métodos Computacionales para Políticas Públicas - URosario

\textbf{Entrega: viernes 18-oct-2019 11:59 PM}

    \textbf{Julián Santiago Ramírez} julians.ramirez@urosario.edu.co

    \subsection{Instrucciones:}\label{instrucciones}

\begin{itemize}
\tightlist
\item
  Guarde una copia de este \emph{Jupyter Notebook} en su computador,
  idealmente en una carpeta destinada al material del curso.
\item
  Modifique el nombre del archivo del \emph{notebook}, agregando al
  final un guión inferior y su nombre y apellido, separados estos
  últimos por otro guión inferior. Por ejemplo, mi \emph{notebook} se
  llamaría: mcpp\_taller8\_santiago\_matallana
\item
  Marque el \emph{notebook} con su nombre y e-mail en el bloque verde
  arriba. Reemplace el texto "{[}Su nombre acá{]}" con su nombre y
  apellido. Similar para su e-mail.
\item
  Desarrolle la totalidad del taller sobre este \emph{notebook},
  insertando las celdas que sea necesario debajo de cada pregunta. Haga
  buen uso de las celdas para código y de las celdas tipo
  \emph{markdown} según el caso.
\item
  Recuerde salvar periódicamente sus avances.
\item
  Cuando termine el taller:

  \begin{enumerate}
  \def\labelenumi{\arabic{enumi}.}
  \tightlist
  \item
    Descárguelo en PDF. Si tiene algún problema con la conversión,
    descárguelo en HTML.
  \item
    Suba todos los archivos a su repositorio en GitHub, en una carpeta
    destinada exclusivamente para este taller, antes de la fecha y hora
    límites.
  \end{enumerate}
\end{itemize}

    \begin{center}\rule{0.5\linewidth}{\linethickness}\end{center}

    \subsubsection{1. {[}1 punto{]}}\label{punto}

Usando expresiones regulares extraiga en una lista todos los números
presentes en el siguiente objeto de Python:

ob1 = "JEFF BEZOS, the founder of Amazon, has reached a divorce
settlement with his wife, MacKenzie. Mr Bezos will keep all the shares
in the Washington Post and Blue Origin, a space-exploration firm, as
well as 75\% of the couple's Amazon stock. Mrs Bezos will retain a 4\%
stake in the tech giant, worth nearly \$36bn, which is likely to make
her the third-richest woman alive when the divorce is finalised."

    \begin{Verbatim}[commandchars=\\\{\}]
{\color{incolor}In [{\color{incolor}1}]:} \PY{c+c1}{\PYZsh{}\PYZsh{}\PYZsh{}\PYZsh{} Importamos la libreria que nos permite usar expresiones regulares \PYZsh{}\PYZsh{}\PYZsh{}\PYZsh{}}
        \PY{k+kn}{import} \PY{n+nn}{re}
\end{Verbatim}


    \begin{Verbatim}[commandchars=\\\{\}]
{\color{incolor}In [{\color{incolor}3}]:} \PY{c+c1}{\PYZsh{}\PYZsh{}\PYZsh{}\PYZsh{} variable ob1}
        \PY{n}{ob1} \PY{o}{=} \PY{l+s+s2}{\PYZdq{}}\PY{l+s+s2}{JEFF BEZOS, the founder of Amazon, has reached a divorce settlement with his wife, MacKenzie. Mr Bezos will keep all the shares in the Washington Post and Blue Origin, a space\PYZhy{}exploration firm, as well as 75}\PY{l+s+si}{\PYZpc{} o}\PY{l+s+s2}{f the couple’s Amazon stock. Mrs Bezos will retain a 4}\PY{l+s+si}{\PYZpc{} s}\PY{l+s+s2}{take in the tech giant, worth nearly \PYZdl{}36bn, which is likely to make her the third\PYZhy{}richest woman alive when the divorce is finalised.}\PY{l+s+s2}{\PYZdq{}}
        
        \PY{n}{numeros}\PY{o}{=}\PY{n}{re}\PY{o}{.}\PY{n}{findall}\PY{p}{(}\PY{l+s+s1}{\PYZsq{}}\PY{l+s+s1}{\PYZbs{}}\PY{l+s+s1}{d+}\PY{l+s+s1}{\PYZsq{}}\PY{p}{,}\PY{n}{ob1}\PY{p}{)}
        
        \PY{n+nb}{print}\PY{p}{(}\PY{l+s+s2}{\PYZdq{}}\PY{l+s+s2}{Números en el texto: }\PY{l+s+s2}{\PYZdq{}}\PY{p}{,}\PY{n}{numeros}\PY{p}{)}
\end{Verbatim}


    \begin{Verbatim}[commandchars=\\\{\}]
Números en el texto:  ['75', '4', '36']

    \end{Verbatim}

    \subsubsection{2. {[}1 punto{]}}\label{punto}

Usando expresiones regulares ahora extraiga de \emph{ob1} sólo los
números que correspondan a porcentajes.

    \begin{Verbatim}[commandchars=\\\{\}]
{\color{incolor}In [{\color{incolor}5}]:} \PY{n}{porcentajes}\PY{o}{=}\PY{n}{re}\PY{o}{.}\PY{n}{findall}\PY{p}{(}\PY{l+s+s1}{\PYZsq{}}\PY{l+s+s1}{\PYZbs{}}\PY{l+s+s1}{d+}\PY{l+s+s1}{\PYZpc{}}\PY{l+s+s1}{\PYZsq{}}\PY{p}{,}\PY{n}{ob1}\PY{p}{)}
        
        \PY{n+nb}{print}\PY{p}{(}\PY{l+s+s2}{\PYZdq{}}\PY{l+s+s2}{Números con porcentaje en el texto: }\PY{l+s+s2}{\PYZdq{}}\PY{p}{,}\PY{n}{porcentajes}\PY{p}{)}
\end{Verbatim}


    \begin{Verbatim}[commandchars=\\\{\}]
Números con porcentaje en el texto:  ['75\%', '4\%']

    \end{Verbatim}

    \subsubsection{3. {[}2 puntos{]}}\label{puntos}

Usando expresiones regulares, escriba una función de Python que reciba
una fecha en formato \textbf{Marzo 7, 2019} y retorne la fecha en
formato \textbf{2019-07-03}

    \begin{Verbatim}[commandchars=\\\{\}]
{\color{incolor}In [{\color{incolor}18}]:} \PY{k}{def} \PY{n+nf}{conversion\PYZus{}fecha}\PY{p}{(}\PY{n}{fecha}\PY{p}{)}\PY{p}{:}
             \PY{n}{def\PYZus{}fecha} \PY{o}{=} \PY{p}{[}\PY{p}{]}
             \PY{n}{meses} \PY{o}{=} \PY{p}{[}\PY{l+s+s1}{\PYZsq{}}\PY{l+s+s1}{Enero}\PY{l+s+s1}{\PYZsq{}}\PY{p}{,}\PY{l+s+s1}{\PYZsq{}}\PY{l+s+s1}{Febrero}\PY{l+s+s1}{\PYZsq{}}\PY{p}{,}\PY{l+s+s1}{\PYZsq{}}\PY{l+s+s1}{Marzo}\PY{l+s+s1}{\PYZsq{}}\PY{p}{,}\PY{l+s+s1}{\PYZsq{}}\PY{l+s+s1}{Abril}\PY{l+s+s1}{\PYZsq{}}\PY{p}{,}\PY{l+s+s1}{\PYZsq{}}\PY{l+s+s1}{Mayo}\PY{l+s+s1}{\PYZsq{}}\PY{p}{,}\PY{l+s+s1}{\PYZsq{}}\PY{l+s+s1}{Junio}\PY{l+s+s1}{\PYZsq{}}\PY{p}{,}\PY{l+s+s1}{\PYZsq{}}\PY{l+s+s1}{Julio}\PY{l+s+s1}{\PYZsq{}}\PY{p}{,}
                      \PY{l+s+s1}{\PYZsq{}}\PY{l+s+s1}{Agosto}\PY{l+s+s1}{\PYZsq{}}\PY{p}{,}\PY{l+s+s1}{\PYZsq{}}\PY{l+s+s1}{Septiembre}\PY{l+s+s1}{\PYZsq{}}\PY{p}{,}\PY{l+s+s1}{\PYZsq{}}\PY{l+s+s1}{Octubre}\PY{l+s+s1}{\PYZsq{}}\PY{p}{,}\PY{l+s+s1}{\PYZsq{}}\PY{l+s+s1}{Noviembre}\PY{l+s+s1}{\PYZsq{}}\PY{p}{,}\PY{l+s+s1}{\PYZsq{}}\PY{l+s+s1}{Diciembre}\PY{l+s+s1}{\PYZsq{}}\PY{p}{]}
             \PY{c+c1}{\PYZsh{}\PYZsh{}\PYZsh{} ahora de la fecha queremos obtener el dia y el año}
             \PY{c+c1}{\PYZsh{} Primero buscamos los numeros}
             \PY{n}{numeros} \PY{o}{=}\PY{n}{re}\PY{o}{.}\PY{n}{search}\PY{p}{(}\PY{l+s+s1}{\PYZsq{}}\PY{l+s+s1}{([}\PY{l+s+s1}{\PYZbs{}}\PY{l+s+s1}{d]+), ([}\PY{l+s+s1}{\PYZbs{}}\PY{l+s+s1}{d]+)}\PY{l+s+s1}{\PYZsq{}}\PY{p}{,}\PY{n}{fecha}\PY{p}{)}
             \PY{n}{fec}\PY{o}{=}\PY{n}{numeros}\PY{o}{.}\PY{n}{groups}\PY{p}{(}\PY{p}{)}
             \PY{n}{def\PYZus{}fecha}\PY{o}{.}\PY{n}{append}\PY{p}{(}\PY{n}{fec}\PY{p}{[}\PY{l+m+mi}{1}\PY{p}{]}\PY{p}{)}
             \PY{c+c1}{\PYZsh{}\PYZsh{} El primer dato de fec es el dia y el segundo es el año, por lo tanto preguntamos si el dia es igual a 23}
             \PY{k}{if} \PY{n}{fec}\PY{p}{[}\PY{l+m+mi}{0}\PY{p}{]} \PY{o}{==} \PY{l+s+s1}{\PYZsq{}}\PY{l+s+s1}{23}\PY{l+s+s1}{\PYZsq{}}\PY{p}{:}
                 \PY{n}{dia\PYZus{}actualizado} \PY{o}{=} \PY{l+s+s1}{\PYZsq{}}\PY{l+s+s1}{23}\PY{l+s+s1}{\PYZsq{}}
                 \PY{n}{def\PYZus{}fecha}\PY{o}{.}\PY{n}{append}\PY{p}{(}\PY{n}{dia\PYZus{}actualizado}\PY{p}{)}
             
             \PY{c+c1}{\PYZsh{}\PYZsh{} Hasta el momento hemos agregado el año y el dia, falta el mes}
             \PY{n}{mes}\PY{o}{=}\PY{l+m+mi}{0}
             \PY{k}{for} \PY{n}{i} \PY{o+ow}{in} \PY{n+nb}{range}\PY{p}{(}\PY{l+m+mi}{0}\PY{p}{,}\PY{n+nb}{len}\PY{p}{(}\PY{n}{meses}\PY{p}{)}\PY{p}{)}\PY{p}{:}
                 \PY{k}{if} \PY{n}{meses}\PY{p}{[}\PY{n}{i}\PY{p}{]}\PY{o}{==}\PY{l+s+s2}{\PYZdq{}}\PY{l+s+s2}{Octubre}\PY{l+s+s2}{\PYZdq{}}\PY{p}{:}
                     \PY{n}{mes}\PY{o}{=}\PY{n}{i}\PY{o}{+}\PY{l+m+mi}{1}
                     \PY{k}{break}
             \PY{c+c1}{\PYZsh{}\PYZsh{} Agregamos el mes a la fecha definitiva}
             \PY{n}{def\PYZus{}fecha}\PY{o}{.}\PY{n}{append}\PY{p}{(}\PY{n}{mes}\PY{p}{)}
             \PY{n}{total}\PY{o}{=}\PY{n+nb}{str}\PY{p}{(}\PY{n}{def\PYZus{}fecha}\PY{p}{[}\PY{l+m+mi}{0}\PY{p}{]}\PY{p}{)}\PY{o}{+}\PY{l+s+s1}{\PYZsq{}}\PY{l+s+s1}{\PYZhy{}}\PY{l+s+s1}{\PYZsq{}}\PY{o}{+}\PY{n+nb}{str}\PY{p}{(}\PY{n}{def\PYZus{}fecha}\PY{p}{[}\PY{l+m+mi}{1}\PY{p}{]}\PY{p}{)}\PY{o}{+}\PY{l+s+s1}{\PYZsq{}}\PY{l+s+s1}{\PYZhy{}}\PY{l+s+s1}{\PYZsq{}}\PY{o}{+}\PY{n+nb}{str}\PY{p}{(}\PY{n}{def\PYZus{}fecha}\PY{p}{[}\PY{l+m+mi}{2}\PY{p}{]}\PY{p}{)}
             \PY{k}{return} \PY{n}{total}
         
         \PY{c+c1}{\PYZsh{}\PYZsh{} variable que guarda la fecha \PYZsh{}\PYZsh{}\PYZsh{}}
         \PY{n}{fecha} \PY{o}{=} \PY{l+s+s1}{\PYZsq{}}\PY{l+s+s1}{Octubre 23, 2019}\PY{l+s+s1}{\PYZsq{}}
         \PY{n}{nueva} \PY{o}{=} \PY{n}{conversion\PYZus{}fecha}\PY{p}{(}\PY{n}{fecha}\PY{p}{)}
         
         \PY{n+nb}{print}\PY{p}{(}\PY{l+s+s2}{\PYZdq{}}\PY{l+s+s2}{Antiguo fórmato: }\PY{l+s+s2}{\PYZdq{}}\PY{p}{,} \PY{n}{fecha}\PY{p}{)}
         \PY{n+nb}{print}\PY{p}{(}\PY{l+s+s2}{\PYZdq{}}\PY{l+s+s2}{Nuevo fórmato: }\PY{l+s+s2}{\PYZdq{}}\PY{p}{,}\PY{n}{nueva}\PY{p}{)}
\end{Verbatim}


    \begin{Verbatim}[commandchars=\\\{\}]
Antiguo fórmato:  Octubre 23, 2019
Nuevo fórmato:  2019-23-10

    \end{Verbatim}

    \subsubsection{4. {[}3 puntos{]}}\label{puntos}

\emph{ob2} es un string que reune una lista de clases en una
universidad. Use expresiones regulares para extraer los códigos de cada
una de las clases. Ejemplo: El código de la clase \textbf{COMPSCI 143
(Spring 2012): Machine Learning} es 143.

ob2 = "COMPSCI 270 (Spring 2019): Introduction to Artificial
Intelligence. COMPSCI 590.2 (Fall 2018): Computational Microeconomics:
Game Theory, Social Choice, and Mechanism Design. COMPSCI 223 (Spring
2018): Computational Microeconomics. COMPSCI 570 (Fall 2017): Artificial
Intelligence. COMPSCI 590.3 (Fall 2017) / 590.1 (Spring 2018): Ethics
and AI. COMPSCI 590.2 (Spring 2017): Computation, Information, and
Learning in Market Design. COMPSCI 590.4 (Spring 2016): Computational
Microeconomics: Game Theory, Social Choice, and Mechanism Design.
COMPSCI 290.4/590.4 (Spring 2015): Crowdsourcing Societal Tradeoffs.
COMPSCI 570 (Fall 2014): Artificial Intelligence. COMPSCI 590.4 (Spring
2014): Computational Microeconomics: Game Theory, Social Choice, and
Mechanism Design. COMPSCI 590.1 (Fall 2012): Linear and Integer
Programming. COMPSCI 173 (Spring 2012): Computational Microeconomics.
COMPSCI 296.1 (Fall 2011): Computational Microeconomics: Game Theory,
Social Choice, and Mechanism Design. COMPSCI 296.1 (Fall 2010): Linear
and Integer Programming. COMPSCI 173 (Spring 2010): Computational
Microeconomics. COMPSCI 196.1/296.1 (Fall 2009): Computational
Microeconomics: Game Theory, Social Choice, and Mechanism Design.
COMPSCI 170 (Spring 2009): Introduction to Artificial Intelligence.
COMPSCI 270 (Fall 2008): Artificial Intelligence. COMPSCI 196/296.2
(Spring 2008): Linear and Integer Programming. COMPSCI 196.2 (Fall
2007): Introduction to Computational Economics. COMPSCI 296.3 (Spring
2007): Topics in Computational Economics. COMPSCI 296.2 (Fall 2006):
Computational Game Theory and Mechanism Design."

    \begin{Verbatim}[commandchars=\\\{\}]
{\color{incolor}In [{\color{incolor}30}]:} \PY{c+c1}{\PYZsh{}\PYZsh{} Guardamos ob2 \PYZsh{}\PYZsh{}}
         \PY{n}{ob2} \PY{o}{=} \PY{l+s+s2}{\PYZdq{}}\PY{l+s+s2}{COMPSCI 270 (Spring 2019): Introduction to Artificial Intelligence. COMPSCI 590.2 (Fall 2018): Computational Microeconomics: Game Theory, Social Choice, and Mechanism Design. COMPSCI 223 (Spring 2018): Computational Microeconomics. COMPSCI 570 (Fall 2017): Artificial Intelligence. COMPSCI 590.3 (Fall 2017) / 590.1 (Spring 2018): Ethics and AI. COMPSCI 590.2 (Spring 2017): Computation, Information, and Learning in Market Design. COMPSCI 590.4 (Spring 2016): Computational Microeconomics: Game Theory, Social Choice, and Mechanism Design. COMPSCI 290.4/590.4 (Spring 2015): Crowdsourcing Societal Tradeoffs. COMPSCI 570 (Fall 2014): Artificial Intelligence. COMPSCI 590.4 (Spring 2014): Computational Microeconomics: Game Theory, Social Choice, and Mechanism Design. COMPSCI 590.1 (Fall 2012): Linear and Integer Programming. COMPSCI 173 (Spring 2012): Computational Microeconomics. COMPSCI 296.1 (Fall 2011): Computational Microeconomics: Game Theory, Social Choice, and Mechanism Design. COMPSCI 296.1 (Fall 2010): Linear and Integer Programming. COMPSCI 173 (Spring 2010): Computational Microeconomics. COMPSCI 196.1/296.1 (Fall 2009): Computational Microeconomics: Game Theory, Social Choice, and Mechanism Design. COMPSCI 170 (Spring 2009): Introduction to Artificial Intelligence. COMPSCI 270 (Fall 2008): Artificial Intelligence. COMPSCI 196/296.2 (Spring 2008): Linear and Integer Programming. COMPSCI 196.2 (Fall 2007): Introduction to Computational Economics. COMPSCI 296.3 (Spring 2007): Topics in Computational Economics. COMPSCI 296.2 (Fall 2006): Computational Game Theory and Mechanism Design.}\PY{l+s+s2}{\PYZdq{}}
         
         \PY{c+c1}{\PYZsh{}\PYZsh{} Buscamos los números que acompañan la palabra COMPSCI }
         
         \PY{c+c1}{\PYZsh{}\PYZsh{} Observemos que hay dos clases de codigo: el primero un numero normal y el segundo se puede ver de esta forma: 290.4/590.4}
         
         \PY{c+c1}{\PYZsh{}\PYZsh{} Codigos tipo 1}
         \PY{n}{codigo\PYZus{}tipo1}\PY{o}{=}\PY{n}{re}\PY{o}{.}\PY{n}{findall}\PY{p}{(}\PY{l+s+s1}{\PYZsq{}}\PY{l+s+s1}{ (}\PY{l+s+s1}{\PYZbs{}}\PY{l+s+s1}{d+.}\PY{l+s+s1}{\PYZbs{}}\PY{l+s+s1}{d+) }\PY{l+s+s1}{\PYZsq{}}\PY{p}{,}\PY{n}{ob2}\PY{p}{)} 
         
         \PY{c+c1}{\PYZsh{}\PYZsh{} Codigos tipo 2}
         \PY{n}{codigo\PYZus{}tipo2}\PY{o}{=}\PY{n}{re}\PY{o}{.}\PY{n}{findall}\PY{p}{(}\PY{l+s+s1}{\PYZsq{}}\PY{l+s+s1}{ (}\PY{l+s+s1}{\PYZbs{}}\PY{l+s+s1}{d+.}\PY{l+s+s1}{\PYZbs{}}\PY{l+s+s1}{d+/}\PY{l+s+s1}{\PYZbs{}}\PY{l+s+s1}{d+.}\PY{l+s+s1}{\PYZbs{}}\PY{l+s+s1}{d+) }\PY{l+s+s1}{\PYZsq{}}\PY{p}{,}\PY{n}{ob2}\PY{p}{)}
         \PY{c+c1}{\PYZsh{}\PYZsh{} Supongo que los codigos tipo 2 son dos clases diferentes y por ende debo separarlos}
         \PY{n}{nuevo\PYZus{}codigo}\PY{o}{=}\PY{p}{[}\PY{p}{]}
         
         \PY{k}{for} \PY{n}{cod} \PY{o+ow}{in} \PY{n}{codigo\PYZus{}tipo2}\PY{p}{:}
             \PY{n}{tupla}\PY{o}{=}\PY{n}{re}\PY{o}{.}\PY{n}{search}\PY{p}{(}\PY{l+s+s1}{\PYZsq{}}\PY{l+s+s1}{(}\PY{l+s+s1}{\PYZbs{}}\PY{l+s+s1}{d+.}\PY{l+s+s1}{\PYZbs{}}\PY{l+s+s1}{d)/(}\PY{l+s+s1}{\PYZbs{}}\PY{l+s+s1}{d+.}\PY{l+s+s1}{\PYZbs{}}\PY{l+s+s1}{d)}\PY{l+s+s1}{\PYZsq{}}\PY{p}{,}\PY{n}{cod}\PY{p}{)}\PY{o}{.}\PY{n}{groups}\PY{p}{(}\PY{p}{)}
             \PY{c+c1}{\PYZsh{} agregamos ambos codigos}
             \PY{n}{nuevo\PYZus{}codigo}\PY{o}{.}\PY{n}{append}\PY{p}{(}\PY{n}{tupla}\PY{p}{[}\PY{l+m+mi}{0}\PY{p}{]}\PY{p}{)}
             \PY{n}{nuevo\PYZus{}codigo}\PY{o}{.}\PY{n}{append}\PY{p}{(}\PY{n}{tupla}\PY{p}{[}\PY{l+m+mi}{1}\PY{p}{]}\PY{p}{)}
             
         \PY{n}{total}\PY{o}{=}\PY{n}{codigo\PYZus{}tipo1}\PY{o}{+}\PY{n}{nuevo\PYZus{}codigo}
         
         \PY{n+nb}{print}\PY{p}{(}\PY{l+s+s2}{\PYZdq{}}\PY{l+s+s2}{Códigos:}\PY{l+s+s2}{\PYZdq{}}\PY{p}{,} \PY{n}{total}\PY{p}{)}
         \PY{c+c1}{\PYZsh{}Cod=re.findall(\PYZsq{}/ (\PYZbs{}d+.\PYZbs{}d+)\PYZsq{},ob2) }
\end{Verbatim}


    \begin{Verbatim}[commandchars=\\\{\}]
Codigos: ['270', '590.2', '223', '570', '590.3', '590.1', '590.2', '590.4', '570', '590.4', '590.1', '173', '296.1', '296.1', '173', '170', '270', '196.2', '296.3', '296.2', '290.4', '590.4', '196.1', '296.1', '196', '296.2']

    \end{Verbatim}

    \subsubsection{5. {[}5 puntos{]}}\label{puntos}

\emph{ob3} es un string que reune una lista de publicaciones. Use
expresiones regulares para extraer todos los \emph{Journals} en los
cuales el autor ha publicado. Ejemplo: El paper \textbf{Bail, CA. "The
configuration of symbolic boundaries against immigrants in Europe."
American Sociological Review 73.1 (January 1, 2008): 37-59. Full Text}
fue publicado en el Journal \emph{American Sociological Review}

ob3 = "Bail, CA, Argyle, LP, Brown, TW, Bumpus, JP, Chen, H, Hunzaker,
MBF, Lee, J, Mann, M, Merhout, F, and Volfovsky, A. "Exposure to
opposing views on social media can increase political polarization."
Proceedings of the National Academy of Sciences of the United States of
America 115.37 (September 2018): 9216-9221. Full Text Open Access
Copy.\n", "Bail, CA, Merhout, F, and Ding, P. "Using Internet search
data to examine the relationship between anti-Muslim and pro-ISIS
sentiment in U.S. counties." Science Advances 4.6 (June 6, 2018):
eaao5948-null. Full Text Open Access Copy.\n", "Bail, CA, Brown, TW, and
Mann, M. "Channeling Hearts and Minds: Advocacy Organizations,
Cognitive-Emotional Currents, and Public Conversation." American
Sociological Review 82.6 (December 1, 2017): 1188-1213. Full Text.\n",
"Bail, CA. "Taming Big Data: Using App Technology to Study
Organizational Behavior on Social Media." Sociological Methods and
Research 46.2 (March 1, 2017): 189-217. Full Text.\n", "McDonnell, TE,
Bail, CA, and Tavory, I. "A Theory of Resonance." Sociological Theory
35.1 (March 1, 2017): 1-14. Full Text.\n", "Bail, CA. "Combining natural
language processing and network analysis to examine how advocacy
organizations stimulate conversation on social media." Proceedings of
the National Academy of Sciences of the United States of America 113.42
(October 2016): 11823-11828. Full Text.\n", "Bail, CA. "Emotional
Feedback and the Viral Spread of Social Media Messages About Autism
Spectrum Disorders." American journal of public health 106.7 (July
2016): 1173-1180. Full Text.\n", "Bail, CA. "The public life of secrets:
Deception, disclosure, and discursive framing in the policy process."
Sociological Theory 33.2 (January 1, 2015): 97-124. Full Text.\n",
"Bail, CA. "The cultural environment: Measuring culture with big data."
Theory and Society 43.3 (January 1, 2014): 465-524. Full Text.""

    \begin{Verbatim}[commandchars=\\\{\}]
{\color{incolor}In [{\color{incolor}57}]:} \PY{c+c1}{\PYZsh{}\PYZsh{} Variable ob3}
         \PY{n}{ob3} \PY{o}{=} \PY{l+s+s1}{\PYZsq{}}\PY{l+s+s1}{\PYZdq{}}\PY{l+s+s1}{Bail, CA, Argyle, LP, Brown, TW, Bumpus, JP, Chen, H, Hunzaker, MBF, Lee, J, Mann, M, Merhout, F, and Volfovsky, A. }\PY{l+s+s1}{\PYZdq{}}\PY{l+s+s1}{Exposure to opposing views on social media can increase political polarization.}\PY{l+s+s1}{\PYZdq{}}\PY{l+s+s1}{ Proceedings of the National Academy of Sciences of the United States of America 115.37 (September 2018): 9216\PYZhy{}9221. Full Text Open Access Copy.}\PY{l+s+se}{\PYZbs{}n}\PY{l+s+s1}{\PYZdq{}}\PY{l+s+s1}{, }\PY{l+s+s1}{\PYZdq{}}\PY{l+s+s1}{Bail, CA, Merhout, F, and Ding, P. }\PY{l+s+s1}{\PYZdq{}}\PY{l+s+s1}{Using Internet search data to examine the relationship between anti\PYZhy{}Muslim and pro\PYZhy{}ISIS sentiment in U.S. counties.}\PY{l+s+s1}{\PYZdq{}}\PY{l+s+s1}{ Science Advances 4.6 (June 6, 2018): eaao5948\PYZhy{}null. Full Text Open Access Copy.}\PY{l+s+se}{\PYZbs{}n}\PY{l+s+s1}{\PYZdq{}}\PY{l+s+s1}{, }\PY{l+s+s1}{\PYZdq{}}\PY{l+s+s1}{Bail, CA, Brown, TW, and Mann, M. }\PY{l+s+s1}{\PYZdq{}}\PY{l+s+s1}{Channeling Hearts and Minds: Advocacy Organizations, Cognitive\PYZhy{}Emotional Currents, and Public Conversation.}\PY{l+s+s1}{\PYZdq{}}\PY{l+s+s1}{ American Sociological Review 82.6 (December 1, 2017): 1188\PYZhy{}1213. Full Text.}\PY{l+s+se}{\PYZbs{}n}\PY{l+s+s1}{\PYZdq{}}\PY{l+s+s1}{, }\PY{l+s+s1}{\PYZdq{}}\PY{l+s+s1}{Bail, CA. }\PY{l+s+s1}{\PYZdq{}}\PY{l+s+s1}{Taming Big Data: Using App Technology to Study Organizational Behavior on Social Media.}\PY{l+s+s1}{\PYZdq{}}\PY{l+s+s1}{ Sociological Methods and Research 46.2 (March 1, 2017): 189\PYZhy{}217. Full Text.}\PY{l+s+se}{\PYZbs{}n}\PY{l+s+s1}{\PYZdq{}}\PY{l+s+s1}{, }\PY{l+s+s1}{\PYZdq{}}\PY{l+s+s1}{McDonnell, TE, Bail, CA, and Tavory, I. }\PY{l+s+s1}{\PYZdq{}}\PY{l+s+s1}{A Theory of Resonance.}\PY{l+s+s1}{\PYZdq{}}\PY{l+s+s1}{ Sociological Theory 35.1 (March 1, 2017): 1\PYZhy{}14. Full Text.}\PY{l+s+se}{\PYZbs{}n}\PY{l+s+s1}{\PYZdq{}}\PY{l+s+s1}{, }\PY{l+s+s1}{\PYZdq{}}\PY{l+s+s1}{Bail, CA. }\PY{l+s+s1}{\PYZdq{}}\PY{l+s+s1}{Combining natural language processing and network analysis to examine how advocacy organizations stimulate conversation on social media.}\PY{l+s+s1}{\PYZdq{}}\PY{l+s+s1}{ Proceedings of the National Academy of Sciences of the United States of America 113.42 (October 2016): 11823\PYZhy{}11828. Full Text.}\PY{l+s+se}{\PYZbs{}n}\PY{l+s+s1}{\PYZdq{}}\PY{l+s+s1}{, }\PY{l+s+s1}{\PYZdq{}}\PY{l+s+s1}{Bail, CA. }\PY{l+s+s1}{\PYZdq{}}\PY{l+s+s1}{Emotional Feedback and the Viral Spread of Social Media Messages About Autism Spectrum Disorders.}\PY{l+s+s1}{\PYZdq{}}\PY{l+s+s1}{ American journal of public health 106.7 (July 2016): 1173\PYZhy{}1180. Full Text.}\PY{l+s+se}{\PYZbs{}n}\PY{l+s+s1}{\PYZdq{}}\PY{l+s+s1}{, }\PY{l+s+s1}{\PYZdq{}}\PY{l+s+s1}{Bail, CA. }\PY{l+s+s1}{\PYZdq{}}\PY{l+s+s1}{The public life of secrets: Deception, disclosure, and discursive framing in the policy process.}\PY{l+s+s1}{\PYZdq{}}\PY{l+s+s1}{ Sociological Theory 33.2 (January 1, 2015): 97\PYZhy{}124. Full Text.}\PY{l+s+se}{\PYZbs{}n}\PY{l+s+s1}{\PYZdq{}}\PY{l+s+s1}{, }\PY{l+s+s1}{\PYZdq{}}\PY{l+s+s1}{Bail, CA. }\PY{l+s+s1}{\PYZdq{}}\PY{l+s+s1}{The cultural environment: Measuring culture with big data.}\PY{l+s+s1}{\PYZdq{}}\PY{l+s+s1}{ Theory and Society 43.3 (January 1, 2014): 465\PYZhy{}524. Full Text.}\PY{l+s+s1}{\PYZdq{}}\PY{l+s+s1}{\PYZsq{}} 
         \PY{c+c1}{\PYZsh{}\PYZsh{} Guardaremos todos los journals en esta lista}
         \PY{n}{journals} \PY{o}{=} \PY{p}{[}\PY{p}{]}
         
         \PY{c+c1}{\PYZsh{} Buscamos en el texto frases: palabra + fecha}
         \PY{n}{frases} \PY{o}{=} \PY{n}{re}\PY{o}{.}\PY{n}{findall}\PY{p}{(}\PY{l+s+s1}{\PYZsq{}}\PY{l+s+s1}{\PYZdq{}}\PY{l+s+s1}{ }\PY{l+s+s1}{\PYZbs{}}\PY{l+s+s1}{w+ .+ }\PY{l+s+s1}{\PYZbs{}}\PY{l+s+s1}{d}\PY{l+s+s1}{\PYZsq{}}\PY{p}{,}\PY{n}{ob3}\PY{p}{)}
         
         \PY{k}{for} \PY{n}{i} \PY{o+ow}{in} \PY{n+nb}{range}\PY{p}{(}\PY{l+m+mi}{0}\PY{p}{,}\PY{n+nb}{len}\PY{p}{(}\PY{n}{frases}\PY{p}{)}\PY{p}{)}\PY{p}{:}
             \PY{n}{jour} \PY{o}{=} \PY{n}{re}\PY{o}{.}\PY{n}{search}\PY{p}{(}\PY{l+s+s1}{\PYZsq{}}\PY{l+s+s1}{\PYZbs{}}\PY{l+s+s1}{w+}\PY{l+s+s1}{\PYZbs{}}\PY{l+s+s1}{D+}\PY{l+s+s1}{\PYZsq{}}\PY{p}{,}\PY{n}{frases}\PY{p}{[}\PY{n}{i}\PY{p}{]}\PY{p}{)}\PY{o}{.}\PY{n}{group}\PY{p}{(}\PY{p}{)}
             \PY{n}{journals}\PY{o}{.}\PY{n}{append}\PY{p}{(}\PY{n}{jour}\PY{p}{)}
         \PY{c+c1}{\PYZsh{} Modificamos el primer journal}
         \PY{n}{journals}\PY{p}{[}\PY{l+m+mi}{0}\PY{p}{]} \PY{o}{=} \PY{l+s+s1}{\PYZsq{}}\PY{l+s+s1}{National Academy of Sciences of the United States of America}\PY{l+s+s1}{\PYZsq{}}
         \PY{n}{borrar}\PY{o}{=}\PY{p}{[}\PY{p}{]}
         \PY{k}{for} \PY{n}{i} \PY{o+ow}{in} \PY{n+nb}{range}\PY{p}{(}\PY{l+m+mi}{0}\PY{p}{,}\PY{n+nb}{len}\PY{p}{(}\PY{n}{journals}\PY{p}{)}\PY{p}{)}\PY{p}{:}
             \PY{k}{if} \PY{n}{journals}\PY{p}{[}\PY{n}{i}\PY{p}{]}\PY{o}{==}\PY{l+s+s1}{\PYZsq{}}\PY{l+s+s1}{Sociological Theory }\PY{l+s+s1}{\PYZsq{}}\PY{p}{:}
                  \PY{n}{borrar}\PY{o}{.}\PY{n}{append}\PY{p}{(}\PY{n}{i}\PY{p}{)}
         \PY{k}{del} \PY{n}{journals}\PY{p}{[}\PY{n}{borrar}\PY{p}{[}\PY{l+m+mi}{0}\PY{p}{]}\PY{p}{]}
         \PY{k}{del} \PY{n}{journals}\PY{p}{[}\PY{n}{borrar}\PY{p}{[}\PY{l+m+mi}{1}\PY{p}{]}\PY{o}{\PYZhy{}}\PY{l+m+mi}{1}\PY{p}{]}
         
         \PY{n+nb}{print}\PY{p}{(}\PY{l+s+s2}{\PYZdq{}}\PY{l+s+s2}{Journals: }\PY{l+s+s2}{\PYZdq{}}\PY{p}{,}\PY{n}{journals}\PY{p}{)}
\end{Verbatim}


    \begin{Verbatim}[commandchars=\\\{\}]
Journals:  ['National Academy of Sciences of the United States of America', 'Science Advances ', 'American Sociological Review ', 'Sociological Methods and Research ', 'Proceedings of the National Academy of Sciences of the United States of America ', 'American journal of public health ', 'Theory and Society ']

    \end{Verbatim}

    \subsubsection{6. {[}10 puntos{]}}\label{puntos}

Vamos a hacer "scraping" a esta página:
https://archive.ics.uci.edu/ml/datasets.php, que contiene un listado de
468 bases de datos que hacen parte del repositorio de la Universidad de
California, Irvine.

    Su tarea consiste en crear un "Pandas dataframe" que contenga 468 filas
(una por base de datos) y las siguientes columnas: - Nombre de la base
de datos - Link a la base de datos - Tipo de datos - Tipo de tarea a
resolver (default task) - Tipo de las variables - Número de
observaciones - Número de variables - Año - Descripción de la base
(Pista: Utilice la opción list view:
https://archive.ics.uci.edu/ml/datasets.php?format=\&task=\&att=\&area=\&numAtt=\&numIns=\&type=\&sort=nameUp\&view=list)

    \begin{Verbatim}[commandchars=\\\{\}]
{\color{incolor}In [{\color{incolor}94}]:} \PY{k+kn}{import} \PY{n+nn}{requests}
         \PY{k+kn}{from} \PY{n+nn}{bs4} \PY{k}{import} \PY{n}{BeautifulSoup}
         \PY{k+kn}{import} \PY{n+nn}{pandas} \PY{k}{as} \PY{n+nn}{pd}
\end{Verbatim}


    \begin{Verbatim}[commandchars=\\\{\}]
{\color{incolor}In [{\color{incolor}60}]:} \PY{c+c1}{\PYZsh{} Info pag}
         \PY{n}{link} \PY{o}{=} \PY{n}{requests}\PY{o}{.}\PY{n}{get}\PY{p}{(}\PY{l+s+s1}{\PYZsq{}}\PY{l+s+s1}{https://archive.ics.uci.edu/ml/datasets.php}\PY{l+s+s1}{\PYZsq{}}\PY{p}{)}\PY{o}{.}\PY{n}{text}
         \PY{n}{l} \PY{o}{=} \PY{n}{BeautifulSoup}\PY{p}{(}\PY{n}{link}\PY{p}{,} \PY{l+s+s2}{\PYZdq{}}\PY{l+s+s2}{lxml}\PY{l+s+s2}{\PYZdq{}}\PY{p}{)}
\end{Verbatim}


    \begin{Verbatim}[commandchars=\\\{\}]
{\color{incolor}In [{\color{incolor}85}]:} \PY{c+c1}{\PYZsh{}\PYZsh{}\PYZsh{} Buscamos primero todas las lineas que inician con \PYZlt{}b\PYZgt{}}
         \PY{n}{lineas} \PY{o}{=} \PY{n}{l}\PY{o}{.}\PY{n}{find\PYZus{}all}\PY{p}{(}\PY{l+s+s1}{\PYZsq{}}\PY{l+s+s1}{b}\PY{l+s+s1}{\PYZsq{}}\PY{p}{)}
         \PY{n}{str\PYZus{}lineas}\PY{o}{=}\PY{n+nb}{str}\PY{p}{(}\PY{n}{lineas}\PY{p}{)}
         
         \PY{c+c1}{\PYZsh{} En esas lineas buscamos unas con una caracteristica especifica}
         \PY{n}{especificas}\PY{o}{=} \PY{n}{re}\PY{o}{.}\PY{n}{findall}\PY{p}{(}\PY{l+s+s1}{\PYZsq{}}\PY{l+s+s1}{(\PYZlt{}b\PYZgt{}\PYZlt{}a href=}\PY{l+s+s1}{\PYZdq{}}\PY{l+s+s1}{datasets/}\PY{l+s+s1}{\PYZbs{}}\PY{l+s+s1}{w+.+?\PYZlt{}/a\PYZgt{})}\PY{l+s+s1}{\PYZsq{}}\PY{p}{,} \PY{n}{str\PYZus{}lineas}\PY{p}{)}
         
         
         \PY{c+c1}{\PYZsh{} Crearemos una lista con todos los nombres}
         \PY{n}{nombres} \PY{o}{=} \PY{p}{[}\PY{p}{]}
         \PY{k}{for} \PY{n}{i} \PY{o+ow}{in} \PY{n+nb}{range}\PY{p}{(}\PY{l+m+mi}{0}\PY{p}{,}\PY{n+nb}{len}\PY{p}{(}\PY{n}{especificas}\PY{p}{)}\PY{p}{)}\PY{p}{:}
             \PY{n}{nom} \PY{o}{=} \PY{n}{re}\PY{o}{.}\PY{n}{search}\PY{p}{(}\PY{l+s+s1}{\PYZsq{}}\PY{l+s+s1}{\PYZgt{}(}\PY{l+s+s1}{\PYZbs{}}\PY{l+s+s1}{w+.+)[\PYZca{}\PYZlt{}]}\PY{l+s+s1}{\PYZsq{}}\PY{p}{,}\PY{n}{especificas}\PY{p}{[}\PY{n}{i}\PY{p}{]}\PY{p}{)}\PY{o}{.}\PY{n}{group}\PY{p}{(}\PY{p}{)}
             \PY{n}{nombres}\PY{o}{.}\PY{n}{append}\PY{p}{(}\PY{n}{nom}\PY{p}{)}
         \PY{c+c1}{\PYZsh{} Ahora de los nombres los quitamos el \PYZlt{}/a\PYZgt{}}
         \PY{k}{for} \PY{n}{i} \PY{o+ow}{in} \PY{n+nb}{range}\PY{p}{(}\PY{l+m+mi}{0}\PY{p}{,}\PY{n+nb}{len}\PY{p}{(}\PY{n}{especificas}\PY{p}{)}\PY{p}{)}\PY{p}{:}
             \PY{n}{nom} \PY{o}{=} \PY{n}{re}\PY{o}{.}\PY{n}{search}\PY{p}{(}\PY{l+s+s1}{\PYZsq{}}\PY{l+s+s1}{[\PYZca{}\PYZgt{}].+[\PYZca{}\PYZgt{}\PYZlt{}/a\PYZgt{}]}\PY{l+s+s1}{\PYZsq{}}\PY{p}{,}\PY{n}{nombres}\PY{p}{[}\PY{n}{i}\PY{p}{]}\PY{p}{)}\PY{o}{.}\PY{n}{group}\PY{p}{(}\PY{p}{)}    
             \PY{n}{nombres}\PY{p}{[}\PY{n}{i}\PY{p}{]}\PY{o}{=}\PY{n}{nom}
             
         \PY{c+c1}{\PYZsh{}print(\PYZdq{}Variables: \PYZdq{}, nombres)}
\end{Verbatim}


    \begin{Verbatim}[commandchars=\\\{\}]
{\color{incolor}In [{\color{incolor}84}]:} \PY{c+c1}{\PYZsh{}\PYZsh{} datos caracteristas}
         \PY{c+c1}{\PYZsh{} lineas que inician con \PYZlt{}p\PYZgt{}}
         \PY{n}{carac} \PY{o}{=} \PY{n}{l}\PY{o}{.}\PY{n}{find\PYZus{}all}\PY{p}{(}\PY{l+s+s1}{\PYZsq{}}\PY{l+s+s1}{p}\PY{l+s+s1}{\PYZsq{}}\PY{p}{)} 
         \PY{c+c1}{\PYZsh{} Caracteristica de los datos, falta el ultimo}
         \PY{n}{caracteristicas}\PY{o}{=}\PY{n}{re}\PY{o}{.}\PY{n}{findall}\PY{p}{(}\PY{l+s+s1}{\PYZsq{}}\PY{l+s+s1}{(href=}\PY{l+s+s1}{\PYZdq{}}\PY{l+s+s1}{datasets/.+?\PYZlt{}b\PYZgt{})}\PY{l+s+s1}{\PYZsq{}}\PY{p}{,}\PY{n+nb}{str}\PY{p}{(}\PY{n}{carac}\PY{p}{)}\PY{p}{)}\PY{o}{+}\PY{p}{[}\PY{l+s+s1}{\PYZsq{}}\PY{l+s+s1}{href=}\PY{l+s+s1}{\PYZdq{}}\PY{l+s+s1}{datasets/Rice+Leaf+Diseases}\PY{l+s+s1}{\PYZdq{}}\PY{l+s+s1}{\PYZgt{}Rice Leaf Diseases\PYZlt{}/a\PYZgt{}\PYZlt{}/b\PYZgt{}\PYZlt{}/p\PYZgt{}, \PYZlt{}p class=}\PY{l+s+s1}{\PYZdq{}}\PY{l+s+s1}{normal}\PY{l+s+s1}{\PYZdq{}}\PY{l+s+s1}{\PYZgt{}Multivariate}\PY{l+s+se}{\PYZbs{}xa0}\PY{l+s+s1}{\PYZlt{}/p\PYZgt{}, \PYZlt{}p class=}\PY{l+s+s1}{\PYZdq{}}\PY{l+s+s1}{normal}\PY{l+s+s1}{\PYZdq{}}\PY{l+s+s1}{\PYZgt{}Classification}\PY{l+s+se}{\PYZbs{}xa0}\PY{l+s+s1}{\PYZlt{}/p\PYZgt{}, \PYZlt{}p class=}\PY{l+s+s1}{\PYZdq{}}\PY{l+s+s1}{normal}\PY{l+s+s1}{\PYZdq{}}\PY{l+s+s1}{\PYZgt{}Integer}\PY{l+s+se}{\PYZbs{}xa0}\PY{l+s+s1}{\PYZlt{}/p\PYZgt{}, \PYZlt{}p class=}\PY{l+s+s1}{\PYZdq{}}\PY{l+s+s1}{normal}\PY{l+s+s1}{\PYZdq{}}\PY{l+s+s1}{\PYZgt{}120}\PY{l+s+se}{\PYZbs{}xa0}\PY{l+s+s1}{\PYZlt{}/p\PYZgt{}, \PYZlt{}p class=}\PY{l+s+s1}{\PYZdq{}}\PY{l+s+s1}{normal}\PY{l+s+s1}{\PYZdq{}}\PY{l+s+s1}{\PYZgt{}}\PY{l+s+se}{\PYZbs{}xa0}\PY{l+s+s1}{\PYZlt{}/p\PYZgt{}, \PYZlt{}p class=}\PY{l+s+s1}{\PYZdq{}}\PY{l+s+s1}{normal}\PY{l+s+s1}{\PYZdq{}}\PY{l+s+s1}{\PYZgt{}2019}\PY{l+s+se}{\PYZbs{}xa0}\PY{l+s+s1}{\PYZlt{}/p\PYZgt{}}\PY{l+s+s1}{\PYZsq{}}\PY{p}{]}
         
         \PY{c+c1}{\PYZsh{} Tipos de datos}
         \PY{c+c1}{\PYZsh{}\PYZsh{} Guardare todo en una lista de listas}
         
         \PY{n}{tipos}\PY{o}{=}\PY{p}{[}\PY{p}{[}\PY{p}{]}\PY{p}{,}\PY{p}{[}\PY{p}{]}\PY{p}{,}\PY{p}{[}\PY{p}{]}\PY{p}{,}\PY{p}{[}\PY{p}{]}\PY{p}{,}\PY{p}{[}\PY{p}{]}\PY{p}{,}\PY{p}{[}\PY{p}{]}\PY{p}{]}
         \PY{k}{for} \PY{n}{i} \PY{o+ow}{in} \PY{n+nb}{range}\PY{p}{(}\PY{l+m+mi}{0}\PY{p}{,}\PY{n+nb}{len}\PY{p}{(}\PY{n}{especificas}\PY{p}{)}\PY{p}{)}\PY{p}{:}
             \PY{n}{carac\PYZus{}es} \PY{o}{=} \PY{n}{re}\PY{o}{.}\PY{n}{search}\PY{p}{(}\PY{l+s+s1}{\PYZsq{}}\PY{l+s+s1}{\PYZlt{}p.+}\PY{l+s+s1}{\PYZsq{}}\PY{p}{,}\PY{n}{caracteristicas}\PY{p}{[}\PY{n}{i}\PY{p}{]}\PY{p}{)}\PY{o}{.}\PY{n}{group}\PY{p}{(}\PY{p}{)}
             \PY{n}{tipo} \PY{o}{=}\PY{n}{re}\PY{o}{.}\PY{n}{findall}\PY{p}{(}\PY{l+s+s1}{\PYZsq{}}\PY{l+s+s1}{(\PYZlt{}p.+?\PYZlt{}/p\PYZgt{})}\PY{l+s+s1}{\PYZsq{}}\PY{p}{,}\PY{n}{carac\PYZus{}es}\PY{p}{)}
             \PY{k}{for} \PY{n}{j} \PY{o+ow}{in} \PY{n+nb}{range}\PY{p}{(}\PY{l+m+mi}{0}\PY{p}{,}\PY{l+m+mi}{6}\PY{p}{)}\PY{p}{:}
                 \PY{n}{pal1}\PY{o}{=}\PY{n}{re}\PY{o}{.}\PY{n}{search}\PY{p}{(}\PY{l+s+s1}{\PYZsq{}}\PY{l+s+s1}{(\PYZgt{}.+?}\PY{l+s+se}{\PYZbs{}xa0}\PY{l+s+s1}{)|(}\PY{l+s+se}{\PYZbs{}xa0}\PY{l+s+s1}{)}\PY{l+s+s1}{\PYZsq{}}\PY{p}{,}\PY{n}{tipo}\PY{p}{[}\PY{n}{j}\PY{p}{]}\PY{p}{)}\PY{o}{.}\PY{n}{group}\PY{p}{(}\PY{p}{)}
                 \PY{n}{pal2}\PY{o}{=}\PY{n}{re}\PY{o}{.}\PY{n}{search}\PY{p}{(}\PY{l+s+s1}{\PYZsq{}}\PY{l+s+s1}{[\PYZca{}\PYZgt{}].+[\PYZca{}}\PY{l+s+se}{\PYZbs{}xa0}\PY{l+s+s1}{]|}\PY{l+s+s1}{\PYZbs{}}\PY{l+s+s1}{w+|}\PY{l+s+se}{\PYZbs{}xa0}\PY{l+s+s1}{\PYZsq{}}\PY{p}{,}\PY{n}{pal1}\PY{p}{)}\PY{o}{.}\PY{n}{group}\PY{p}{(}\PY{p}{)}
                 \PY{n}{tipos}\PY{p}{[}\PY{n}{j}\PY{p}{]}\PY{o}{.}\PY{n}{append}\PY{p}{(}\PY{n}{pal2}\PY{p}{)} 
         \PY{c+c1}{\PYZsh{} \PYZbs{}xa0 indica missing}
         \PY{c+c1}{\PYZsh{} Cambiados por NA}
         \PY{k}{for} \PY{n}{j} \PY{o+ow}{in} \PY{n}{tipos}\PY{p}{:}
             \PY{k}{for} \PY{n}{i} \PY{o+ow}{in} \PY{n+nb}{range}\PY{p}{(}\PY{l+m+mi}{0}\PY{p}{,}\PY{n+nb}{len}\PY{p}{(}\PY{n}{especificas}\PY{p}{)}\PY{p}{)}\PY{p}{:}
                 \PY{k}{if} \PY{n}{j}\PY{p}{[}\PY{n}{i}\PY{p}{]} \PY{o}{==} \PY{l+s+s1}{\PYZsq{}}\PY{l+s+se}{\PYZbs{}xa0}\PY{l+s+s1}{\PYZsq{}}\PY{p}{:}
                     \PY{n}{j}\PY{p}{[}\PY{n}{i}\PY{p}{]}\PY{o}{=} \PY{l+s+s1}{\PYZsq{}}\PY{l+s+s1}{NA}\PY{l+s+s1}{\PYZsq{}}
\end{Verbatim}


    \begin{Verbatim}[commandchars=\\\{\}]
{\color{incolor}In [{\color{incolor}90}]:} \PY{c+c1}{\PYZsh{} LINKS}
         
         \PY{n}{l1} \PY{o}{=} \PY{p}{[}\PY{p}{]}
         \PY{k}{for} \PY{n}{i} \PY{o+ow}{in} \PY{n+nb}{range}\PY{p}{(}\PY{l+m+mi}{0}\PY{p}{,}\PY{n+nb}{len}\PY{p}{(}\PY{n}{especificas}\PY{p}{)}\PY{p}{)}\PY{p}{:}
             \PY{n}{pal1} \PY{o}{=} \PY{n}{re}\PY{o}{.}\PY{n}{search}\PY{p}{(}\PY{l+s+s1}{\PYZsq{}}\PY{l+s+s1}{\PYZdq{}}\PY{l+s+s1}{(datasets/.+?)}\PY{l+s+s1}{\PYZdq{}}\PY{l+s+s1}{\PYZsq{}}\PY{p}{,}\PY{n}{especificas}\PY{p}{[}\PY{n}{i}\PY{p}{]}\PY{p}{)}\PY{o}{.}\PY{n}{group}\PY{p}{(}\PY{p}{)}
             \PY{n}{pal2} \PY{o}{=} \PY{n}{re}\PY{o}{.}\PY{n}{search}\PY{p}{(}\PY{l+s+s1}{\PYZsq{}}\PY{l+s+s1}{[\PYZca{}}\PY{l+s+s1}{\PYZdq{}}\PY{l+s+s1}{].+[\PYZca{}}\PY{l+s+s1}{\PYZdq{}}\PY{l+s+s1}{]}\PY{l+s+s1}{\PYZsq{}}\PY{p}{,}\PY{n}{pal1}\PY{p}{)}\PY{o}{.}\PY{n}{group}\PY{p}{(}\PY{p}{)}
             \PY{n}{l1}\PY{o}{.}\PY{n}{append}\PY{p}{(}\PY{n}{pal2}\PY{p}{)}
         \PY{n}{l2} \PY{o}{=} \PY{p}{[}\PY{p}{]}
         \PY{k}{for} \PY{n}{i} \PY{o+ow}{in} \PY{n+nb}{range}\PY{p}{(}\PY{l+m+mi}{0}\PY{p}{,}\PY{n+nb}{len}\PY{p}{(}\PY{n}{l1}\PY{p}{)}\PY{p}{)}\PY{p}{:}
             \PY{n}{pal} \PY{o}{=} \PY{l+s+s1}{\PYZsq{}}\PY{l+s+s1}{https://archive.ics.uci.edu/ml/}\PY{l+s+s1}{\PYZsq{}}\PY{o}{+}\PY{n}{l1}\PY{p}{[}\PY{n}{i}\PY{p}{]}
             \PY{n}{l2}\PY{o}{.}\PY{n}{append}\PY{p}{(}\PY{n}{pal}\PY{p}{)}
             
         \PY{c+c1}{\PYZsh{} Links parte III}
         \PY{n}{l3} \PY{o}{=} \PY{p}{[}\PY{p}{]}
         \PY{k}{for} \PY{n}{i} \PY{o+ow}{in} \PY{n+nb}{range}\PY{p}{(}\PY{l+m+mi}{0}\PY{p}{,}\PY{n+nb}{len}\PY{p}{(}\PY{n}{l1}\PY{p}{)}\PY{p}{)}\PY{p}{:}
             \PY{n}{html} \PY{o}{=} \PY{n}{requests}\PY{o}{.}\PY{n}{get}\PY{p}{(}\PY{n}{l2}\PY{p}{[}\PY{n}{i}\PY{p}{]}\PY{p}{)}\PY{o}{.}\PY{n}{text}
             \PY{n}{s\PYZus{}html} \PY{o}{=} \PY{n}{BeautifulSoup}\PY{p}{(}\PY{n}{html}\PY{p}{,} \PY{l+s+s2}{\PYZdq{}}\PY{l+s+s2}{lxml}\PY{l+s+s2}{\PYZdq{}}\PY{p}{)}
             \PY{n}{pal\PYZus{}a} \PY{o}{=} \PY{n}{s\PYZus{}html}\PY{o}{.}\PY{n}{find\PYZus{}all}\PY{p}{(}\PY{l+s+s1}{\PYZsq{}}\PY{l+s+s1}{a}\PY{l+s+s1}{\PYZsq{}}\PY{p}{)} 
             \PY{n}{html\PYZus{}nue} \PY{o}{=} \PY{n}{re}\PY{o}{.}\PY{n}{findall}\PY{p}{(}\PY{l+s+s1}{\PYZsq{}}\PY{l+s+s1}{\PYZlt{}a( href=}\PY{l+s+s1}{\PYZdq{}}\PY{l+s+s1}{../machine\PYZhy{}learning\PYZhy{}databases/.+?}\PY{l+s+s1}{\PYZdq{}}\PY{l+s+s1}{\PYZgt{})}\PY{l+s+s1}{\PYZsq{}}\PY{p}{,}\PY{n+nb}{str}\PY{p}{(}\PY{n}{pal\PYZus{}a}\PY{p}{)}\PY{p}{)}
             \PY{k}{if} \PY{n}{html\PYZus{}nue} \PY{o}{==} \PY{p}{[}\PY{p}{]}\PY{p}{:}
                 \PY{n}{l3}\PY{o}{.}\PY{n}{append}\PY{p}{(}\PY{l+s+s1}{\PYZsq{}}\PY{l+s+s1}{NA}\PY{l+s+s1}{\PYZsq{}}\PY{p}{)}
             \PY{k}{else}\PY{p}{:}
                 \PY{n}{l3}\PY{o}{.}\PY{n}{append}\PY{p}{(}\PY{n}{html\PYZus{}nue}\PY{p}{[}\PY{l+m+mi}{0}\PY{p}{]}\PY{p}{)}
\end{Verbatim}


    \begin{Verbatim}[commandchars=\\\{\}]
{\color{incolor}In [{\color{incolor}92}]:} \PY{n}{l\PYZus{}def} \PY{o}{=} \PY{p}{[}\PY{p}{]}
         \PY{k}{for} \PY{n}{i} \PY{o+ow}{in} \PY{n+nb}{range}\PY{p}{(}\PY{l+m+mi}{0}\PY{p}{,}\PY{n+nb}{len}\PY{p}{(}\PY{n}{especificas}\PY{p}{)}\PY{p}{)}\PY{p}{:}
             \PY{k}{if} \PY{n}{l3}\PY{p}{[}\PY{n}{i}\PY{p}{]} \PY{o}{==} \PY{l+s+s1}{\PYZsq{}}\PY{l+s+s1}{NA}\PY{l+s+s1}{\PYZsq{}}\PY{p}{:}
                 \PY{n}{l\PYZus{}def}\PY{o}{.}\PY{n}{append}\PY{p}{(}\PY{l+s+s1}{\PYZsq{}}\PY{l+s+s1}{NA}\PY{l+s+s1}{\PYZsq{}}\PY{p}{)}
             \PY{k}{else}\PY{p}{:}
                 \PY{n}{pal1} \PY{o}{=} \PY{n}{re}\PY{o}{.}\PY{n}{search}\PY{p}{(}\PY{l+s+s1}{\PYZsq{}}\PY{l+s+s1}{/.+[\PYZca{}}\PY{l+s+s1}{\PYZdq{}}\PY{l+s+s1}{\PYZgt{}]}\PY{l+s+s1}{\PYZsq{}}\PY{p}{,}\PY{n}{l3}\PY{p}{[}\PY{n}{i}\PY{p}{]}\PY{p}{)}\PY{o}{.}\PY{n}{group}\PY{p}{(}\PY{p}{)}
                 \PY{n}{pal2} \PY{o}{=} \PY{l+s+s1}{\PYZsq{}}\PY{l+s+s1}{https://archive.ics.uci.edu/ml}\PY{l+s+s1}{\PYZsq{}}\PY{o}{+}\PY{n}{pal1}
                 \PY{n}{l\PYZus{}def}\PY{o}{.}\PY{n}{append}\PY{p}{(}\PY{n}{pal2}\PY{p}{)}
\end{Verbatim}


    \begin{Verbatim}[commandchars=\\\{\}]
{\color{incolor}In [{\color{incolor}99}]:} \PY{n}{tabla} \PY{o}{=} \PY{p}{\PYZob{}}\PY{l+s+s2}{\PYZdq{}}\PY{l+s+s2}{Base de datos}\PY{l+s+s2}{\PYZdq{}}\PY{p}{:} \PY{n}{nombres}\PY{p}{,}
                 \PY{l+s+s2}{\PYZdq{}}\PY{l+s+s2}{Tipo de datos}\PY{l+s+s2}{\PYZdq{}}\PY{p}{:} \PY{n}{tipos}\PY{p}{[}\PY{l+m+mi}{0}\PY{p}{]}\PY{p}{,}
                 \PY{l+s+s2}{\PYZdq{}}\PY{l+s+s2}{Tipo de tarea a resolver}\PY{l+s+s2}{\PYZdq{}}\PY{p}{:} \PY{n}{tipos}\PY{p}{[}\PY{l+m+mi}{1}\PY{p}{]}\PY{p}{,}
                 \PY{l+s+s2}{\PYZdq{}}\PY{l+s+s2}{Tipo de variables}\PY{l+s+s2}{\PYZdq{}}\PY{p}{:} \PY{n}{tipos}\PY{p}{[}\PY{l+m+mi}{2}\PY{p}{]}\PY{p}{,}
                 \PY{l+s+s2}{\PYZdq{}}\PY{l+s+s2}{Número de obs}\PY{l+s+s2}{\PYZdq{}}\PY{p}{:} \PY{n}{tipos}\PY{p}{[}\PY{l+m+mi}{3}\PY{p}{]}\PY{p}{,}
                 \PY{l+s+s2}{\PYZdq{}}\PY{l+s+s2}{Número de variables}\PY{l+s+s2}{\PYZdq{}}\PY{p}{:} \PY{n}{tipos}\PY{p}{[}\PY{l+m+mi}{4}\PY{p}{]}\PY{p}{,}
                 \PY{l+s+s2}{\PYZdq{}}\PY{l+s+s2}{Año}\PY{l+s+s2}{\PYZdq{}}\PY{p}{:} \PY{n}{tipos}\PY{p}{[}\PY{l+m+mi}{5}\PY{p}{]}\PY{p}{,}
                 \PY{l+s+s2}{\PYZdq{}}\PY{l+s+s2}{Links}\PY{l+s+s2}{\PYZdq{}}\PY{p}{:} \PY{n}{l\PYZus{}def}\PY{p}{\PYZcb{}}
         \PY{n}{definitiva} \PY{o}{=} \PY{n}{pd}\PY{o}{.}\PY{n}{DataFrame}\PY{p}{(}\PY{n}{tabla}\PY{p}{)}
         
         \PY{c+c1}{\PYZsh{} Ordenando el data frame}
         
         \PY{n}{definitiva}\PY{p}{[}\PY{l+s+s2}{\PYZdq{}}\PY{l+s+s2}{nombres mayu}\PY{l+s+s2}{\PYZdq{}}\PY{p}{]} \PY{o}{=} \PY{n}{definitiva}\PY{p}{[}\PY{l+s+s2}{\PYZdq{}}\PY{l+s+s2}{Base de datos}\PY{l+s+s2}{\PYZdq{}}\PY{p}{]}\PY{o}{.}\PY{n}{str}\PY{o}{.}\PY{n}{upper}\PY{p}{(}\PY{p}{)}
         \PY{n}{definitiva}\PY{o}{.}\PY{n}{sort\PYZus{}values}\PY{p}{(}\PY{p}{[}\PY{l+s+s2}{\PYZdq{}}\PY{l+s+s2}{nombres mayu}\PY{l+s+s2}{\PYZdq{}}\PY{p}{]}\PY{p}{,} \PY{n}{axis}\PY{o}{=}\PY{l+m+mi}{0}\PY{p}{,} 
                          \PY{n}{ascending}\PY{o}{=}\PY{p}{[}\PY{k+kc}{True}\PY{p}{]}\PY{p}{,} \PY{n}{inplace}\PY{o}{=}\PY{k+kc}{True}\PY{p}{)} 
         \PY{c+c1}{\PYZsh{} Borramos la variable creada del data frame}
         \PY{k}{del} \PY{n}{definitiva}\PY{p}{[}\PY{l+s+s2}{\PYZdq{}}\PY{l+s+s2}{nombres mayu}\PY{l+s+s2}{\PYZdq{}}\PY{p}{]}   
\end{Verbatim}


    \begin{Verbatim}[commandchars=\\\{\}]
{\color{incolor}In [{\color{incolor}100}]:} \PY{c+c1}{\PYZsh{} Descripción de la base}
          \PY{n}{link} \PY{o}{=} \PY{n}{requests}\PY{o}{.}\PY{n}{get}\PY{p}{(}\PY{l+s+s1}{\PYZsq{}}\PY{l+s+s1}{https://archive.ics.uci.edu/ml/datasets.php?format=\PYZam{}task=\PYZam{}att=\PYZam{}area=\PYZam{}numAtt=\PYZam{}numIns=\PYZam{}type=\PYZam{}sort=nameUp\PYZam{}view=list}\PY{l+s+s1}{\PYZsq{}}\PY{p}{)}\PY{o}{.}\PY{n}{text}
          
          \PY{n}{s\PYZus{}link} \PY{o}{=} \PY{n}{BeautifulSoup}\PY{p}{(}\PY{n}{link}\PY{p}{,} \PY{l+s+s2}{\PYZdq{}}\PY{l+s+s2}{lxml}\PY{l+s+s2}{\PYZdq{}}\PY{p}{)}
          
          \PY{n}{des} \PY{o}{=} \PY{n}{re}\PY{o}{.}\PY{n}{findall}\PY{p}{(}\PY{l+s+s1}{\PYZsq{}}\PY{l+s+s1}{\PYZlt{}b\PYZgt{}\PYZlt{}a href=}\PY{l+s+s1}{\PYZdq{}}\PY{l+s+s1}{datasets[}\PY{l+s+s1}{\PYZbs{}}\PY{l+s+s1}{s}\PY{l+s+s1}{\PYZbs{}}\PY{l+s+s1}{S]*?\PYZlt{}/p\PYZgt{}}\PY{l+s+s1}{\PYZsq{}}\PY{p}{,}\PY{n+nb}{str}\PY{p}{(}\PY{n}{s\PYZus{}link}\PY{p}{)}\PY{p}{)}
          
          \PY{n}{descripcion} \PY{o}{=} \PY{p}{[}\PY{p}{]}
          \PY{k}{for} \PY{n}{i} \PY{o+ow}{in} \PY{n+nb}{range}\PY{p}{(}\PY{l+m+mi}{0}\PY{p}{,}\PY{n+nb}{len}\PY{p}{(}\PY{n}{des}\PY{p}{)}\PY{p}{)}\PY{p}{:}
              \PY{k}{try}\PY{p}{:}
                  \PY{n}{pal1} \PY{o}{=} \PY{n}{re}\PY{o}{.}\PY{n}{search}\PY{p}{(}\PY{l+s+s1}{\PYZsq{}}\PY{l+s+s1}{\PYZlt{}/b\PYZgt{}([}\PY{l+s+s1}{\PYZbs{}}\PY{l+s+s1}{s}\PY{l+s+s1}{\PYZbs{}}\PY{l+s+s1}{S]*?)\PYZlt{}/p\PYZgt{}}\PY{l+s+s1}{\PYZsq{}}\PY{p}{,}\PY{n}{des}\PY{p}{[}\PY{n}{i}\PY{p}{]}\PY{p}{)}\PY{o}{.}\PY{n}{group}\PY{p}{(}\PY{p}{)}
                  \PY{n}{pal2} \PY{o}{=} \PY{n}{re}\PY{o}{.}\PY{n}{search}\PY{p}{(}\PY{l+s+s1}{\PYZsq{}}\PY{l+s+s1}{[\PYZca{}:\PYZlt{}/b\PYZgt{} ][}\PY{l+s+s1}{\PYZbs{}}\PY{l+s+s1}{s}\PY{l+s+s1}{\PYZbs{}}\PY{l+s+s1}{S]*[\PYZca{}\PYZlt{}/p\PYZgt{} ]}\PY{l+s+s1}{\PYZsq{}}\PY{p}{,}\PY{n}{pal1}\PY{p}{)}\PY{o}{.}\PY{n}{group}\PY{p}{(}\PY{p}{)}
                  \PY{n}{descripcion}\PY{o}{.}\PY{n}{append}\PY{p}{(}\PY{n}{re}\PY{o}{.}\PY{n}{sub}\PY{p}{(}\PY{l+s+s1}{\PYZsq{}}\PY{l+s+se}{\PYZbs{}r}\PY{l+s+se}{\PYZbs{}n}\PY{l+s+s1}{\PYZsq{}}\PY{p}{,} \PY{l+s+s1}{\PYZsq{}}\PY{l+s+s1}{\PYZsq{}}\PY{p}{,}\PY{n}{pal2}\PY{p}{)}\PY{p}{)}
              \PY{k}{except} \PY{n+ne}{AttributeError}\PY{p}{:}
                  \PY{n}{descripcion}\PY{o}{.}\PY{n}{append}\PY{p}{(}\PY{l+s+s1}{\PYZsq{}}\PY{l+s+s1}{NA}\PY{l+s+s1}{\PYZsq{}}\PY{p}{)}
          
          \PY{n}{definitiva}\PY{p}{[}\PY{l+s+s1}{\PYZsq{}}\PY{l+s+s1}{Descripción base datos}\PY{l+s+s1}{\PYZsq{}}\PY{p}{]} \PY{o}{=} \PY{n}{descripcion}
          
          \PY{c+c1}{\PYZsh{} Resultado}
          \PY{n}{definitiva}
\end{Verbatim}


\begin{Verbatim}[commandchars=\\\{\}]
{\color{outcolor}Out[{\color{outcolor}100}]:}                                          Base de datos  \textbackslash{}
          462                2.4 GHZ Indoor Channel Measurements   
          237           3D Road Network (North Jutland, Denmark)   
          301                          AAAI 2013 Accepted Papers   
          294                          AAAI 2014 Accepted Papers   
          0                                              Abalone   
          170                    Abscisic Acid Signaling Network   
          427                                Absenteeism at work   
          260  Activities of Daily Living (ADLs) Recognition {\ldots}   
          275  Activity Recognition from Single Chest-Mounted{\ldots}   
          351  Activity Recognition system based on Multisens{\ldots}   
          411  Activity recognition with healthy older people{\ldots}   
          181                                Acute Inflammations   
          1                                                Adult   
          371                                        Air quality   
          345                                        Air Quality   
          279                                 Airfoil Self-Noise   
          476                         Alcohol QCM Sensor Dataset   
          208                              Amazon Access Samples   
          207                        Amazon Commerce reviews set   
          2                                            Annealing   
          3                          Anonymous Microsoft Web Dat   
          390                               Anuran Calls (MFCCs)   
          358                       Appliances energy prediction   
          405                       APS Failure at Scania Trucks   
          164                                             Arcene   
          4                                            Arrhythmi   
          5                                Artificial Characters   
          6                                 Audiology (Original)   
          7                             Audiology (Standardized)   
          457                                          Audit Dat   
          ..                                                 {\ldots}   
          248                            User Knowledge Modeling   
          257  USPTO Algorithm Challenge, run by NASA-Harvard{\ldots}   
          204                                   Vertebral Column   
          206                     Vicon Physical Action Data Set   
          436               Victorian Era Authorship Attribution   
          139            Volcanoes on Venus - JARtool experiment   
          190                Wall-Following Robot Navigation Dat   
          103                              Water Treatment Plant   
          474                             Wave Energy Converters   
          104            Waveform Database Generator (Version 1)   
          105            Waveform Database Generator (Version 2)   
          241  Wearable Computing: Classification of Body Pos{\ldots}   
          363                                   Website Phishing   
          262  Weight Lifting Exercises monitored with Inerti{\ldots}   
          447       WESAD (Wearable Stress and Affect Detection)   
          280                                Wholesale customers   
          321                                            wiki4HE   
          273                                               Wilt   
          106                                               Wine   
          182                                       Wine Quality   
          406                       Wireless Indoor Localization   
          487  WISDM Smartphone and Smartwatch Activity and B{\ldots}   
          234                                Yacht Hydrodynamics   
          196                                  YearPredictionMSD   
          107                                              Yeast   
          215                 YouTube Comedy Slam Preference Dat   
          258              YouTube Multiview Video Games Dataset   
          364                            YouTube Spam Collection   
          396                                    Z-Alizadeh Sani   
          108                                                Zoo   
          
                                       Tipo de datos  \textbackslash{}
          462                           Multivariate   
          237                       Sequential, Text   
          301                           Multivariate   
          294                           Multivariate   
          0                             Multivariate   
          170                           Multivariate   
          427              Multivariate, Time-Series   
          260  Multivariate, Sequential, Time-Series   
          275    Univariate, Sequential, Time-Series   
          351  Multivariate, Sequential, Time-Series   
          411                             Sequential   
          181                           Multivariate   
          1                             Multivariate   
          371              Multivariate, Time-Series   
          345              Multivariate, Time-Series   
          279                           Multivariate   
          476                           Multivariate   
          208             Time-Series, Domain-Theory   
          207      Multivariate, Text, Domain-Theory   
          2                             Multivariate   
          3                                       NA   
          390                           Multivariate   
          358              Multivariate, Time-Series   
          405                           Multivariate   
          164                           Multivariate   
          4                             Multivariate   
          5                             Multivariate   
          6                             Multivariate   
          7                             Multivariate   
          457                           Multivariate   
          ..                                     {\ldots}   
          248                           Multivariate   
          257                          Domain-Theory   
          204                           Multivariate   
          206                            Time-Series   
          436                                   Text   
          139                                  Image   
          190               Multivariate, Sequential   
          103                           Multivariate   
          474                           Multivariate   
          104           Multivariate, Data-Generator   
          105           Multivariate, Data-Generator   
          241                             Sequential   
          363                           Multivariate   
          262                           Multivariate   
          447              Multivariate, Time-Series   
          280                           Multivariate   
          321                           Multivariate   
          273                           Multivariate   
          106                           Multivariate   
          182                           Multivariate   
          406                           Multivariate   
          487                           Multivariate   
          234                           Multivariate   
          196                           Multivariate   
          107                           Multivariate   
          215                                   Text   
          258                     Multivariate, Text   
          364                                   Text   
          396                                     NA   
          108                           Multivariate   
          
                               Tipo de tarea a resolver           Tipo de variables  \textbackslash{}
          462                            Classification                        Real   
          237                    Regression, Clustering                        Real   
          301                                Clustering                          NA   
          294                                Clustering                          NA   
          0                              Classification  Categorical, Integer, Real   
          170                          Causal-Discovery                     Integer   
          427                Classification, Clustering               Integer, Real   
          260                Classification, Clustering                          NA   
          275                Classification, Clustering                        Real   
          351                            Classification                        Real   
          411                            Classification                        Real   
          181                            Classification        Categorical, Integer   
          1                              Classification        Categorical, Integer   
          371                                Regression                        Real   
          345                                Regression                        Real   
          279                                Regression                        Real   
          476    Classification, Regression, Clustering                        Real   
          208  Regression, Clustering, Causal-Discovery                          NA   
          207                            Classification                        Real   
          2                              Classification  Categorical, Integer, Real   
          3                         Recommender-Systems                 Categorical   
          390                Classification, Clustering                        Real   
          358                                Regression                        Real   
          405                            Classification               Integer, Real   
          164                            Classification                        Real   
          4                              Classification  Categorical, Integer, Real   
          5                              Classification  Categorical, Integer, Real   
          6                              Classification                 Categorical   
          7                              Classification                 Categorical   
          457                            Classification                        Real   
          ..                                        {\ldots}                         {\ldots}   
          248                Classification, Clustering                     Integer   
          257                            Classification                     Integer   
          204                            Classification                        Real   
          206                            Classification                        Real   
          436                            Classification                          NA   
          139                            Classification                          NA   
          190                            Classification                        Real   
          103                                Clustering               Integer, Real   
          474                                Regression                        Real   
          104                            Classification                        Real   
          105                            Classification                        Real   
          241                            Classification               Integer, Real   
          363                            Classification                     Integer   
          262                            Classification                        Real   
          447                Classification, Regression                        Real   
          280                Classification, Clustering                     Integer   
          321  Regression, Clustering, Causal-Discovery                          NA   
          273                            Classification                          NA   
          106                            Classification               Integer, Real   
          182                Classification, Regression                        Real   
          406                            Classification                        Real   
          487                            Classification                     Integer   
          234                                Regression                        Real   
          196                                Regression                        Real   
          107                            Classification                        Real   
          215                            Classification                          NA   
          258                Classification, Clustering               Integer, Real   
          364                            Classification                          NA   
          396                            Classification               Integer, Real   
          108                            Classification        Categorical, Integer   
          
              Número de obs Número de variables   Año  \textbackslash{}
          462          7840                   5  2018   
          237        434874                   4  2013   
          301           150                   5  2014   
          294           399                   6  2014   
          0            4177                   8  1995   
          170           300                  43  2008   
          427           740                  21  2018   
          260          2747                  NA  2013   
          275            NA                  NA  2014   
          351         42240                   6  2016   
          411         75128                   9  2016   
          181           120                   6  2009   
          1           48842                  14  1996   
          371          9358                  15  2016   
          345          9358                  15  2016   
          279          1503                   6  2014   
          476           125                   8  2019   
          208         30000               20000  2011   
          207          1500               10000  2011   
          2             798                  38    NA   
          3           37711                 294  1998   
          390          7195                  22  2017   
          358         19735                  29  2017   
          405         60000                 171  2017   
          164           900               10000  2008   
          4             452                 279  1998   
          5            6000                   7  1992   
          6             226                  NA  1987   
          7             226                  69  1992   
          457           777                  18  2018   
          ..            {\ldots}                 {\ldots}   {\ldots}   
          248           403                   5  2013   
          257           306                   5  2013   
          204           310                   6  2011   
          206          3000                  27  2011   
          436         93600                1000  2018   
          139            NA                  NA    NA   
          190          5456                  24  2010   
          103           527                  38  1993   
          474        288000                  49  2019   
          104          5000                  21  1988   
          105          5000                  40  1988   
          241        165632                  18  2013   
          363          1353                  10  2016   
          262         39242                 152  2013   
          447      63000000                  12  2018   
          280           440                   8  2014   
          321           913                  53  2015   
          273          4889                   6  2014   
          106           178                  13  1991   
          182          4898                  12  2009   
          406          2000                   7  2017   
          487           120                  NA  2019   
          234           308                   7  2013   
          196        515345                  90  2011   
          107          1484                   8  1996   
          215       1138562                   3  2012   
          258        120000             1000000  2013   
          364          1956                   5  2017   
          396           303                  56  2017   
          108           101                  17  1990   
          
                                                           Links  \textbackslash{}
          462  https://archive.ics.uci.edu/ml/machine-learnin{\ldots}   
          237  https://archive.ics.uci.edu/ml/machine-learnin{\ldots}   
          301  https://archive.ics.uci.edu/ml/machine-learnin{\ldots}   
          294  https://archive.ics.uci.edu/ml/machine-learnin{\ldots}   
          0    https://archive.ics.uci.edu/ml/machine-learnin{\ldots}   
          170  https://archive.ics.uci.edu/ml/machine-learnin{\ldots}   
          427  https://archive.ics.uci.edu/ml/machine-learnin{\ldots}   
          260  https://archive.ics.uci.edu/ml/machine-learnin{\ldots}   
          275  https://archive.ics.uci.edu/ml/machine-learnin{\ldots}   
          351  https://archive.ics.uci.edu/ml/machine-learnin{\ldots}   
          411  https://archive.ics.uci.edu/ml/machine-learnin{\ldots}   
          181  https://archive.ics.uci.edu/ml/machine-learnin{\ldots}   
          1    https://archive.ics.uci.edu/ml/machine-learnin{\ldots}   
          371  https://archive.ics.uci.edu/ml/machine-learnin{\ldots}   
          345  https://archive.ics.uci.edu/ml/machine-learnin{\ldots}   
          279  https://archive.ics.uci.edu/ml/machine-learnin{\ldots}   
          476  https://archive.ics.uci.edu/ml/machine-learnin{\ldots}   
          208  https://archive.ics.uci.edu/ml/machine-learnin{\ldots}   
          207  https://archive.ics.uci.edu/ml/machine-learnin{\ldots}   
          2    https://archive.ics.uci.edu/ml/machine-learnin{\ldots}   
          3    https://archive.ics.uci.edu/ml/machine-learnin{\ldots}   
          390  https://archive.ics.uci.edu/ml/machine-learnin{\ldots}   
          358  https://archive.ics.uci.edu/ml/machine-learnin{\ldots}   
          405  https://archive.ics.uci.edu/ml/machine-learnin{\ldots}   
          164  https://archive.ics.uci.edu/ml/machine-learnin{\ldots}   
          4    https://archive.ics.uci.edu/ml/machine-learnin{\ldots}   
          5    https://archive.ics.uci.edu/ml/machine-learnin{\ldots}   
          6    https://archive.ics.uci.edu/ml/machine-learnin{\ldots}   
          7    https://archive.ics.uci.edu/ml/machine-learnin{\ldots}   
          457  https://archive.ics.uci.edu/ml/machine-learnin{\ldots}   
          ..                                                 {\ldots}   
          248  https://archive.ics.uci.edu/ml/machine-learnin{\ldots}   
          257  https://archive.ics.uci.edu/ml/machine-learnin{\ldots}   
          204  https://archive.ics.uci.edu/ml/machine-learnin{\ldots}   
          206  https://archive.ics.uci.edu/ml/machine-learnin{\ldots}   
          436  https://archive.ics.uci.edu/ml/machine-learnin{\ldots}   
          139  https://archive.ics.uci.edu/ml/machine-learnin{\ldots}   
          190  https://archive.ics.uci.edu/ml/machine-learnin{\ldots}   
          103  https://archive.ics.uci.edu/ml/machine-learnin{\ldots}   
          474  https://archive.ics.uci.edu/ml/machine-learnin{\ldots}   
          104  https://archive.ics.uci.edu/ml/machine-learnin{\ldots}   
          105  https://archive.ics.uci.edu/ml/machine-learnin{\ldots}   
          241  https://archive.ics.uci.edu/ml/machine-learnin{\ldots}   
          363  https://archive.ics.uci.edu/ml/machine-learnin{\ldots}   
          262  https://archive.ics.uci.edu/ml/machine-learnin{\ldots}   
          447  https://archive.ics.uci.edu/ml/machine-learnin{\ldots}   
          280  https://archive.ics.uci.edu/ml/machine-learnin{\ldots}   
          321  https://archive.ics.uci.edu/ml/machine-learnin{\ldots}   
          273  https://archive.ics.uci.edu/ml/machine-learnin{\ldots}   
          106  https://archive.ics.uci.edu/ml/machine-learnin{\ldots}   
          182  https://archive.ics.uci.edu/ml/machine-learnin{\ldots}   
          406  https://archive.ics.uci.edu/ml/machine-learnin{\ldots}   
          487  https://archive.ics.uci.edu/ml/machine-learnin{\ldots}   
          234  https://archive.ics.uci.edu/ml/machine-learnin{\ldots}   
          196  https://archive.ics.uci.edu/ml/machine-learnin{\ldots}   
          107  https://archive.ics.uci.edu/ml/machine-learnin{\ldots}   
          215  https://archive.ics.uci.edu/ml/machine-learnin{\ldots}   
          258  https://archive.ics.uci.edu/ml/machine-learnin{\ldots}   
          364  https://archive.ics.uci.edu/ml/machine-learnin{\ldots}   
          396  https://archive.ics.uci.edu/ml/machine-learnin{\ldots}   
          108  https://archive.ics.uci.edu/ml/machine-learnin{\ldots}   
          
                                          Descripción base datos  
          462  Measurement of the S21,consists of 10 sweeps, {\ldots}  
          237  3D road network with highly accurate elevation{\ldots}  
          301  This data set compromises the metadata for the{\ldots}  
          294  This data set compromises the metadata for the{\ldots}  
          0    Predict the age of abalone from physical measu{\ldots}  
          170  The objective is to determine the set of boole{\ldots}  
          427  The database was created with records of absen{\ldots}  
          260  This dataset comprises information regarding t{\ldots}  
          275  The dataset collects data from a wearable acce{\ldots}  
          351  This dataset contains temporal data from a Wir{\ldots}  
          411  Sequential motion data from 14 healthy older p{\ldots}  
          181  The data was created by a medical expert as a {\ldots}  
          1    Predict whether income exceeds \$50K/yr based o{\ldots}  
          371  Contains the responses of a gas multisensor de{\ldots}  
          345  Contains the responses of a gas multisensor de{\ldots}  
          279  NASA data set, obtained from a series of aerod{\ldots}  
          476  Five different QCM gas sensors are used, and f{\ldots}  
          208  Amazon's InfoSec is getting smarter about the {\ldots}  
          207  The dataset is used for authorship identificat{\ldots}  
          2                                 Steel annealing data  
          3    Log of anonymous users of www.microsoft.com; p{\ldots}  
          390  Acoustic features extracted from syllables of {\ldots}  
          358  Experimental data used to create regression mo{\ldots}  
          405  The datasets' positive class consists of compo{\ldots}  
          164  ARCENE's task is to distinguish cancer versus {\ldots}  
          4    Distinguish between the presence and absence o{\ldots}  
          5    Dataset artificially generated by using first {\ldots}  
          6                Nominal audiology dataset from Baylor  
          7    Standardized version of the original audiology{\ldots}  
          457  Exhaustive one year non-confidential data in t{\ldots}  
          ..                                                 {\ldots}  
          248  It is the real dataset about the students' kno{\ldots}  
          257  Data used for USPTO Algorithm Competition. Con{\ldots}  
          204  Data set containing values for six biomechanic{\ldots}  
          206  The Physical Action Data Set includes 10 norma{\ldots}  
          436  To create the largest authorship attribution d{\ldots}  
          139  The JARtool project was a pioneering effort to{\ldots}  
          190  The data were collected as the SCITOS G5 robot{\ldots}  
          103               Multiple classes predict plant state  
          474  This data set consists of positions and absorb{\ldots}  
          104                       CART book's waveform domains  
          105                       CART book's waveform domains  
          241  A dataset with 5 classes (sitting-down, standi{\ldots}  
          363                                                     
          262  Six young health subjects were asked to perfor{\ldots}  
          447  WESAD (Wearable Stress and Affect Detection) c{\ldots}  
          280  The data set refers to clients of a wholesale {\ldots}  
          321  Survey of faculty members from two Spanish uni{\ldots}  
          273  High-resolution Remote Sensing data set (Quick{\ldots}  
          106  Using chemical analysis determine the origin o{\ldots}  
          182  Two datasets are included, related to red and {\ldots}  
          406  Collected in indoor space by observing signal {\ldots}  
          487  Contains accelerometer and gyroscope time-seri{\ldots}  
          234  Delft data set, used to predict the hydodynami{\ldots}  
          196  Prediction of the release year of a song from {\ldots}  
          107  Predicting the Cellular Localization Sites of {\ldots}  
          215  This dataset provides user vote data on which {\ldots}  
          258  This dataset contains about 120k instances, ea{\ldots}  
          364  It is a public set of comments collected for s{\ldots}  
          396                It was collected for CAD diagnosis.  
          108                   Artificial, 7 classes of animals  
          
          [488 rows x 9 columns]
\end{Verbatim}
            

    % Add a bibliography block to the postdoc
    
    
    
    \end{document}
